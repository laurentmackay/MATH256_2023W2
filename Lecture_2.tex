\documentclass[11pt, dvipsnames, handout]{beamer}
\newtoggle{full}
\settoggle{full}{true}

\newtoggle{covered}
\settoggle{covered}{false}

\newtoggle{presentable}
\settoggle{presentable}{false}

\newtoggle{dualscreen}
\settoggle{dualscreen}{false}

\usepackage{pgfplots}
%\pgfplotsset{compat = newest}

\usepackage{pgfpages}

\setbeamertemplate{note page}{\pagecolor{yellow!5}\vfill \insertnote \vfill}
\usepackage{collect}
\definecollection{notes}
\newcounter{notestaken}

\usepackage{xpatch}

\usepackage{ulem}

\usepackage[framemethod=tikz]{mdframed}

\usepackage{scalerel}
\usepackage{calc}

%\usepackage{enumitem}
\setlength\fboxsep{.2em}

\usepackage{graphicx} % Allows including images
\usepackage{booktabs} % Allows the use of \toprule, \midrule and \bottomrule in tables

\xpatchcmd{\itemize}
  {\def\makelabel}
  {\setlength{\itemsep}{0.65 em}\def\makelabel}
  {}
  {}


\xpatchcmd{\beamer@enum@}
  {\def\makelabel}
  {\setlength{\itemsep}{0.65 em}\def\makelabel}
  {}
  {}


%\makeatletter
%\renewcommand{\itemize}[1][]{%
%  \beamer@ifempty{#1}{}{\def\beamer@defaultospec{#1}}%
%  \ifnum \@itemdepth >2\relax\@toodeep\else
%    \advance\@itemdepth\@ne
%    \beamer@computepref\@itemdepth% sets \beameritemnestingprefix
%    \usebeamerfont{itemize/enumerate \beameritemnestingprefix body}%
%    \usebeamercolor[fg]{itemize/enumerate \beameritemnestingprefix body}%
%    \usebeamertemplate{itemize/enumerate \beameritemnestingprefix body begin}%
%    \list
%      {\usebeamertemplate{itemize \beameritemnestingprefix item}}
%      {%
%        \setlength\topsep{1em}%NEW
%        \setlength\partopsep{1em}%NEW
%        \setlength\itemsep{1em}%NEW
%        \def\makelabel##1{%
%          {%
%            \hss\llap{{%
%                \usebeamerfont*{itemize \beameritemnestingprefix item}%
%                \usebeamercolor[fg]{itemize \beameritemnestingprefix item}##1}}%
%          }%
%        }%
%      }
%  \fi%
%  \beamer@cramped%
%  \raggedright%
%  \beamer@firstlineitemizeunskip%
%}
%
%
%
%
%
%\makeatother

%\setlist[beamer@enum@]{topsep=1 em}
%\let\origcheckmark\checkmark %screw you dingbat
%\let\checkmark\undefined %screw you dingbat
%\usepackage{dingbat} 
%\let\checkmark\origcheckmark %screw you dingbat






%\usepackage{fontawesome}

\usepackage{mathtools}
\usepackage{etoolbox, calculator}

\usepackage{xcolor}
\usepackage{tikz}
\usetikzlibrary{arrows.meta}
\usetikzlibrary{calc}
\usepackage[nomessages]{fp}
\usepackage{transparent}
\usepackage{accsupp}
%\usepackage{color, xcolor}

%colorblind-friendly palette
%\definecolor{dblue}{RGB}{51,34,136}
\definecolor{lblue}{RGB}{136,204,238}
%\definecolor{green}{RGB}{17,119,51}
\definecolor{tan}{RGB}{221,204,119}
%\definecolor{mauve}{RGB}{204,102,119}

\usepackage{tcolorbox}



\usepackage{xifthen}
\usepackage{nicefrac}
\usepackage{amsmath}
\usepackage{amsthm}
\usepackage{amssymb}
\theoremstyle{definition}
\newtheorem*{define}{Definition}
\newtheorem*{recall}{Recall}


\DeclareMathOperator{\tr}{tr}

\usepackage{multicol}
%\setlength{\columnsep}{1cm}

\usepackage{tablists, amsmath,vwcol, cancel, polynom}
\usetikzlibrary{shapes, patterns, decorations.shapes}
%\usepackage{tikzpeople}
\tikzstyle{vertex}=[shape=circle, minimum size=2mm, inner sep=0, fill]
\tikzstyle{opendot}=[shape=circle, minimum size=2mm, inner sep=0, fill=white, draw]

% common math quick commands
\newcommand{\nicedd}[2]{\nicefrac{\text{d}#1}{\text{d}#2}}
\newcommand{\dd}[2]{\dfrac{\text{d}#1}{\text{d}#2}}
\newcommand{\pd}[2]{\dfrac{\partial #1}{\partial#2}}
\renewcommand{\d}[1]{\text{d}#1}
\newcommand{\ddn}[3]{\dfrac{\text{d}^{#3}#1}{\text{d}#2^{#3}}}
\newcommand{\pdn}[3]{\dfrac{\partial^{#3}#1}{\partial#2^{#3}}}
\newcommand{\p}[0]{^{\prime}}
\newcommand{\pp}[0]{^{\prime\prime}}
\newcommand{\op}[2][\text{L}]{#1 \left[ #2 \right]}

\newcommand{\lap}[1]{\mathcal{L}\left\{#1\right\}}
\newcommand{\lapinv}[1]{\mathcal{L}^{-1}\left\{#1\right\}}
\newcommand{\lapint}[1]{\int_0^\infty e^{-st}#1dt}
\newcommand{\evalat}[2]{\Big|_{#1}^{#2}}

\newcommand{\paren}[1]{ \left( #1 \right)}

\newcommand{\haxis}[4][\normcolor]{\draw[#1, <->] (-#2,0)--(#3,0) node[right]{$#4$}; }

\newcommand{\circled}[1]{\raisebox{.5pt}{\textcircled{\raisebox{-.9pt} {#1}}}}
\newcommand{\axis}[4]{\draw[\normcolor, <->] (-#1,0)--(#2,0) 
node[right]{$x$};
\draw[help lines, <->] (0,-#3)--(0,#4) node[above]{$y$};}

\newcommand{\laxis}[6]{\draw[<->] (-#1,0)--(#2,0) 
node[right]{$#5$};
\draw[ <->] (0,-#3)--(0,#4) node[above]{$#6$};}
\newcommand{\xcoord}[2]{
	\draw (#1,.2)--(#1,-.2) node[below]{$#2$};}
\newcommand{\textnode}[3]{
	\draw (#1,#2) node[below]{$#3$};}
	
\newcommand{\nxcoord}[2]{
	\draw (#1,-.2)--(#1,.2) node[above]{$#2$};}
\newcommand{\ycoord}[2]{
	\draw (.2,#1)--(-.2,#1) node[left]{$#2$};}
\newcommand{\nycoord}[2]{
	\draw (-.2,#1)--(.2,#1) node[right]{$#2$};}
\newcommand{\dlim}{\displaystyle\lim}
\newcommand{\dlimx}[1]{\displaystyle\lim_{x \rightarrow #1}}
\newcommand{\stickfig}[2]{
	\draw (#1,#2) arc(-90:270:2mm);
	\draw (#1,#2)--(#1,#2-.5) (#1-.25,#2-.75)--(#1,#2-.5)--(#1+.25,#2-.75) (#1-.2,#2-.2)--(#1+.2,#2-.2);}	

%\newcounter{example}
%\setcounter{example}{1}
%\newcounter{preFrameExample}
%\AtBeginEnvironment{frame}{\setcounter{preFrameExample}{\value{example}}}
%\newcommand{\ex}[1]{
%	 \setcounter{example}{\value{preFrameExample}}
%	 \textcolor{green}{\small\fbox{Example \arabic{example}: #1}}\\[8pt]
%	\stepcounter{example}}
%\newcommand{\exans}[1]{
%	\SUBTRACT{\value{preFrameExample}}{1}{\n}
%	 \textcolor{green}{\small\fbox{Solution \n: #1}}\\[8pt]}
\mode<presentation> {

% The Beamer class comes with a number of default slide themes
% which change the colors and layouts of slides. Below this is a list
% of all the themes, uncomment each in turn to see what they look like.


\usetheme{CambridgeUS}
\usecolortheme[named=black]{structure}


\newcommand{\studentcolor}[0]{ForestGreen}
\newcommand{\normcolor}[0]{NavyBlue}
\newcommand{\alertcolor}{Red}

\setbeamercolor{normal text}{fg=\normcolor}
\setbeamercolor{frametitle}{fg=\normcolor}
\setbeamercolor{section in head/foot}{fg=Black, bg=Gray!20}
\setbeamercolor{subsection in head/foot}{fg=Green!70!Black, bg=Gray!10}
\setbeamercolor{alerted text}{fg=\alertcolor}
\setbeamerfont{alerted text}{series=\bf}
\setbeamertemplate{enumerate items}[default]
\setbeamercolor{enumerate item}{fg=\normcolor}

\setbeamertemplate{footline} % To remove the footer line in all slides uncomment this line
%\setbeamertemplate{footline}[page number] % To replace the footer line in all slides with a simple slide count uncomment this line

\setbeamertemplate{navigation symbols}{} % To remove the navigation symbols from the bottom of all slides uncomment this line
}

\newcommand{\alertbox}[1]{\tcbox[on line, colframe=\alertcolor, colback=White, left=2pt,right=2pt,top=2pt,bottom=2pt]{\usebeamercolor*{normal text}#1}}


\newcommand{\startstu}{\setbeamercolor{normal text}{fg=\studentcolor}\usebeamercolor*{normal text}\setbeamercolor{enumerate item}{fg=\studentcolor}\usebeamercolor*{enumerate item}}
\newcommand{\stopstu}{\setbeamercolor{normal text}{fg=\normcolor}\usebeamercolor*{normal text}\setbeamercolor{enumerate item}{fg=\normcolor}\usebeamercolor*{enumerate item}}

\newcommand{\takenote}[1]{ \begin{collect}{notes}{}{}{}{}  #1  \end{collect}  \addtocounter{notestaken}{1}} %\ifthenelse{\value{notestaken}>0}{\hrulefill\\}{}

\makeatletter
\newcommand{\cover}{\alt{\beamer@makecovered}{\beamer@fakeinvisible}}
\newcommand{\ucover}[1]{\iftoggle{full}{}{\beamer@endcovered} \stopstu #1\startstu \iftoggle{full}{}{\beamer@startcovered} }
%\newcommand{\ucover}[1]{\beamer@endcovered \stopstu #1\startstu \beamer@startcovered }
\makeatother

\newcommand{\skippause}{ \addtocounter{beamerpauses}{-1}}
\newcommand{\blockpres}{ \skippause \pause }

\newcommand{\studentify}[1]{\startstu #1  \stopstu }
\newcommand{\student}[1]{\iftoggle{full}{ \pause  \studentify{#1} }{\iftoggle{covered}{\studentify{#1}}{\cover{  #1 }}}}
\newcommand{\cstudent}[1]{\student{\begin{center} #1 \end{center}}}
\newcommand{\fullonly}[1]{\iftoggle{full}{ #1}{}}
\newcommand{\presentonly}[1]{\iftoggle{presentable}{ #1}{}}

\usepackage{xparse}
\usepackage{xifthen}

% shortcuts for commonly-used presentation elements
%\NewDocumentCommand{\slide}{o m}
% {\IfValueTF{#1}{\begin{frame}[t]{#1}}{\begin{frame}[t]} #2 \end{frame}}

\newtoggle{iscovered}

\newcommand{\slide}[2][]{%
%\setcounter{notestaken}{0}
\takenote{#2} 
%\ifthenelse{\equal{#1}{}}{\begin{frame}[t]}{\begin{frame}[t]{#1}} #2 \ifthenelse{\value{notestaken}>0}{ \note{\includecollection{notes}}}{} \end{frame}%
\ifthenelse{\equal{#1}{}}{\begin{frame}[t]}{\begin{frame}[t]{#1}} #2 \iftoggle{covered}{\settoggle{iscovered}{true}}{\settoggle{iscovered}{false}}  \note{ \iftoggle{iscovered}{}{\settoggle{covered}{true}} #2 \iftoggle{iscovered}{}{\settoggle{covered}{false}} } \end{frame}%
%\setcounter{notestaken}{0}
}
\newcommand{\defn}[2][]{%
 \setcounter{listcounter}{0}%
\ifthenelse{\equal{#1}{}}{\begin{block}{Definition}}{\begin{block}{#1 :}}%
 #2 \vspace{0.25em} \ifthenelse{\value{listcounter}>0}{\skippause}{} \pause \end{block}%
}



\newcommand{\arr}[2]{\begin{array}{#1}#2\end{array}}
\newcommand{\mat}[2]{\left[\arr{#1}{#2}\right]}
\newcommand{\carray}[1]{\arr{c}{#1}}
\newcommand{\larray}[1]{\arr{l}{#1}}
\newcommand{\rarray}[1]{\arr{r}{#1}}
\newcommand{\colvec}[1]{\mat{c}{#1}}

\newcommand{\itmz}[1]{\addtocounter{listcounter}{1} \begin{itemize}#1 \end{itemize} }
\newcommand{\subitem}[1]{\addtocounter{listcounter}{1} \begin{itemize} \item #1 \end{itemize}}
%
\newcommand{\enum}[1]{\addtocounter{listcounter}{1} \begin{enumerate} #1  \end{enumerate}  }


\newcommand{\algnlbl}[1]{\begin{align}#1  \end{align}} 
\newcommand{\algn}[1]{\begin{align*}#1  \end{align*}} 
\newcommand{\lgn}[1]{ \action<+->{#1} }
\newcommand{\slgn}[1]{\iftoggle{full}{\action<+->{ \startstu #1 \stopstu}}{ \cover{ #1 } } \takenote{$#1$}}

\newcommand{\chckmrk}{\alert{\checkmark}}

\usepackage{pifont}
\newcommand{\xmark}{\alert{\text{\large \ding{55}}}}

\newcommand{\return}[0]{\raisebox{.5ex}{\rotatebox[origin=c]{180}{$\Lsh$}}}
\usepackage{pbox}
%\newcommand{\ex}[1]{\rotatebox[origin=c]{10}{\uline{ex}}:$\;$\pbox[t][][b]{0.9\linewidth}{#1}}
\newcommand{\ex}[1]{\uline{ex}:$\;$\pbox[t][][t]{0.9\linewidth}{#1}}
\newcommand{\eg}[1]{e.g.,$\;$\pbox[t][][t]{0.9\linewidth}{#1}}
\newcommand{\tikzplot}[8][]{%
\begin{tikzpicture}

\begin{scope}[]%
\clip(-#2,-#4) rectangle (#3,#5);%
#8%
\end{scope}%
\laxis{#2}{#3}{#4}{#5}{#6}{#7}%
#1
\end{tikzpicture}%
}


\newcommand{\cancelslide}[1]{%
\begingroup%
\setbeamertemplate{background canvas}{%
\begin{tikzpicture}[remember picture,overlay]%
\draw[line width=2pt,red!60!black] %
  (current page.north west) -- (current page.south east);%
\draw[line width=2pt,red!60!black] %
  (current page.south west) -- (current page.north east);%
\end{tikzpicture}}%
#1%
\endgroup%
}
\renewcommand{\CancelColor}{\color{red}}
\newcommand{\twocols}[3][0.5]{\begin{columns}\begin{column}{#1\textwidth}#2\end{column}\hspace{1em}\vrule{}\hspace{1em}\begin{column}{#1\textwidth}#3\end{column}\end{columns}}

\newcommand{\twomini}[5][1]{\calculatespace \begin{minipage}[t]{\columnwidth}\begin{minipage}[][#1\contentheight][t]{#2\columnwidth}#4\end{minipage}\hfill\begin{minipage}[][#1\contentheight][t]{#3\columnwidth}#5\end{minipage}\end{minipage}}

\newcommand{\threemini}[7][1]{\calculatespace \begin{minipage}[t]{\columnwidth}\begin{minipage}[][#1\contentheight][t]{#2\columnwidth}#5\end{minipage}\hfill\begin{minipage}[][#1\contentheight][t]{#4\columnwidth}#6\end{minipage}\hfill\begin{minipage}[][#1\contentheight][t]{#3\columnwidth}#7\end{minipage}\end{minipage}}


\newcounter{listcounter}
\setcounter{listcounter}{0}



\newif\ifsidebartheme
\sidebarthemetrue

\newdimen\contentheight
\newdimen\contentwidth
\newdimen\contentleft
\newdimen\contentbottom
\makeatletter
\newcommand*{\calculatespace}{%
\contentheight=\paperheight%
\ifx\beamer@frametitle\@empty%
    \setbox\@tempboxa=\box\voidb@x%
  \else%
    \setbox\@tempboxa=\vbox{%
      \vbox{}%
      {\parskip0pt\usebeamertemplate***{frametitle}}%
    }%
    \ifsidebartheme%
      \advance\contentheight by-1em%
    \fi%
  \fi%
\advance\contentheight by-\ht\@tempboxa%
\advance\contentheight by-\dp\@tempboxa%
\advance\contentheight by-\beamer@frametopskip%
\ifbeamer@plainframe%
\contentbottom=0pt%
\else%
\advance\contentheight by-\headheight%
\advance\contentheight by\headdp%
\advance\contentheight by-\footheight%
\advance\contentheight by4pt%
\contentbottom=\footheight%
\advance\contentbottom by-4pt%
\fi%
\contentwidth=\paperwidth%
\ifbeamer@plainframe%
\contentleft=0pt%
\else%
\advance\contentwidth by-\beamer@rightsidebar%
\advance\contentwidth by-\beamer@leftsidebar\relax%
\contentleft=\beamer@leftsidebar%
\fi%
}
\makeatother


\iftoggle{dualscreen}{\setbeameroption{show notes on second screen=right}}{}
%\settoggle{covered}{true}
\begin{document}


\section{Lecture 2}

\subsection{Review of integration}

\slide[Review: Evaluating Integrals]{
Indefinite integrals OR antiderivatives \[F(t)  = \int f(t) dt\]
\student{Only defined up to an (arbitrary) additive constant \[\dd{}{t} \left[F(t)+C\right] = f(t)\]
%\centerline{ arbitrary constant $\Rightarrow$ a \uline{general solution}}.

 }
\vfill
Definite integrals \algn{\int_{t_0}^t f(s) ds &=\student{ F(t)+\cancel{C}-\left(F(t_0) + \cancel{C}\right)}\\&= \student{F(t)-F(t_0)}}
%\student{\centerline{no arbitrary constant $\Rightarrow$ a \uline{particular solution}}}

}

\slide[Intuitive approach to integrating a DE]{
Suppose we have $y\p=f(t)$ with an initial condition $y(0)=y_0$, then we can find the \uline{general} solution with \[ y(t) =\intop f(t) \text{d}t= F(t) + C,\]  and then use the initial condition to find the \uline{particular} solution.

\vfill

\ex{Solve $y\p=e^{-3t}$ with $y(0)=2.$}
\student{\algn{y(t) &= \int e^{-3t}dt = -\frac13e^{-3t}+C &\text{\uline{general} solution}\\
y(0)=2&=-\frac13+C \implies C=\frac73\\
y(t)&=\frac73 -\frac13e^{-3t} &\text{\uline{particular} solution}
}
Particular solutions have NO arbitrary constants.
}

}


\slide[Integration becomes harder once we have $y'=f(y,t)$. ]{
\ex{Solve $y'=2y^2t$ with y(0)=1/4. }
\vfill
\uline{hint}: $\dd{y}{t}=g(y)h(t) \Leftrightarrow \frac{\text{d}y}{g(y)} = h(t)dt $. \hfill(seperable equation)
\vfill
\student{
\algn{
\frac{\text{d}y}{y^2} & =2t\text{d}t   & \intop \frac{\text{d}y}{y^2} & = 2 \intop t\text{d}t\\
\Rightarrow -\frac1y&=t^2+C &  \frac1y&=- t^2+C \\
y&=\frac{1}{C-t^2}\intertext{initial condition $y(0)=1/4$}
y(0)&=\frac1C=\frac14 \Rightarrow C=4 \Aboxed{y(t)=\frac{1}{4-t^2}}
}
Solution blows up to $+\infty$ at $t=2$.
}

}


\slide[Integration becomes harder once we have $f=f(y,t)$. ]{ 
\ex{Try to solve $y\p=-ay+t^3$ where $a$ is some constant.}
\vfill
\student{
Lets try a similar strategy
\algn{\dd{y}{t}+ay&=t^3\\
\text{d}y + ay \text{d}t& =t^3dt\\
y(t) +a\intop y(t)\text{d}t &=\frac{t^4}{4}+C}\vfill
\centerline{We dont know $y(t)$, so can't integrate it. \hfill $\Rightarrow$ \hfill Need more tricks}\vfill
}
}



\subsection{Existence and uniqueness}

\slide[Can we always integrate $y\p=f(y,t)$?]{
Potential issues:
\itmz{\item Does the integral exist?
\student{\subitem{Does not mean: Can you solve the integral? \item Means: Does the derivative have a well-defined value for all time?
\item e.g., $y\p = \frac{y}{t-1}$  is undefined at $t=1$.\subitem{Solutions may not be defined for all values of the indep. variable.}
}
}\vfill
\item Is there only one solution $y(t)$?
\student{\subitem{Yes if and only if: \enum{\item $f(y,t)$ is well-defined and \item  differentiable in $y$ everywhere along the solution $y(t)$.} \item Need to specify initial conditions to get a unique particular solution.}}

}
}
\subsection{Classifying DEs and their Solution Structure}
\slide[Classifying differential equations]{
\itmz{ \item \uline{Linear} first-order:  $\quad y\p+p(t) y = g(t); \qquad y(t_0)=y_0$\vfill
\subitem{\student{Linear  with respect to $y$, $y\p$, $y\pp \dots$, $p(t)$doesn't matter for linearity.}\vfill

\item \uline{Homogeneous}:\student{$g(t)=0$}\vfill
\item \uline{Inhomogeneous}:\student{$g(t)$ is not zero everywhere.}\vfill
\item Constant coefficient $p(t)=a$, with constant $a$.
\student{\vfill\subitem{Solvable if $g(t)$ is ``nice'', and has unique solutions.}\vfill}

}\vfill
}
}

\slide[Solution Structure of Linear DEs \hfill ]{
\vspace{-1em}

The linear ODE \vspace{-.25em}\[\op{y}=g(t) \hspace{3em} \text{ \footnotesize\ex{$y'+ay=t^3$ is a 1$^{\rm st}$ order linear ODE}}\]


always has the \uline{general} solution structure
\algn{y(t)&= Cy_h(t) +y_p(t) & \text{where $y_h$ is the \uline{general} solution to } \op{y_h}=0 \\
&&\text{(i.e., the \alert{associated homogeneous problem}),}\phantom{\underset{A}{=}}\\
&&\text{$y_p$ is a \uline{particular} solution to }  \op{y_p}=g(t),\phantom{\underset{A}{=}}\\
&&\text{and $C $ is any constant.}}


\vfill

\uline{Proof:}

\vfill
\student{
\algn{\op{y(t)}=\op{Cy_h+y_p} &\overset{\text{Linearity 1}}{=} \op{Cy_h}+\op{y_p}\\
&\overset{\text{Linearity 2}}{=} C\underbrace{\op{y_h}}_{=0}+\underbrace{\op{y_p}}_{=g(t)}=g(t)
}
}\vfill
}



\subsection{First-Order DEs with Constant Coefficients}

\slide[First-Order Homogeneous Constant Coefficient Equations]{\vspace{-1em}
\[ y\p+ay=0 \] where $a\neq 0$ is a constant.
\vfill
General solution:
\student{
\algn{\dd{y}{t} + a y &= 0 & \Rightarrow \dd{y}{t} &= -ay \\  \frac1y \dd{y}{t} &= -a  &\Rightarrow \int \frac{\text{d}y}{y}& = -\int a \text{d}t \\\\
\ln{|y|} &= -at+C_1 & \Rightarrow e^{\ln{|y|}} &= e^{-at+C_1}\\
|y|&=e^{-at}\underbrace{e^{C_1}}_{C_2} &\Rightarrow \quad  y=&\underbrace{\pm C_2}_{C} e^{-at}  \\
}\vspace{-2em}
\[\boxed{y(t) = Ce^{-at}}}\]
\vfill
}

\slide{
The ODE \[y'+2y = t\] has a \uline{particular} solution given by  $y_p=\frac12 t - \frac14$.
\vfill
Find the general solution, and then solve the following IVP: \[y'+2y = t, \quad \text{with } y(0)=-1.\] 
\student{General Solution:
\[y(t)=Ce^{-2t} + \frac12 t -\frac14 \]
Impose the initial condtion constraint $y(0)=-1$
\[y(0) = C -\frac14  = -1\qquad \Rightarrow C = -\frac34\]
\[\boxed{y(t) = -\frac34 e^{-2t} +  \frac12 t -\frac14 }\]
\vfill
\uline{Note:} $y_p$ alone can only solve an IVP with $y(0)=-1/4$
}

}

\slide{\vspace{-.75em}
The ODE \[\frac13 y' = y +\frac13 e^{3t}\] has a \uline{particular} solution given by  $y_p=t e^{3t}$.
\vfill
Find the general solution, and then solve the following IVP: \[\frac13 y' = y +\frac13 e^{3t}, \quad \text{with } y(0)=-1.\]
\vfill
\student{
Re-arrange DE:
\[y'-3y = e^{3t}\qquad \Rightarrow \qquad \text{Gen. Soln. }y(t)=Ce^{3t} + te^{3t}\]
 initial condtion  $y(0)=-1$
\[y(0) = C  = -1 \qquad \Rightarrow \qquad \boxed{y(t) = - e^{3t} + te^{3t} } \]\vfill
\uline{Note:} $y_p$ alone can only solve an IVP with $y(0)=0$
}

}


\subsection{Summary}
\slide[Summary of Lecture 2]{
\itmz{\item Integrating $y'=f(y,t)$ :

\itmz{
\item Indefinite integral (antiderivative) gives general solution.
\vfill
\item Can we always integrate a 1st order DE? \subitem{No, but we avoid those cases.}}
\vfill
\item Linear constant coefficient DEs: $y\p + a y = g(t)$\vfill
\itmz{
\item Solving the homogeneous problem: $y_h\p+ay_h=0 \Rightarrow y_h=Ce^{-at}$

\vfill
\subitem{General form of inhomogeneous solution: $y=y_h+y_p$ \vfill \item $y_p$ alone can only solve a unique IVP, but not the general case.}
\vfill
\item How do we find $y_p(t)$?
\vfill
\subitem{ ~\student{Next Lecture: Method of Undetermined Coefficients}}
}
}
}


\end{document}

