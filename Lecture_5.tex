\documentclass[11pt, dvipsnames, handout]{beamer}
\newtoggle{full}
\settoggle{full}{true}

\newtoggle{covered}
\settoggle{covered}{false}

\newtoggle{presentable}
\settoggle{presentable}{false}

\newtoggle{dualscreen}
\settoggle{dualscreen}{false}

\usepackage{pgfplots}
%\pgfplotsset{compat = newest}

\usepackage{pgfpages}

\setbeamertemplate{note page}{\pagecolor{yellow!5}\vfill \insertnote \vfill}
\usepackage{collect}
\definecollection{notes}
\newcounter{notestaken}

\usepackage{xpatch}

\usepackage{ulem}

\usepackage[framemethod=tikz]{mdframed}

\usepackage{scalerel}
\usepackage{calc}

%\usepackage{enumitem}
\setlength\fboxsep{.2em}

\usepackage{graphicx} % Allows including images
\usepackage{booktabs} % Allows the use of \toprule, \midrule and \bottomrule in tables

\xpatchcmd{\itemize}
  {\def\makelabel}
  {\setlength{\itemsep}{0.65 em}\def\makelabel}
  {}
  {}


\xpatchcmd{\beamer@enum@}
  {\def\makelabel}
  {\setlength{\itemsep}{0.65 em}\def\makelabel}
  {}
  {}


%\makeatletter
%\renewcommand{\itemize}[1][]{%
%  \beamer@ifempty{#1}{}{\def\beamer@defaultospec{#1}}%
%  \ifnum \@itemdepth >2\relax\@toodeep\else
%    \advance\@itemdepth\@ne
%    \beamer@computepref\@itemdepth% sets \beameritemnestingprefix
%    \usebeamerfont{itemize/enumerate \beameritemnestingprefix body}%
%    \usebeamercolor[fg]{itemize/enumerate \beameritemnestingprefix body}%
%    \usebeamertemplate{itemize/enumerate \beameritemnestingprefix body begin}%
%    \list
%      {\usebeamertemplate{itemize \beameritemnestingprefix item}}
%      {%
%        \setlength\topsep{1em}%NEW
%        \setlength\partopsep{1em}%NEW
%        \setlength\itemsep{1em}%NEW
%        \def\makelabel##1{%
%          {%
%            \hss\llap{{%
%                \usebeamerfont*{itemize \beameritemnestingprefix item}%
%                \usebeamercolor[fg]{itemize \beameritemnestingprefix item}##1}}%
%          }%
%        }%
%      }
%  \fi%
%  \beamer@cramped%
%  \raggedright%
%  \beamer@firstlineitemizeunskip%
%}
%
%
%
%
%
%\makeatother

%\setlist[beamer@enum@]{topsep=1 em}
%\let\origcheckmark\checkmark %screw you dingbat
%\let\checkmark\undefined %screw you dingbat
%\usepackage{dingbat} 
%\let\checkmark\origcheckmark %screw you dingbat






%\usepackage{fontawesome}

\usepackage{mathtools}
\usepackage{etoolbox, calculator}

\usepackage{xcolor}
\usepackage{tikz}
\usetikzlibrary{arrows.meta}
\usetikzlibrary{calc}
\usepackage[nomessages]{fp}
\usepackage{transparent}
\usepackage{accsupp}
%\usepackage{color, xcolor}

%colorblind-friendly palette
%\definecolor{dblue}{RGB}{51,34,136}
\definecolor{lblue}{RGB}{136,204,238}
%\definecolor{green}{RGB}{17,119,51}
\definecolor{tan}{RGB}{221,204,119}
%\definecolor{mauve}{RGB}{204,102,119}

\usepackage{tcolorbox}



\usepackage{xifthen}
\usepackage{nicefrac}
\usepackage{amsmath}
\usepackage{amsthm}
\usepackage{amssymb}
\theoremstyle{definition}
\newtheorem*{define}{Definition}
\newtheorem*{recall}{Recall}


\DeclareMathOperator{\tr}{tr}

\usepackage{multicol}
%\setlength{\columnsep}{1cm}

\usepackage{tablists, amsmath,vwcol, cancel, polynom}
\usetikzlibrary{shapes, patterns, decorations.shapes}
%\usepackage{tikzpeople}
\tikzstyle{vertex}=[shape=circle, minimum size=2mm, inner sep=0, fill]
\tikzstyle{opendot}=[shape=circle, minimum size=2mm, inner sep=0, fill=white, draw]

% common math quick commands
\newcommand{\nicedd}[2]{\nicefrac{\text{d}#1}{\text{d}#2}}
\newcommand{\dd}[2]{\dfrac{\text{d}#1}{\text{d}#2}}
\newcommand{\pd}[2]{\dfrac{\partial #1}{\partial#2}}
\renewcommand{\d}[1]{\text{d}#1}
\newcommand{\ddn}[3]{\dfrac{\text{d}^{#3}#1}{\text{d}#2^{#3}}}
\newcommand{\pdn}[3]{\dfrac{\partial^{#3}#1}{\partial#2^{#3}}}
\newcommand{\p}[0]{^{\prime}}
\newcommand{\pp}[0]{^{\prime\prime}}
\newcommand{\op}[2][\text{L}]{#1 \left[ #2 \right]}

\newcommand{\lap}[1]{\mathcal{L}\left\{#1\right\}}
\newcommand{\lapinv}[1]{\mathcal{L}^{-1}\left\{#1\right\}}
\newcommand{\lapint}[1]{\int_0^\infty e^{-st}#1dt}
\newcommand{\evalat}[2]{\Big|_{#1}^{#2}}

\newcommand{\paren}[1]{ \left( #1 \right)}

\newcommand{\haxis}[4][\normcolor]{\draw[#1, <->] (-#2,0)--(#3,0) node[right]{$#4$}; }

\newcommand{\circled}[1]{\raisebox{.5pt}{\textcircled{\raisebox{-.9pt} {#1}}}}
\newcommand{\axis}[4]{\draw[\normcolor, <->] (-#1,0)--(#2,0) 
node[right]{$x$};
\draw[help lines, <->] (0,-#3)--(0,#4) node[above]{$y$};}

\newcommand{\laxis}[6]{\draw[<->] (-#1,0)--(#2,0) 
node[right]{$#5$};
\draw[ <->] (0,-#3)--(0,#4) node[above]{$#6$};}
\newcommand{\xcoord}[2]{
	\draw (#1,.2)--(#1,-.2) node[below]{$#2$};}
\newcommand{\textnode}[3]{
	\draw (#1,#2) node[below]{$#3$};}
	
\newcommand{\nxcoord}[2]{
	\draw (#1,-.2)--(#1,.2) node[above]{$#2$};}
\newcommand{\ycoord}[2]{
	\draw (.2,#1)--(-.2,#1) node[left]{$#2$};}
\newcommand{\nycoord}[2]{
	\draw (-.2,#1)--(.2,#1) node[right]{$#2$};}
\newcommand{\dlim}{\displaystyle\lim}
\newcommand{\dlimx}[1]{\displaystyle\lim_{x \rightarrow #1}}
\newcommand{\stickfig}[2]{
	\draw (#1,#2) arc(-90:270:2mm);
	\draw (#1,#2)--(#1,#2-.5) (#1-.25,#2-.75)--(#1,#2-.5)--(#1+.25,#2-.75) (#1-.2,#2-.2)--(#1+.2,#2-.2);}	

%\newcounter{example}
%\setcounter{example}{1}
%\newcounter{preFrameExample}
%\AtBeginEnvironment{frame}{\setcounter{preFrameExample}{\value{example}}}
%\newcommand{\ex}[1]{
%	 \setcounter{example}{\value{preFrameExample}}
%	 \textcolor{green}{\small\fbox{Example \arabic{example}: #1}}\\[8pt]
%	\stepcounter{example}}
%\newcommand{\exans}[1]{
%	\SUBTRACT{\value{preFrameExample}}{1}{\n}
%	 \textcolor{green}{\small\fbox{Solution \n: #1}}\\[8pt]}
\mode<presentation> {

% The Beamer class comes with a number of default slide themes
% which change the colors and layouts of slides. Below this is a list
% of all the themes, uncomment each in turn to see what they look like.


\usetheme{CambridgeUS}
\usecolortheme[named=black]{structure}


\newcommand{\studentcolor}[0]{ForestGreen}
\newcommand{\normcolor}[0]{NavyBlue}
\newcommand{\alertcolor}{Red}

\setbeamercolor{normal text}{fg=\normcolor}
\setbeamercolor{frametitle}{fg=\normcolor}
\setbeamercolor{section in head/foot}{fg=Black, bg=Gray!20}
\setbeamercolor{subsection in head/foot}{fg=Green!70!Black, bg=Gray!10}
\setbeamercolor{alerted text}{fg=\alertcolor}
\setbeamerfont{alerted text}{series=\bf}
\setbeamertemplate{enumerate items}[default]
\setbeamercolor{enumerate item}{fg=\normcolor}

\setbeamertemplate{footline} % To remove the footer line in all slides uncomment this line
%\setbeamertemplate{footline}[page number] % To replace the footer line in all slides with a simple slide count uncomment this line

\setbeamertemplate{navigation symbols}{} % To remove the navigation symbols from the bottom of all slides uncomment this line
}

\newcommand{\alertbox}[1]{\tcbox[on line, colframe=\alertcolor, colback=White, left=2pt,right=2pt,top=2pt,bottom=2pt]{\usebeamercolor*{normal text}#1}}


\newcommand{\startstu}{\setbeamercolor{normal text}{fg=\studentcolor}\usebeamercolor*{normal text}\setbeamercolor{enumerate item}{fg=\studentcolor}\usebeamercolor*{enumerate item}}
\newcommand{\stopstu}{\setbeamercolor{normal text}{fg=\normcolor}\usebeamercolor*{normal text}\setbeamercolor{enumerate item}{fg=\normcolor}\usebeamercolor*{enumerate item}}

\newcommand{\takenote}[1]{ \begin{collect}{notes}{}{}{}{}  #1  \end{collect}  \addtocounter{notestaken}{1}} %\ifthenelse{\value{notestaken}>0}{\hrulefill\\}{}

\makeatletter
\newcommand{\cover}{\alt{\beamer@makecovered}{\beamer@fakeinvisible}}
\newcommand{\ucover}[1]{\iftoggle{full}{}{\beamer@endcovered} \stopstu #1\startstu \iftoggle{full}{}{\beamer@startcovered} }
%\newcommand{\ucover}[1]{\beamer@endcovered \stopstu #1\startstu \beamer@startcovered }
\makeatother

\newcommand{\skippause}{ \addtocounter{beamerpauses}{-1}}
\newcommand{\blockpres}{ \skippause \pause }

\newcommand{\studentify}[1]{\startstu #1  \stopstu }
\newcommand{\student}[1]{\iftoggle{full}{ \pause  \studentify{#1} }{\iftoggle{covered}{\studentify{#1}}{\cover{  #1 }}}}
\newcommand{\cstudent}[1]{\student{\begin{center} #1 \end{center}}}
\newcommand{\fullonly}[1]{\iftoggle{full}{ #1}{}}
\newcommand{\presentonly}[1]{\iftoggle{presentable}{ #1}{}}

\usepackage{xparse}
\usepackage{xifthen}

% shortcuts for commonly-used presentation elements
%\NewDocumentCommand{\slide}{o m}
% {\IfValueTF{#1}{\begin{frame}[t]{#1}}{\begin{frame}[t]} #2 \end{frame}}

\newtoggle{iscovered}

\newcommand{\slide}[2][]{%
%\setcounter{notestaken}{0}
\takenote{#2} 
%\ifthenelse{\equal{#1}{}}{\begin{frame}[t]}{\begin{frame}[t]{#1}} #2 \ifthenelse{\value{notestaken}>0}{ \note{\includecollection{notes}}}{} \end{frame}%
\ifthenelse{\equal{#1}{}}{\begin{frame}[t]}{\begin{frame}[t]{#1}} #2 \iftoggle{covered}{\settoggle{iscovered}{true}}{\settoggle{iscovered}{false}}  \note{ \iftoggle{iscovered}{}{\settoggle{covered}{true}} #2 \iftoggle{iscovered}{}{\settoggle{covered}{false}} } \end{frame}%
%\setcounter{notestaken}{0}
}
\newcommand{\defn}[2][]{%
 \setcounter{listcounter}{0}%
\ifthenelse{\equal{#1}{}}{\begin{block}{Definition}}{\begin{block}{#1 :}}%
 #2 \vspace{0.25em} \ifthenelse{\value{listcounter}>0}{\skippause}{} \pause \end{block}%
}



\newcommand{\arr}[2]{\begin{array}{#1}#2\end{array}}
\newcommand{\mat}[2]{\left[\arr{#1}{#2}\right]}
\newcommand{\carray}[1]{\arr{c}{#1}}
\newcommand{\larray}[1]{\arr{l}{#1}}
\newcommand{\rarray}[1]{\arr{r}{#1}}
\newcommand{\colvec}[1]{\mat{c}{#1}}

\newcommand{\itmz}[1]{\addtocounter{listcounter}{1} \begin{itemize}#1 \end{itemize} }
\newcommand{\subitem}[1]{\addtocounter{listcounter}{1} \begin{itemize} \item #1 \end{itemize}}
%
\newcommand{\enum}[1]{\addtocounter{listcounter}{1} \begin{enumerate} #1  \end{enumerate}  }


\newcommand{\algnlbl}[1]{\begin{align}#1  \end{align}} 
\newcommand{\algn}[1]{\begin{align*}#1  \end{align*}} 
\newcommand{\lgn}[1]{ \action<+->{#1} }
\newcommand{\slgn}[1]{\iftoggle{full}{\action<+->{ \startstu #1 \stopstu}}{ \cover{ #1 } } \takenote{$#1$}}

\newcommand{\chckmrk}{\alert{\checkmark}}

\usepackage{pifont}
\newcommand{\xmark}{\alert{\text{\large \ding{55}}}}

\newcommand{\return}[0]{\raisebox{.5ex}{\rotatebox[origin=c]{180}{$\Lsh$}}}
\usepackage{pbox}
%\newcommand{\ex}[1]{\rotatebox[origin=c]{10}{\uline{ex}}:$\;$\pbox[t][][b]{0.9\linewidth}{#1}}
\newcommand{\ex}[1]{\uline{ex}:$\;$\pbox[t][][t]{0.9\linewidth}{#1}}
\newcommand{\eg}[1]{e.g.,$\;$\pbox[t][][t]{0.9\linewidth}{#1}}
\newcommand{\tikzplot}[8][]{%
\begin{tikzpicture}

\begin{scope}[]%
\clip(-#2,-#4) rectangle (#3,#5);%
#8%
\end{scope}%
\laxis{#2}{#3}{#4}{#5}{#6}{#7}%
#1
\end{tikzpicture}%
}


\newcommand{\cancelslide}[1]{%
\begingroup%
\setbeamertemplate{background canvas}{%
\begin{tikzpicture}[remember picture,overlay]%
\draw[line width=2pt,red!60!black] %
  (current page.north west) -- (current page.south east);%
\draw[line width=2pt,red!60!black] %
  (current page.south west) -- (current page.north east);%
\end{tikzpicture}}%
#1%
\endgroup%
}
\renewcommand{\CancelColor}{\color{red}}
\newcommand{\twocols}[3][0.5]{\begin{columns}\begin{column}{#1\textwidth}#2\end{column}\hspace{1em}\vrule{}\hspace{1em}\begin{column}{#1\textwidth}#3\end{column}\end{columns}}

\newcommand{\twomini}[5][1]{\calculatespace \begin{minipage}[t]{\columnwidth}\begin{minipage}[][#1\contentheight][t]{#2\columnwidth}#4\end{minipage}\hfill\begin{minipage}[][#1\contentheight][t]{#3\columnwidth}#5\end{minipage}\end{minipage}}

\newcommand{\threemini}[7][1]{\calculatespace \begin{minipage}[t]{\columnwidth}\begin{minipage}[][#1\contentheight][t]{#2\columnwidth}#5\end{minipage}\hfill\begin{minipage}[][#1\contentheight][t]{#4\columnwidth}#6\end{minipage}\hfill\begin{minipage}[][#1\contentheight][t]{#3\columnwidth}#7\end{minipage}\end{minipage}}


\newcounter{listcounter}
\setcounter{listcounter}{0}



\newif\ifsidebartheme
\sidebarthemetrue

\newdimen\contentheight
\newdimen\contentwidth
\newdimen\contentleft
\newdimen\contentbottom
\makeatletter
\newcommand*{\calculatespace}{%
\contentheight=\paperheight%
\ifx\beamer@frametitle\@empty%
    \setbox\@tempboxa=\box\voidb@x%
  \else%
    \setbox\@tempboxa=\vbox{%
      \vbox{}%
      {\parskip0pt\usebeamertemplate***{frametitle}}%
    }%
    \ifsidebartheme%
      \advance\contentheight by-1em%
    \fi%
  \fi%
\advance\contentheight by-\ht\@tempboxa%
\advance\contentheight by-\dp\@tempboxa%
\advance\contentheight by-\beamer@frametopskip%
\ifbeamer@plainframe%
\contentbottom=0pt%
\else%
\advance\contentheight by-\headheight%
\advance\contentheight by\headdp%
\advance\contentheight by-\footheight%
\advance\contentheight by4pt%
\contentbottom=\footheight%
\advance\contentbottom by-4pt%
\fi%
\contentwidth=\paperwidth%
\ifbeamer@plainframe%
\contentleft=0pt%
\else%
\advance\contentwidth by-\beamer@rightsidebar%
\advance\contentwidth by-\beamer@leftsidebar\relax%
\contentleft=\beamer@leftsidebar%
\fi%
}
\makeatother


\iftoggle{dualscreen}{\setbeameroption{show notes on second screen=right}}{}

\usepackage{circuitikz}
\newcommand{\circled}[1]{ \raisebox{.5pt}{\textcircled{\raisebox{-.9pt}{#1}}}}

\begin{document}
\section{Lecture 5}
\subsection{Preamble}

\slide[Recall:  Linear 1$^{st}$ Order ODEs \hfill $y\p +p(t) y = g(t)$]{
Operator form: $\op{y}=g(t)$
\vfill
\itmz{
\item Linear 1$^{st}$ order operator ${\text L}$
\item $g(t)$ is called the \alert{inhomogeneity}\vfill
\item Linear + $g(t)=0\quad \Rightarrow$ \alert{ Homogeneous ODE}\vfill
\item Linear + $g(t)\neq0\quad \Rightarrow$ \alert{ Inhomogeneous ODE}\vfill
}\vfill

Solved by method of integrating factors:
\algn{\mu(t) y(t) &= \intop \mu(t) g(t) dt + C\\
y(t)& = \frac{ \intop \mu(t) g(t) dt}{ \mu(t)} + \underbrace{\frac{C}{ \mu(t)}}_{\text{indep. of $g(t)$}}}



}

\slide[General Solution Structure of Linear ODEs]{
\ex{1$^{st}$ Order Initial Value Problems}
 \[y\p +p(t) y = g(t); \qquad y(0)=y_0 \]\vspace{.5em}
\vfill
\twomini[.5]{.5}{.5}{
\uline{ General Solution:}\algn{ y(t) &= \frac{ \intop \mu(t) g(t) dt}{ \mu(t)} + \frac{C}{ \mu(t)} \\\\ \student{y(t)} &\student{= \underbrace{y_p(t)}_{\text{particular part}} + \underbrace{y_h(t)}_{\text{homogeneous part}}}}\
}{\uline{Associated Homogeneous Problem:}
\algn{y_h'+p(t)y_h&=0 \quad\text{(i.e., $g(t)=0$)}\\\ \Rightarrow \quad y_h = \frac{C}{ \mu(t)} }
}\vfill
\student{
\centering
All linear DEs have this type of solution structure.
\[ y(t) = \text{particular part + homogeneous part}\]
}
}

\slide[ Linear 2$^{nd}$ order ODEs]{
General DE: \student{\[y\pp +p(t)y\p +q(t)y=g(t)\]}Initial Conditions: \student{\[y(t_0)=y_0,\qquad y\p(t_0)=v_0\]}

Focus on the  homogeneous case first\vfill
\student{
\uline{simplest case}: constant coefficients \[ ay\pp +by\p+cy=0\]
\vfill
\centering
We want intuition for how to obtain and work with homogeneous solutions.

}

}%end slide
\subsection{The Ansatz Methods}

\slide[Find two solutions to $y''-2y'-3y=0$]{
Based on the constant coefficient first-order DEs, guess $y=e^{rt}$
\vfill
\student{
We call $e^{rt}$ an ansatz or ``trial solution'' \hfill (ansatz method)
\vfill
\algn{y'&=re^{rt} &y''&=r^2e^{rt}\\
r^2e^{rt} & -2re^{rt}-3e^{rt}=0& \paren{r^2  -2r-3}&\cancel{e^{rt}}=0\\
r^2 & -2r-3=0 &\text{(characteristic }&\text{polynomial)}}\vfill
\[r=\frac{2\pm\sqrt{4+12}}{2} =\frac{2\pm\sqrt{16}}{2} =\frac{2\pm 4}{2} = 3,-1  \]
\vfill
two solutions:
\algn{y_1&=e^{3t} &\text{\uline{and}}&&y_2&=e^{-t}}
}
}

\slide[Find two solutions to $t^2y''-ty'-3y=0$]{
Use an ansatz of $y=t^k$. \hfill (Check the logic by solving $ty'-ay=0$)
\student{
\algn{y'&=kt^{k-1} &y''&=k(k-1)t^{k-2}\\
t^2k(k-1)t^{k-2}-&tkt^{k-1}-3t^k=0 &(k^2-&2k-3)\cancel{t^{k}}=0\\
k^2 & -2k-3=0&\text{(characteristic }&\text{polynomial)}}\vfill
\[k = \frac{2\pm\sqrt{4+12}}{2} = 3, -1 \]
\vfill
two solutions:
\algn{y_1&=t^3 &\text{\uline{and}}&&y_2&=t^{-1}}
}
}
\subsection{Theory of Linear ODEs}
\slide[Superposition Principle for Linear Homogeneous ODEs]{
Suppose the functions $y_1(t)$ and $y_2(t)$ both independently solve a \uline{linear} homogeneous ODE\[ \op{y}=0\] then \[y =c_1y_1(t) + c_2y_2(t)\] is also a solution to the same ODE.\vfill
Proof:
\student{
\algn{\op{c_1y_1+c_2y_2} &\overset{\text{Linearity 1}}{=}  \op{c_1y_1}+\op{c_2y_2}\\
&\overset{\text{Linearity 2}}{=} c_1\op{y_1} + c_2 \op{y_2}\\
&\quad\;\; =c_1\cdot 0 + c_2 \cdot 0  \\&\quad\;\; = 0
}}
}

\slide[Solve $y''-2y'-3y=0$ with $y(0)=1,\; y'(0)=0$.]{
\student{
General Solution
\algn{y(t)&=c_1e^{3t}+c_2e^{-t}\intertext{initial conditions}
y(0)&=c_1+c_2=1\quad \circled{1}\\
y'(0)&=3c_1-c_2=0\quad \circled{2}&\Rightarrow c_2&=3c_1\quad \circled{3}\\
\uline{\circled{3} \rightarrow \circled{1}\hspace{-.25em}:}\;\;&4c_1=1 &\Aboxed{c_1&=\frac14}\\
&&\Aboxed{c_2&=\frac34}\\
y(t)&=\frac14e^{3t}+\frac34e^{-t}}
}

}

\slide[Solve $t^2y''-ty'-3y=0$ with $y(1)=2,\; y'(1)=1$.]{
\student{
General Solution
\algn{y(t)&=c_1t^3+c_2t^{-1}\intertext{initial conditions}
y(1)&=c_1+c_2=2 \quad \circled{1}\\
y'(1)&=3c_1-c_2=1\quad \circled{2}\\
\uline{\circled{1}+\circled{2}\hspace{-.25em}:}\;\; & 4c_1=3 & \Rightarrow \Aboxed{c_1&=\frac34}\quad \circled{3}\\
\uline{\circled{3} \rightarrow \circled{1}\hspace{-.25em}:}\;\;& \frac34 +c_2 =2 & \Rightarrow \Aboxed{c_2&=\frac54}\\
y(t)&=\frac34 t^3 + \frac54 t^{-1}}
}
}

\slide[Satisfying Initial Conditions: The General Case]{\vspace{-1em}
Given two functions $y_1$ and $y_2$ that both solve \vspace{-.25em}
\[y''+p(t)y'+q(t)y=0 \qquad \text{with } y(t_0) = y_0, \quad y\p(t_0)=v_0, \]

\vspace{-.25em}
 then the general solution is 
 \vspace{-.75em} 
 \[y=c_1y_1+c_2y_2.\]
 
 \vspace{-.25em}
 \uline{Try to satisfy the initial condition}\vfill
 \student{
 @$t=t_0$
 \twomini[0.25]{.45}{.55}{
\algn{c_1 y_1 +c_2 y_2 &= y_0\\c_1y_1\p+c_2y_2\p&=v_0}
 }{
 \algn{\longrightarrow \mat{cc}{y_1&y_2\\y_1\p&y_2\p} \mat{c}{c_1\\c_2} &= \mat{c}{y_0\\v_0}\\   A \quad\qquad \vec{c} \quad &=  \quad \vec{b}}
 }\vfill
Solution: $ \qquad \vec{c}=A^{-1}\vec{b} \qquad \text{if $A^{-1}$ exists} \quad $ (i.e., if $\det A \neq 0$)
\vfill
\uline{Note:} $\det A$ is called the Wronskian determinant.
 }
}


\subsection{The Wronskian}
\slide[Wronskian Determinant]{
For two differentiable functions, $y_1(t)$ and $y_2(t)$, the Wronskian determinant is denoted $W\left(y_1, y_2\right)(t)$, and is given by 
\[ W\left(y_1, y_2\right)(t) = \left| \begin{array}{cc} y_1(t)&y_2(t)\\y_1\p(t)&y_2\p(t) \end{array} \right| = y_1(t)y_2\p(t)-y_2(t)y_1\p(t)\]
\vfill
\student{\subitem{$W\left(y_1, y_2\right)(t_0)\neq0 \implies y=c_1y_1+c_2y_2$ can satisfy ANY initial conditions at $t_0$.\vfill
\item $W\left(y_1, y_2\right)(t)\neq0 $ for some $t$ is equivalent to stating that $y_1$ and $y_2$ are \uline{linearly independent} functions.  }}
\vfill

}


\subsection{Linear Independece and Fundamental sets of solutions}
\slide[Linear dependence of functions]{
\itmz{
\item Two functions $f(t)$ and $g(t)$ are \textbf{linearly dependent}  (or L.D.) if there exist a non-zero constant $k$
such that \[f(t)= kg(t)  \quad \forall t\] 
\vfill
\twomini[.4]{.5}{.5}{
Linearly Dependent\\
\tikzplot{.1}{4.3}{1}{1.25}{t}{}{
\draw[red ,domain=0:2.5, samples=120,  thick] plot ({\x}, {0.16*\x*\x}) node[right]{$f(t)=4t^2$};
\draw[blue ,domain=0:2.5, samples=120,  thick] plot ({\x}, {0.04*\x*\x}) node[right, ]{$g(t)=t^2$};
\draw[black ,domain=0:2.1, samples=120,  thick] plot ({\x}, {-0.08*\x*\x}) node[right, ]{$h(t)=-2t^2$};
}
}{
Not Linearly Dependent\\
\tikzplot{.1}{4}{.65}{1.65}{t}{}{
\draw[black ,domain=0:2, samples=120,  thick] plot ({\x}, {0.06*exp(1.5*\x)}) node[right, xshift=-.2em]{$h(t)=e^{\frac{3}{2}t}$};
\draw[red ,domain=0:2.5, samples=120,  thick] plot ({\x}, {0.06*exp(\x)}) node[right, xshift=-.6em, yshift=-.7em]{$f(t)=e^t$};
\draw[blue ,domain=0:2.5, samples=120,  thick] plot ({\x}, {0.4*sin(4*deg(\x))}) node[right, yshift=-.6em, xshift=-2em]{$g(t)=\sin(t)$};
}

}
\vfill
\item If functions are not linearly dependent, then we say they are \textbf{linearly independent}  (or L.I.).
\vfill

}
}

\slide[Fundamental sets of solutions]{
Suppose that $y_1(t)$ and $y_2(t)$ both solve a 2$^{nd}$ order linear homogeneous ODE\[\op{y}=0\]
If $y_1$ and $y_2$ are \textbf{linearly independent} functions, then the we call the set \[\{y_1,y_2\}\] a \textbf{fundamental set of solutions}, and the general solution for all homogeneous problem IVPs is \[y_h(t)=c_1y_1(t)+c_2y_2(t)\]\vfill
\alert{Take home message:} 
\student{If you can find two \uline{linearly independent} solutions to a homogeneous 2$^{nd}$ order linear DE, then you have found all of them}\vfill
Proof: DiffyQs \S 2.1 (Theorem 2.1.3)
}

\slide[Constant Coefficient IVP: The General Case]{\vspace{-1.5em}
\[ay''+by'+cy=0 \qquad \text{with } y(t_0) = y_0, \quad y\p(t_0)=v_0.\]
Guess $y=e^{rt}$.
\student{
\algn{ar^2e^{rt}+bre^{rt}+ce^{rt}=&0&
ar^2+br+c&=0 \quad\text{(char. poly.)}\\
r_{1,2} = \frac{-b\pm\sqrt{b^2-4ac}}{2a}&&\Rightarrow&\arr{l}{y_1=e^{r_1t}\\y_2=e^{r_2t}}\intertext{check for linear independence}
W=e^{r_1t}r_2e^{r_2t}&-r_1e^{r_1t}e^{r_2t}\\
=r_2e^{(r_1+r_2)t}&-r_1e^{(r_1+r_2)t}\\
=(r_2-r_1)&e^{(r_1+r_2)t}\\ \neq 0\qquad\quad &\quad \quad  \text{if $r_1\neq r_2$} &\Rightarrow y_1 & \text{ and } y_2\text{ are L.I.}}
Fundamental solution: $\quad y(t)=c_1e^{r_1t}+c_2e^{r_2t}$
}
}
\subsection{Summary}
\slide[Summary: 2$^{nd}$ Order Homogeneous ODEs]{
\[y''+p(t)y'+q(t)y=0\]
\itmz{
\item Always has two linearly independent solutions $y_1$ and $y_2$.
\subitem{Ansatz method allows us to guess them. \item Constant coefficients: $ay''+by'+cy=0 \;\; \Rightarrow y=e^{rt}$}\vfill
\item Superposition Principle: \subitem{The linear combination $y=c_1y_1+c_2y_2$ also solves the homogeneous ODE.}\vfill
\item Fundamental Solutions: \subitem{Linear combinations of $y_1$ and $y_2$ can solve all IVPs.\item Use the Wronskian to determine linear independence and establish a fundamental set of solutions.
}
}
}

\end{document}