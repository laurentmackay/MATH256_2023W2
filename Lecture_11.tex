\documentclass[11pt, dvipsnames, handout]{beamer}
\newtoggle{full}
\settoggle{full}{true}

\newtoggle{covered}
\settoggle{covered}{false}

\newtoggle{presentable}
\settoggle{presentable}{false}

\newtoggle{dualscreen}
\settoggle{dualscreen}{false}

\usepackage{pgfplots}
%\pgfplotsset{compat = newest}

\usepackage{pgfpages}

\setbeamertemplate{note page}{\pagecolor{yellow!5}\vfill \insertnote \vfill}
\usepackage{collect}
\definecollection{notes}
\newcounter{notestaken}

\usepackage{xpatch}

\usepackage{ulem}

\usepackage[framemethod=tikz]{mdframed}

\usepackage{scalerel}
\usepackage{calc}

%\usepackage{enumitem}
\setlength\fboxsep{.2em}

\usepackage{graphicx} % Allows including images
\usepackage{booktabs} % Allows the use of \toprule, \midrule and \bottomrule in tables

\xpatchcmd{\itemize}
  {\def\makelabel}
  {\setlength{\itemsep}{0.65 em}\def\makelabel}
  {}
  {}


\xpatchcmd{\beamer@enum@}
  {\def\makelabel}
  {\setlength{\itemsep}{0.65 em}\def\makelabel}
  {}
  {}


%\makeatletter
%\renewcommand{\itemize}[1][]{%
%  \beamer@ifempty{#1}{}{\def\beamer@defaultospec{#1}}%
%  \ifnum \@itemdepth >2\relax\@toodeep\else
%    \advance\@itemdepth\@ne
%    \beamer@computepref\@itemdepth% sets \beameritemnestingprefix
%    \usebeamerfont{itemize/enumerate \beameritemnestingprefix body}%
%    \usebeamercolor[fg]{itemize/enumerate \beameritemnestingprefix body}%
%    \usebeamertemplate{itemize/enumerate \beameritemnestingprefix body begin}%
%    \list
%      {\usebeamertemplate{itemize \beameritemnestingprefix item}}
%      {%
%        \setlength\topsep{1em}%NEW
%        \setlength\partopsep{1em}%NEW
%        \setlength\itemsep{1em}%NEW
%        \def\makelabel##1{%
%          {%
%            \hss\llap{{%
%                \usebeamerfont*{itemize \beameritemnestingprefix item}%
%                \usebeamercolor[fg]{itemize \beameritemnestingprefix item}##1}}%
%          }%
%        }%
%      }
%  \fi%
%  \beamer@cramped%
%  \raggedright%
%  \beamer@firstlineitemizeunskip%
%}
%
%
%
%
%
%\makeatother

%\setlist[beamer@enum@]{topsep=1 em}
%\let\origcheckmark\checkmark %screw you dingbat
%\let\checkmark\undefined %screw you dingbat
%\usepackage{dingbat} 
%\let\checkmark\origcheckmark %screw you dingbat






%\usepackage{fontawesome}

\usepackage{mathtools}
\usepackage{etoolbox, calculator}

\usepackage{xcolor}
\usepackage{tikz}
\usetikzlibrary{arrows.meta}
\usetikzlibrary{calc}
\usepackage[nomessages]{fp}
\usepackage{transparent}
\usepackage{accsupp}
%\usepackage{color, xcolor}

%colorblind-friendly palette
%\definecolor{dblue}{RGB}{51,34,136}
\definecolor{lblue}{RGB}{136,204,238}
%\definecolor{green}{RGB}{17,119,51}
\definecolor{tan}{RGB}{221,204,119}
%\definecolor{mauve}{RGB}{204,102,119}

\usepackage{tcolorbox}



\usepackage{xifthen}
\usepackage{nicefrac}
\usepackage{amsmath}
\usepackage{amsthm}
\usepackage{amssymb}
\theoremstyle{definition}
\newtheorem*{define}{Definition}
\newtheorem*{recall}{Recall}


\DeclareMathOperator{\tr}{tr}

\usepackage{multicol}
%\setlength{\columnsep}{1cm}

\usepackage{tablists, amsmath,vwcol, cancel, polynom}
\usetikzlibrary{shapes, patterns, decorations.shapes}
%\usepackage{tikzpeople}
\tikzstyle{vertex}=[shape=circle, minimum size=2mm, inner sep=0, fill]
\tikzstyle{opendot}=[shape=circle, minimum size=2mm, inner sep=0, fill=white, draw]

% common math quick commands
\newcommand{\nicedd}[2]{\nicefrac{\text{d}#1}{\text{d}#2}}
\newcommand{\dd}[2]{\dfrac{\text{d}#1}{\text{d}#2}}
\newcommand{\pd}[2]{\dfrac{\partial #1}{\partial#2}}
\renewcommand{\d}[1]{\text{d}#1}
\newcommand{\ddn}[3]{\dfrac{\text{d}^{#3}#1}{\text{d}#2^{#3}}}
\newcommand{\pdn}[3]{\dfrac{\partial^{#3}#1}{\partial#2^{#3}}}
\newcommand{\p}[0]{^{\prime}}
\newcommand{\pp}[0]{^{\prime\prime}}
\newcommand{\op}[2][\text{L}]{#1 \left[ #2 \right]}

\newcommand{\lap}[1]{\mathcal{L}\left\{#1\right\}}
\newcommand{\lapinv}[1]{\mathcal{L}^{-1}\left\{#1\right\}}
\newcommand{\lapint}[1]{\int_0^\infty e^{-st}#1dt}
\newcommand{\evalat}[2]{\Big|_{#1}^{#2}}

\newcommand{\paren}[1]{ \left( #1 \right)}

\newcommand{\haxis}[4][\normcolor]{\draw[#1, <->] (-#2,0)--(#3,0) node[right]{$#4$}; }

\newcommand{\circled}[1]{\raisebox{.5pt}{\textcircled{\raisebox{-.9pt} {#1}}}}
\newcommand{\axis}[4]{\draw[\normcolor, <->] (-#1,0)--(#2,0) 
node[right]{$x$};
\draw[help lines, <->] (0,-#3)--(0,#4) node[above]{$y$};}

\newcommand{\laxis}[6]{\draw[<->] (-#1,0)--(#2,0) 
node[right]{$#5$};
\draw[ <->] (0,-#3)--(0,#4) node[above]{$#6$};}
\newcommand{\xcoord}[2]{
	\draw (#1,.2)--(#1,-.2) node[below]{$#2$};}
\newcommand{\textnode}[3]{
	\draw (#1,#2) node[below]{$#3$};}
	
\newcommand{\nxcoord}[2]{
	\draw (#1,-.2)--(#1,.2) node[above]{$#2$};}
\newcommand{\ycoord}[2]{
	\draw (.2,#1)--(-.2,#1) node[left]{$#2$};}
\newcommand{\nycoord}[2]{
	\draw (-.2,#1)--(.2,#1) node[right]{$#2$};}
\newcommand{\dlim}{\displaystyle\lim}
\newcommand{\dlimx}[1]{\displaystyle\lim_{x \rightarrow #1}}
\newcommand{\stickfig}[2]{
	\draw (#1,#2) arc(-90:270:2mm);
	\draw (#1,#2)--(#1,#2-.5) (#1-.25,#2-.75)--(#1,#2-.5)--(#1+.25,#2-.75) (#1-.2,#2-.2)--(#1+.2,#2-.2);}	

%\newcounter{example}
%\setcounter{example}{1}
%\newcounter{preFrameExample}
%\AtBeginEnvironment{frame}{\setcounter{preFrameExample}{\value{example}}}
%\newcommand{\ex}[1]{
%	 \setcounter{example}{\value{preFrameExample}}
%	 \textcolor{green}{\small\fbox{Example \arabic{example}: #1}}\\[8pt]
%	\stepcounter{example}}
%\newcommand{\exans}[1]{
%	\SUBTRACT{\value{preFrameExample}}{1}{\n}
%	 \textcolor{green}{\small\fbox{Solution \n: #1}}\\[8pt]}
\mode<presentation> {

% The Beamer class comes with a number of default slide themes
% which change the colors and layouts of slides. Below this is a list
% of all the themes, uncomment each in turn to see what they look like.


\usetheme{CambridgeUS}
\usecolortheme[named=black]{structure}


\newcommand{\studentcolor}[0]{ForestGreen}
\newcommand{\normcolor}[0]{NavyBlue}
\newcommand{\alertcolor}{Red}

\setbeamercolor{normal text}{fg=\normcolor}
\setbeamercolor{frametitle}{fg=\normcolor}
\setbeamercolor{section in head/foot}{fg=Black, bg=Gray!20}
\setbeamercolor{subsection in head/foot}{fg=Green!70!Black, bg=Gray!10}
\setbeamercolor{alerted text}{fg=\alertcolor}
\setbeamerfont{alerted text}{series=\bf}
\setbeamertemplate{enumerate items}[default]
\setbeamercolor{enumerate item}{fg=\normcolor}

\setbeamertemplate{footline} % To remove the footer line in all slides uncomment this line
%\setbeamertemplate{footline}[page number] % To replace the footer line in all slides with a simple slide count uncomment this line

\setbeamertemplate{navigation symbols}{} % To remove the navigation symbols from the bottom of all slides uncomment this line
}

\newcommand{\alertbox}[1]{\tcbox[on line, colframe=\alertcolor, colback=White, left=2pt,right=2pt,top=2pt,bottom=2pt]{\usebeamercolor*{normal text}#1}}


\newcommand{\startstu}{\setbeamercolor{normal text}{fg=\studentcolor}\usebeamercolor*{normal text}\setbeamercolor{enumerate item}{fg=\studentcolor}\usebeamercolor*{enumerate item}}
\newcommand{\stopstu}{\setbeamercolor{normal text}{fg=\normcolor}\usebeamercolor*{normal text}\setbeamercolor{enumerate item}{fg=\normcolor}\usebeamercolor*{enumerate item}}

\newcommand{\takenote}[1]{ \begin{collect}{notes}{}{}{}{}  #1  \end{collect}  \addtocounter{notestaken}{1}} %\ifthenelse{\value{notestaken}>0}{\hrulefill\\}{}

\makeatletter
\newcommand{\cover}{\alt{\beamer@makecovered}{\beamer@fakeinvisible}}
\newcommand{\ucover}[1]{\iftoggle{full}{}{\beamer@endcovered} \stopstu #1\startstu \iftoggle{full}{}{\beamer@startcovered} }
%\newcommand{\ucover}[1]{\beamer@endcovered \stopstu #1\startstu \beamer@startcovered }
\makeatother

\newcommand{\skippause}{ \addtocounter{beamerpauses}{-1}}
\newcommand{\blockpres}{ \skippause \pause }

\newcommand{\studentify}[1]{\startstu #1  \stopstu }
\newcommand{\student}[1]{\iftoggle{full}{ \pause  \studentify{#1} }{\iftoggle{covered}{\studentify{#1}}{\cover{  #1 }}}}
\newcommand{\cstudent}[1]{\student{\begin{center} #1 \end{center}}}
\newcommand{\fullonly}[1]{\iftoggle{full}{ #1}{}}
\newcommand{\presentonly}[1]{\iftoggle{presentable}{ #1}{}}

\usepackage{xparse}
\usepackage{xifthen}

% shortcuts for commonly-used presentation elements
%\NewDocumentCommand{\slide}{o m}
% {\IfValueTF{#1}{\begin{frame}[t]{#1}}{\begin{frame}[t]} #2 \end{frame}}

\newtoggle{iscovered}

\newcommand{\slide}[2][]{%
%\setcounter{notestaken}{0}
\takenote{#2} 
%\ifthenelse{\equal{#1}{}}{\begin{frame}[t]}{\begin{frame}[t]{#1}} #2 \ifthenelse{\value{notestaken}>0}{ \note{\includecollection{notes}}}{} \end{frame}%
\ifthenelse{\equal{#1}{}}{\begin{frame}[t]}{\begin{frame}[t]{#1}} #2 \iftoggle{covered}{\settoggle{iscovered}{true}}{\settoggle{iscovered}{false}}  \note{ \iftoggle{iscovered}{}{\settoggle{covered}{true}} #2 \iftoggle{iscovered}{}{\settoggle{covered}{false}} } \end{frame}%
%\setcounter{notestaken}{0}
}
\newcommand{\defn}[2][]{%
 \setcounter{listcounter}{0}%
\ifthenelse{\equal{#1}{}}{\begin{block}{Definition}}{\begin{block}{#1 :}}%
 #2 \vspace{0.25em} \ifthenelse{\value{listcounter}>0}{\skippause}{} \pause \end{block}%
}



\newcommand{\arr}[2]{\begin{array}{#1}#2\end{array}}
\newcommand{\mat}[2]{\left[\arr{#1}{#2}\right]}
\newcommand{\carray}[1]{\arr{c}{#1}}
\newcommand{\larray}[1]{\arr{l}{#1}}
\newcommand{\rarray}[1]{\arr{r}{#1}}
\newcommand{\colvec}[1]{\mat{c}{#1}}

\newcommand{\itmz}[1]{\addtocounter{listcounter}{1} \begin{itemize}#1 \end{itemize} }
\newcommand{\subitem}[1]{\addtocounter{listcounter}{1} \begin{itemize} \item #1 \end{itemize}}
%
\newcommand{\enum}[1]{\addtocounter{listcounter}{1} \begin{enumerate} #1  \end{enumerate}  }


\newcommand{\algnlbl}[1]{\begin{align}#1  \end{align}} 
\newcommand{\algn}[1]{\begin{align*}#1  \end{align*}} 
\newcommand{\lgn}[1]{ \action<+->{#1} }
\newcommand{\slgn}[1]{\iftoggle{full}{\action<+->{ \startstu #1 \stopstu}}{ \cover{ #1 } } \takenote{$#1$}}

\newcommand{\chckmrk}{\alert{\checkmark}}

\usepackage{pifont}
\newcommand{\xmark}{\alert{\text{\large \ding{55}}}}

\newcommand{\return}[0]{\raisebox{.5ex}{\rotatebox[origin=c]{180}{$\Lsh$}}}
\usepackage{pbox}
%\newcommand{\ex}[1]{\rotatebox[origin=c]{10}{\uline{ex}}:$\;$\pbox[t][][b]{0.9\linewidth}{#1}}
\newcommand{\ex}[1]{\uline{ex}:$\;$\pbox[t][][t]{0.9\linewidth}{#1}}
\newcommand{\eg}[1]{e.g.,$\;$\pbox[t][][t]{0.9\linewidth}{#1}}
\newcommand{\tikzplot}[8][]{%
\begin{tikzpicture}

\begin{scope}[]%
\clip(-#2,-#4) rectangle (#3,#5);%
#8%
\end{scope}%
\laxis{#2}{#3}{#4}{#5}{#6}{#7}%
#1
\end{tikzpicture}%
}


\newcommand{\cancelslide}[1]{%
\begingroup%
\setbeamertemplate{background canvas}{%
\begin{tikzpicture}[remember picture,overlay]%
\draw[line width=2pt,red!60!black] %
  (current page.north west) -- (current page.south east);%
\draw[line width=2pt,red!60!black] %
  (current page.south west) -- (current page.north east);%
\end{tikzpicture}}%
#1%
\endgroup%
}
\renewcommand{\CancelColor}{\color{red}}
\newcommand{\twocols}[3][0.5]{\begin{columns}\begin{column}{#1\textwidth}#2\end{column}\hspace{1em}\vrule{}\hspace{1em}\begin{column}{#1\textwidth}#3\end{column}\end{columns}}

\newcommand{\twomini}[5][1]{\calculatespace \begin{minipage}[t]{\columnwidth}\begin{minipage}[][#1\contentheight][t]{#2\columnwidth}#4\end{minipage}\hfill\begin{minipage}[][#1\contentheight][t]{#3\columnwidth}#5\end{minipage}\end{minipage}}

\newcommand{\threemini}[7][1]{\calculatespace \begin{minipage}[t]{\columnwidth}\begin{minipage}[][#1\contentheight][t]{#2\columnwidth}#5\end{minipage}\hfill\begin{minipage}[][#1\contentheight][t]{#4\columnwidth}#6\end{minipage}\hfill\begin{minipage}[][#1\contentheight][t]{#3\columnwidth}#7\end{minipage}\end{minipage}}


\newcounter{listcounter}
\setcounter{listcounter}{0}



\newif\ifsidebartheme
\sidebarthemetrue

\newdimen\contentheight
\newdimen\contentwidth
\newdimen\contentleft
\newdimen\contentbottom
\makeatletter
\newcommand*{\calculatespace}{%
\contentheight=\paperheight%
\ifx\beamer@frametitle\@empty%
    \setbox\@tempboxa=\box\voidb@x%
  \else%
    \setbox\@tempboxa=\vbox{%
      \vbox{}%
      {\parskip0pt\usebeamertemplate***{frametitle}}%
    }%
    \ifsidebartheme%
      \advance\contentheight by-1em%
    \fi%
  \fi%
\advance\contentheight by-\ht\@tempboxa%
\advance\contentheight by-\dp\@tempboxa%
\advance\contentheight by-\beamer@frametopskip%
\ifbeamer@plainframe%
\contentbottom=0pt%
\else%
\advance\contentheight by-\headheight%
\advance\contentheight by\headdp%
\advance\contentheight by-\footheight%
\advance\contentheight by4pt%
\contentbottom=\footheight%
\advance\contentbottom by-4pt%
\fi%
\contentwidth=\paperwidth%
\ifbeamer@plainframe%
\contentleft=0pt%
\else%
\advance\contentwidth by-\beamer@rightsidebar%
\advance\contentwidth by-\beamer@leftsidebar\relax%
\contentleft=\beamer@leftsidebar%
\fi%
}
\makeatother



\iftoggle{dualscreen}{\setbeameroption{show notes on second screen=right}}{}


\begin{document}
\settoggle{covered}{true}
\section{Lecture 11}
\subsection{Preamble}
\slide[Recall: Inhomogeneous ODEs]{
\uline{1$^{st}$ Order ODEs:}
\[y'+p(t)y=g(t)\]
\student{
Solve by Method of Integrating Factors
\subitem{Requirement: $p(t)$ and $g(t)$ are continuous functions}
}
\vfill
\uline{2$^{nd}$ Order ODEs:}
\[ay''+by'+cy=g(t)\]
\student{
Solve by Method of Undetermined Coefficients
\subitem{Requirement:  $g(t)$ has a finite family of functional forms \subitem{Pre-requisite: $g(t)$ and all its derivatives are continuous.}}
}
\vfill
\student{\centering How do we handle cases where $g(t)$ is discontinuous? 
\vfill
Laplace Transforms!}
}

\slide[What is a Laplace Transform?]{
An operator:\student{ \subitem{ Takes as an input a function of one variable, e.g. $y(t)$ \item  Yields another function of a new variable $Y(s)$ \item Mapping between functions in the ``time-domain'' and the ``s domain''.}
\vfill
\[\lap{y(t)}= Y(s)=\intop_0^\infty e^{-st} y(t) dt\]
\vfill

Convention: Lowercase letter in ``time-domain'', uppercase in ''s-domain''
}


}
\settoggle{covered}{false}

\slide[Fun properties of Laplace Transforms]{
\[\lap{y(t)}= Y(s)=\intop_0^\infty e^{-st} y(t) dt\]

\vfill\student{\enum{\item 
Laplace transforms ``eat'' derivatives.
\subitem{Converts ODEs into algebraic expressions: \vfill \centerline{ Solve for $Y(s)$ in the ``s-domain''  $\Leftrightarrow$ Solve the ODE in the ``t-domain''}}
\vfill
\item Laplace transforms smooth out discontinuities \subitem{ODEs with discontinuities become continuous functions of $s$}
\vfill
Before solving ODEs, let practice taking LT of simple functions.
}
}
}



\slide{\ex{$y(t)=\frac12$}\student{\hspace{3em}$\lap{y(t)}=Y(s)=\lapint{\frac12}$
\algn{&=-\frac{1}{2s}e^{-st}\Big|_0^\infty
&=-\lim_{A\to\infty}\frac{1}{2s}e^{-st}\Big|_0^A\\
&=-\frac{1}{2s}\lim_{A\to\infty}\paren{e^{-sA}-1}\\
&=\begin{cases}
\frac{1}{2s}& \text{if }s>0\\
DNE & \text{if }s\leq0
\end{cases}}
}\vfill
\centerline{
\tikzplot[\textnode{4.75}{1.5}{\overset{\mathcal{L}}{\rightarrow}}]{1}{3}{.1}{2.1}{\color{RubineRed}t}{\color{RubineRed}y(t)}{
\draw[domain=-1:6, smooth, RubineRed, thick, samples=150] plot ({\x}, {1.5});}
\hfill 
\tikzplot{1}{3}{.1}{2.1}{\color{YellowOrange}s}{\color{YellowOrange}Y(s)}{
\draw[domain=0.01:6, YellowOrange, thick, samples=100] plot ({\x}, {.02/(\x/5)});
\draw[pattern=north west lines, pattern color=\normcolor] (-2,-2) rectangle (0,4);}
}
}


\slide{
\ex{$y(t)=e^{-6t}$}\student{
\algn{\lap{y(t)}&=Y(s)=\lapint{e^{-6t}}\\
&=\intop_0^\infty{e^{-(s+6)t}} = -\frac{1}{s+6} 
\lim_{A\to\infty}\left( e^{-(s+6)A} - 1\right) \\
&=\begin{cases}
\frac{1}{s+6} & \text{if }s>-6\\
DNE & \text{if }s\leq-6
\end{cases}
}}
\vfill
\centerline{
\tikzplot[\textnode{4.75}{1.5}{\overset{\mathcal{L}}{\rightarrow}}]{1}{3}{.1}{1.5}
{\color{RubineRed}t}
{\color{RubineRed}y(t)}{
\draw[domain=-1:6, smooth, RubineRed, thick, samples=150] plot ({\x}, {.4*exp(-\x)});}
\hfill 
\tikzplot[\xcoord{-.5}{-6}]{1}{3}{.1}{1.25}{\color{YellowOrange}s}{\color{YellowOrange}Y(s)}{
\draw[domain=-.499:6, YellowOrange, thick, samples=100] plot ({\x}, {.15/(\x+.5)});
\draw[pattern=north west lines, pattern color=\normcolor] (-2,-2) rectangle (-.5,4);}
}
}

\slide[General Results]{
For any constants $C$ and $a$ we have the Laplace Transforms of the following functions $y(t)$:

\vfill
\algn{y(t)&=C&\lap{C}&=\frac{C}{s}&\text{Constant}\\
y(t)&=e^{at}&\lap{e^{at}}&=\frac{1}{s-a}&\text{Exponential Function}\\
}

\vfill
\student{
From now on, we don't worry too much about the domain of definition.\vfill
In general, there are always some conditions on $s$ for the integrals to exist.}
}


\slide[General Result: Linearity of Laplace Transforms]{
Given any two function $f(t)$ and $g(t)$ as well as any constant $c$.
\vfill
\enum{\item $\lap{f(t)+g(t)}=\student{\lapint{\paren{f(t)+g(t)}}}$\\~\\
\student{$=\lap{f(t)}+\lap{g(t)}=F(s)+G(s)$}\vfill
\item $\lap{cf(t)}=\student{c\lap{f(t)}=cF(s)}$\vfill
}
\student{\centerline{The Laplace tranform is \uline{linear}.}}
}

\slide{
\ex{$y(t)=\cos(at)$ or $y(t)=\sin(at)$} \\~\\ \student{
Euler's Identity:\[e^{iat} = \cos(at) + i\sin(at) \] 
\algn{\lap{e^{iat}} &= \lap{\cos(at)} + i \lap{sin(at)}\\
&=\frac{1}{s-ia}  = \frac{1}{s-ia} \times  \frac{s+ia}{s+ia}\\
&= \frac{s+ia}{s^2-ias+ias-i^2a^2} =  \frac{s+ia}{s^2+a^2}\\\\
&=\underbrace{\frac{s}{s^2+a^2}}_{\lap{\cos(at)}} + i \underbrace{\frac{a}{s^2+a^2}}_{\lap{\sin(at)}}   }
\vfill
Same result can be found through integration by parts (twice).
}
}


\slide{
\ex{$y(t)=t$}\vspace{-1em}
\student{
\algn{\lap{t} =\lapint{t} &= \lapint{t} \\\\ \text{let}  \larray{u=t,\; du=dt\\dv =e^{-st}dt,\; v=-\frac{e^{-st}}{s}}&\\
\intop t e^{-st} dt &= uv - \intop vdu  = \frac{te^{-st}}{s} + \intop \frac{e^{-st}}{s}\\
&= \frac{te^{-st}}{s}- \frac{e^{-st}}{s^2} = -\frac{e^{-st}(st+1)}{s^2}\\
\lap{t} =   \lim_{A\to\infty}-\frac{e^{-st}(st+1)}{s^2} \Big|_0^A& =\lim_{A\to\infty} - \frac{e^{-sA}(sA+1)}{s^2}   + \frac{1}{s^2}\\
&=\frac{1}{s^2} \qquad \text{($s>0$)}
}
For $y(t)=t^k$,  integrate by parts $k$ times.
}
}

\slide[General Results]{
For any constants $C$, $a$, $\omega$, and $k$ we have the Laplace Transforms of the following functions $y(t)$:

\vfill
\algn{y(t)&=C&\lap{C}&=\frac{C}{s}&\text{Constant}\\
y(t)&=e^{at}&\lap{e^{at}}&=\frac{1}{s-a}&\text{Exponential Function}\\
y(t)&=\cos(\omega t)&\lap{\cos{\omega t}}&=\frac{s}{s^2+\omega^2}&\text{Cosine}\\
y(t)&=\sin(\omega t)&\lap{\sin{\omega  t}}&=\frac{\omega}{s^2+\omega^2}&\text{Sine}\\
y(t)&=t^k&\lap{t^k}&=\frac{k!}{s^{k+1}}&\text{Power Function}\\
}\vfill
}

\slide[Summary]{
\vfill
\itmz{

\item Laplace transform (LT) maps $f(t) \rightarrow F(s)$ \vfill
\subitem{From "t-space" to "s-space" . \item We will learn to invert the transform in the next lecture.}\vfill
\item LT: $\lap{f(t)}=F(s)=\lapint{f(t)}$ \vfill \subitem{Evaluation of the integrals is tedious. \item We use general results to quickly transform functions. \item Many tables exist online and in textbooks.} \vfill
\item  LT is linear because the integral is linear \vfill
\enum{\item $\lap{f+g} = F(s)+G(s)$\item $\lap{cf}=cF(s)$}\vfill
}
}



\begin{comment}
\slide[Recall: $ay''+by'+cy=f(t)$]{
We have studied this DE extensively, and know how to solve it for a large class of functions $f(t)$ that are \uline{continuous}.

\vfill

\textbf{Note: }Discontinuities in $f(t)$ create kinks or jumps in the solution.
\tikzplot[\xcoord{3}{\text{kink}} ]{.1}{5}{.1}{1.5}{t}{}{

\draw[domain=0:2, thick, samples=100] plot ({\x*1.5}, {.8*exp(-\x)});
\draw[domain=2:10, thick, samples=100] plot ({\x*1.5}, {.8*exp(-\x)+(1-exp(-(\x-2)))*1.25});

}\hfil \tikzplot[\xcoord{3}{\text{jump}} ]{.1}{5}{.1}{1.5}{t}{}{

\draw[domain=0:2, thick, samples=100] plot ({\x*1.5}, {.8*exp(-\x)});
\draw[domain=2:10, thick, samples=100] plot ({\x*1.5}, {.8*exp(-(\x-2))});
\draw[thick , dashed] (3,0.1)--(3,.8);

}\vfill 

How can we deal with a RHS with, a potentially infinite number of, discontinuities?
\vfill
\student{
Idea: Apply a transformation to the RHS to make it ``nice''.
\vfill
\centerline{The Laplace Transform.}
}


}


\subsection{Laplace Transforms}
\slide[Laplace Transform Workflow]{
\vspace{-1em}
Transforms the problem from the time domain to the ``solution''-domain.
\vspace{-0.15em}
\[
\begin{array}{cccccc}
\color{RubineRed}
\tiny{t-domain} &
\color{RubineRed}
\begin{array}{c}
\text{Unknown y(t)}\\
\text{that solves ODE}
\end{array} &   \color{RubineRed}\underrightarrow{\text{Solve ODE}} &\color{RubineRed} \text{Found y(t)}\\&
\student{
\begin{array}{ccc}
\qquad\qquad & \left\downarrow \vcenter{\hrule height6em}\right.\kern-\nulldelimiterspace & \begin{array}{c}
\text{Transform }\\
\text{y(t) and}\\
\text{the ODE}
\end{array}\end{array}} &    & \student{\begin{array}{ccc}
\begin{array}{c}
\text{Invert }\\
\text{the}\\
\text{transform}
\end{array} & \left\uparrow \vcenter{\hrule height6em}\right.\kern-\nulldelimiterspace & \qquad\qquad\end{array}}\\\\
\color{YellowOrange}s-domain&\color{YellowOrange}\begin{array}{c}
\text{Unknown Y(s)}\\
\text{that solves an }\\\text{algebraic equation}
\end{array} &  \color{YellowOrange} \underrightarrow{\text{Solve alg. eq.}}   &\color{YellowOrange}\text{Isolate Y(s)}\\
\\
\end{array}
\]
\vfill
\student{
Laplace Transform of $y(t)$:
\[\lap{y(t)}=Y(s)=\int_0^\infty e^{-st} y(t)dt\]

}
}




\slide{\ex{$y(t)=\frac12$}\student{\hspace{3em}$\lap{y(t)}=Y(s)=\lapint{\frac12}$
\algn{&=-\frac{1}{2s}e^{-st}\Big|_0^\infty
&=-\lim_{A\to\infty}\frac{1}{2s}e^{-st}\Big|_0^A\\
&=-\frac{1}{2s}\lim_{A\to\infty}\paren{e^{-sA}-1}\\
&=\begin{cases}
\frac{1}{2s}& \text{if }s>0\\
DNE & \text{if }s\leq0
\end{cases}}
}\vfill
\centerline{
\tikzplot[\textnode{4.75}{1.5}{\overset{\mathcal{L}}{\rightarrow}}]{1}{3}{.1}{2.1}{\color{RubineRed}t}{\color{RubineRed}y(t)}{
\draw[domain=-1:6, smooth, RubineRed, thick, samples=150] plot ({\x}, {1.5});}
\hfill 
\tikzplot{1}{3}{.1}{2.1}{\color{YellowOrange}s}{\color{YellowOrange}Y(s)}{
\draw[domain=0.01:6, YellowOrange, thick, samples=100] plot ({\x}, {.02/(\x/5)});
\draw[pattern=north west lines, pattern color=\normcolor] (-2,-2) rectangle (0,4);}
}
}



\end{comment}
\end{document}