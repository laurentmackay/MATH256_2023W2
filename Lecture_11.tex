\input{notes.tex}


\iftoggle{dualscreen}{\setbeameroption{show notes on second screen=right}}{}


\begin{document}
\settoggle{covered}{true}
\section{Lecture 11}
\subsection{Preamble}
\slide[Recall: Inhomogeneous ODEs]{
\uline{1$^{st}$ Order ODEs:}
\[y'+p(t)y=g(t)\]
\student{
Solve by Method of Integrating Factors
\subitem{Requirement: $p(t)$ and $g(t)$ are continuous functions}
}
\vfill
\uline{2$^{nd}$ Order ODEs:}
\[ay''+by'+cy=g(t)\]
\student{
Solve by Method of Undetermined Coefficients
\subitem{Requirement:  $g(t)$ has a finite family of functional forms \subitem{Pre-requisite: $g(t)$ and all its derivatives are continuous.}}
}
\vfill
\student{\centering How do we handle cases where $g(t)$ is discontinuous? 
\vfill
Laplace Transforms!}
}

\slide[What is a Laplace Transform?]{
An operator:\student{ \subitem{ Takes as an input a function of one variable, e.g. $y(t)$ \item  Yields another function of a new variable $Y(s)$ \item Mapping between functions in the ``time-domain'' and the ``s domain''.}
\vfill
\[\lap{y(t)}= Y(s)=\intop_0^\infty e^{-st} y(t) dt\]
\vfill

Convention: Lowercase letter in ``time-domain'', uppercase in ''s-domain''
}


}
\settoggle{covered}{false}

\slide[Fun properties of Laplace Transforms]{
\[\lap{y(t)}= Y(s)=\intop_0^\infty e^{-st} y(t) dt\]

\vfill\student{\enum{\item 
Laplace transforms ``eat'' derivatives.
\subitem{Converts ODEs into algebraic expressions: \vfill \centerline{ Solve for $Y(s)$ in the ``s-domain''  $\Leftrightarrow$ Solve the ODE in the ``t-domain''}}
\vfill
\item Laplace transforms smooth out discontinuities \subitem{ODEs with discontinuities become continuous functions of $s$}
\vfill
Before solving ODEs, let practice taking LT of simple functions.
}
}
}



\slide{\ex{$y(t)=\frac12$}\student{\hspace{3em}$\lap{y(t)}=Y(s)=\lapint{\frac12}$
\algn{&=-\frac{1}{2s}e^{-st}\Big|_0^\infty
&=-\lim_{A\to\infty}\frac{1}{2s}e^{-st}\Big|_0^A\\
&=-\frac{1}{2s}\lim_{A\to\infty}\paren{e^{-sA}-1}\\
&=\begin{cases}
\frac{1}{2s}& \text{if }s>0\\
DNE & \text{if }s\leq0
\end{cases}}
}\vfill
\centerline{
\tikzplot[\textnode{4.75}{1.5}{\overset{\mathcal{L}}{\rightarrow}}]{1}{3}{.1}{2.1}{\color{RubineRed}t}{\color{RubineRed}y(t)}{
\draw[domain=-1:6, smooth, RubineRed, thick, samples=150] plot ({\x}, {1.5});}
\hfill 
\tikzplot{1}{3}{.1}{2.1}{\color{YellowOrange}s}{\color{YellowOrange}Y(s)}{
\draw[domain=0.01:6, YellowOrange, thick, samples=100] plot ({\x}, {.02/(\x/5)});
\draw[pattern=north west lines, pattern color=\normcolor] (-2,-2) rectangle (0,4);}
}
}


\slide{
\ex{$y(t)=e^{-6t}$}\student{
\algn{\lap{y(t)}&=Y(s)=\lapint{e^{-6t}}\\
&=\intop_0^\infty{e^{-(s+6)t}} = -\frac{1}{s+6} 
\lim_{A\to\infty}\left( e^{-(s+6)A} - 1\right) \\
&=\begin{cases}
\frac{1}{s+6} & \text{if }s>-6\\
DNE & \text{if }s\leq-6
\end{cases}
}}
\vfill
\centerline{
\tikzplot[\textnode{4.75}{1.5}{\overset{\mathcal{L}}{\rightarrow}}]{1}{3}{.1}{1.5}
{\color{RubineRed}t}
{\color{RubineRed}y(t)}{
\draw[domain=-1:6, smooth, RubineRed, thick, samples=150] plot ({\x}, {.4*exp(-\x)});}
\hfill 
\tikzplot[\xcoord{-.5}{-6}]{1}{3}{.1}{1.25}{\color{YellowOrange}s}{\color{YellowOrange}Y(s)}{
\draw[domain=-.499:6, YellowOrange, thick, samples=100] plot ({\x}, {.15/(\x+.5)});
\draw[pattern=north west lines, pattern color=\normcolor] (-2,-2) rectangle (-.5,4);}
}
}

\slide[General Results]{
For any constants $C$ and $a$ we have the Laplace Transforms of the following functions $y(t)$:

\vfill
\algn{y(t)&=C&\lap{C}&=\frac{C}{s}&\text{Constant}\\
y(t)&=e^{at}&\lap{e^{at}}&=\frac{1}{s-a}&\text{Exponential Function}\\
}

\vfill
\student{
From now on, we don't worry too much about the domain of definition.\vfill
In general, there are always some conditions on $s$ for the integrals to exist.}
}


\slide[General Result: Linearity of Laplace Transforms]{
Given any two function $f(t)$ and $g(t)$ as well as any constant $c$.
\vfill
\enum{\item $\lap{f(t)+g(t)}=\student{\lapint{\paren{f(t)+g(t)}}}$\\~\\
\student{$=\lap{f(t)}+\lap{g(t)}=F(s)+G(s)$}\vfill
\item $\lap{cf(t)}=\student{c\lap{f(t)}=cF(s)}$\vfill
}
\student{\centerline{The Laplace tranform is \uline{linear}.}}
}

\slide{
\ex{$y(t)=\cos(at)$ or $y(t)=\sin(at)$} \\~\\ \student{
Euler's Identity:\[e^{iat} = \cos(at) + i\sin(at) \] 
\algn{\lap{e^{iat}} &= \lap{\cos(at)} + i \lap{sin(at)}\\
&=\frac{1}{s-ia}  = \frac{1}{s-ia} \times  \frac{s+ia}{s+ia}\\
&= \frac{s+ia}{s^2-ias+ias-i^2a^2} =  \frac{s+ia}{s^2+a^2}\\\\
&=\underbrace{\frac{s}{s^2+a^2}}_{\lap{\cos(at)}} + i \underbrace{\frac{a}{s^2+a^2}}_{\lap{\sin(at)}}   }
\vfill
Same result can be found through integration by parts (twice).
}
}


\slide{
\ex{$y(t)=t$}\vspace{-1em}
\student{
\algn{\lap{t} =\lapint{t} &= \lapint{t} \\\\ \text{let}  \larray{u=t,\; du=dt\\dv =e^{-st}dt,\; v=-\frac{e^{-st}}{s}}&\\
\intop t e^{-st} dt &= uv - \intop vdu  = \frac{te^{-st}}{s} + \intop \frac{e^{-st}}{s}\\
&= \frac{te^{-st}}{s}- \frac{e^{-st}}{s^2} = -\frac{e^{-st}(st+1)}{s^2}\\
\lap{t} =   \lim_{A\to\infty}-\frac{e^{-st}(st+1)}{s^2} \Big|_0^A& =\lim_{A\to\infty} - \frac{e^{-sA}(sA+1)}{s^2}   + \frac{1}{s^2}\\
&=\frac{1}{s^2} \qquad \text{($s>0$)}
}
For $y(t)=t^k$,  integrate by parts $k$ times.
}
}

\slide[General Results]{
For any constants $C$, $a$, $\omega$, and $k$ we have the Laplace Transforms of the following functions $y(t)$:

\vfill
\algn{y(t)&=C&\lap{C}&=\frac{C}{s}&\text{Constant}\\
y(t)&=e^{at}&\lap{e^{at}}&=\frac{1}{s-a}&\text{Exponential Function}\\
y(t)&=\cos(\omega t)&\lap{\cos{\omega t}}&=\frac{s}{s^2+\omega^2}&\text{Cosine}\\
y(t)&=\sin(\omega t)&\lap{\sin{\omega  t}}&=\frac{\omega}{s^2+\omega^2}&\text{Sine}\\
y(t)&=t^k&\lap{t^k}&=\frac{k!}{s^{k+1}}&\text{Power Function}\\
}\vfill
}

\slide[Summary]{
\vfill
\itmz{

\item Laplace transform (LT) maps $f(t) \rightarrow F(s)$ \vfill
\subitem{From "t-space" to "s-space" . \item We will learn to invert the transform in the next lecture.}\vfill
\item LT: $\lap{f(t)}=F(s)=\lapint{f(t)}$ \vfill \subitem{Evaluation of the integrals is tedious. \item We use general results to quickly transform functions. \item Many tables exist online and in textbooks.} \vfill
\item  LT is linear because the integral is linear \vfill
\enum{\item $\lap{f+g} = F(s)+G(s)$\item $\lap{cf}=cF(s)$}\vfill
}
}



\begin{comment}
\slide[Recall: $ay''+by'+cy=f(t)$]{
We have studied this DE extensively, and know how to solve it for a large class of functions $f(t)$ that are \uline{continuous}.

\vfill

\textbf{Note: }Discontinuities in $f(t)$ create kinks or jumps in the solution.
\tikzplot[\xcoord{3}{\text{kink}} ]{.1}{5}{.1}{1.5}{t}{}{

\draw[domain=0:2, thick, samples=100] plot ({\x*1.5}, {.8*exp(-\x)});
\draw[domain=2:10, thick, samples=100] plot ({\x*1.5}, {.8*exp(-\x)+(1-exp(-(\x-2)))*1.25});

}\hfil \tikzplot[\xcoord{3}{\text{jump}} ]{.1}{5}{.1}{1.5}{t}{}{

\draw[domain=0:2, thick, samples=100] plot ({\x*1.5}, {.8*exp(-\x)});
\draw[domain=2:10, thick, samples=100] plot ({\x*1.5}, {.8*exp(-(\x-2))});
\draw[thick , dashed] (3,0.1)--(3,.8);

}\vfill 

How can we deal with a RHS with, a potentially infinite number of, discontinuities?
\vfill
\student{
Idea: Apply a transformation to the RHS to make it ``nice''.
\vfill
\centerline{The Laplace Transform.}
}


}


\subsection{Laplace Transforms}
\slide[Laplace Transform Workflow]{
\vspace{-1em}
Transforms the problem from the time domain to the ``solution''-domain.
\vspace{-0.15em}
\[
\begin{array}{cccccc}
\color{RubineRed}
\tiny{t-domain} &
\color{RubineRed}
\begin{array}{c}
\text{Unknown y(t)}\\
\text{that solves ODE}
\end{array} &   \color{RubineRed}\underrightarrow{\text{Solve ODE}} &\color{RubineRed} \text{Found y(t)}\\&
\student{
\begin{array}{ccc}
\qquad\qquad & \left\downarrow \vcenter{\hrule height6em}\right.\kern-\nulldelimiterspace & \begin{array}{c}
\text{Transform }\\
\text{y(t) and}\\
\text{the ODE}
\end{array}\end{array}} &    & \student{\begin{array}{ccc}
\begin{array}{c}
\text{Invert }\\
\text{the}\\
\text{transform}
\end{array} & \left\uparrow \vcenter{\hrule height6em}\right.\kern-\nulldelimiterspace & \qquad\qquad\end{array}}\\\\
\color{YellowOrange}s-domain&\color{YellowOrange}\begin{array}{c}
\text{Unknown Y(s)}\\
\text{that solves an }\\\text{algebraic equation}
\end{array} &  \color{YellowOrange} \underrightarrow{\text{Solve alg. eq.}}   &\color{YellowOrange}\text{Isolate Y(s)}\\
\\
\end{array}
\]
\vfill
\student{
Laplace Transform of $y(t)$:
\[\lap{y(t)}=Y(s)=\int_0^\infty e^{-st} y(t)dt\]

}
}




\slide{\ex{$y(t)=\frac12$}\student{\hspace{3em}$\lap{y(t)}=Y(s)=\lapint{\frac12}$
\algn{&=-\frac{1}{2s}e^{-st}\Big|_0^\infty
&=-\lim_{A\to\infty}\frac{1}{2s}e^{-st}\Big|_0^A\\
&=-\frac{1}{2s}\lim_{A\to\infty}\paren{e^{-sA}-1}\\
&=\begin{cases}
\frac{1}{2s}& \text{if }s>0\\
DNE & \text{if }s\leq0
\end{cases}}
}\vfill
\centerline{
\tikzplot[\textnode{4.75}{1.5}{\overset{\mathcal{L}}{\rightarrow}}]{1}{3}{.1}{2.1}{\color{RubineRed}t}{\color{RubineRed}y(t)}{
\draw[domain=-1:6, smooth, RubineRed, thick, samples=150] plot ({\x}, {1.5});}
\hfill 
\tikzplot{1}{3}{.1}{2.1}{\color{YellowOrange}s}{\color{YellowOrange}Y(s)}{
\draw[domain=0.01:6, YellowOrange, thick, samples=100] plot ({\x}, {.02/(\x/5)});
\draw[pattern=north west lines, pattern color=\normcolor] (-2,-2) rectangle (0,4);}
}
}



\end{comment}
\end{document}