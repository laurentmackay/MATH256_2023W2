\input{notes.tex}


\iftoggle{dualscreen}{\setbeameroption{show notes on second screen=right}}{}
\usetikzlibrary{arrows}
\settoggle{covered}{true}
\begin{document}
\section{Lecture 17}
\subsection{Theory}

\slide[Recall: Homogeneous Linear Systems]{
 \[\dd{}{t}\vec{x} = \mathbf{A} \vec{x} \qquad \text{with } \mathbf{A} \text{ an } n\times n \text{ constant matrix} \]
\vfill
We can (usually) find $n$ solutions $\vec{x}_i(t)=e^{\lambda_i t}\vec{v_i}$ by determining \vfill
\enum{\item the eigenvalues $\lambda_i$, and \item \vfill the eigenvectors, $\vec{v}_i$} \vfill of  matrix $\mathbf{A}$.\vfill
\vfill
i.e., solving
   \[\det(\mathbf{A} - \lambda \mathbf{I}) = 0\quad\text{and}\quad (\mathbf{A} - \lambda \mathbf{I})\vec{v}= 0\]
 \vfill

 \student{ then we have the general solution
 \[\vec{x}(t) = c_1\vec{x}_1(t)+c_2\vec{x}_2(t)+\dots+c_n\vec{x}_n(t)\]}
}

\subsection{Fundamental Solutions} 

\slide[The fundamental set of solutions]{
Suppose you find $n$ linearly independent $n$-dimensional vector functions \[ \left\{ \vec{x}_1(t), \;\vec{x}_2(t),\;\dots, \;\vec{x}_n(t) \right\} \] that each solves $ \dd{}{t}\vec{x} = \mathbf{A} x$.
\vfill
\student{
This is called the \alert{fundamental set of solutions}.
\subitem{All solution lie within the span of this set \vfill \subitem{i.e., general solution $\vec{x}(t)=c_1\vec{x}_1 + c_2 \vec{x}_2+\dots+c_n\vec{x}_n$} \item The set forms a basis for all solutions. \item A new solution basis can be constructed via linear combinations.  }\vfill
Sometimes we can only find find $n-1$ eigenvalues
\subitem{We'll need some trick to find the ``missing'' fundamental solution}}

}

\slide[The fundamental matrix]{
\vfill Construct a matrix $\mathbf{X}(t)$ with each vector as a column\[\mathbf{X}(t) =\mat{cccc}{\vec{x}_1(t)  & \vec{x}_2(t)  & \cdots &\vec{x}_n(t) }=\mat{cccc}{x_{1,1}(t)  & x_{2,1}(t)  & \cdots &x_{n,1}(t) \\ 
\vdots & \ddots &&\vdots\\
\vdots && \ddots &\vdots\\
x_{1,n}(t)  & \cdots & \cdots &x_{n,n}(t)
 }\]\vfill
 \student{
Recall that: \[\text{L.I. columns of } \mathbf{X}\quad \Leftrightarrow \quad \det(\mathbf{X}) \neq 0 \quad \Leftrightarrow  \quad \mathbf{X}^{-1} \text{ exists}\]
}


}

\begin{comment}
\slide[Find the fundamental matrix for \hfill \small $\larray{ \dd{x}{t} = - 2y \\ \dd{y}{t} = -2x - 3y}$]{\vspace{-1.5em}
\student{\algn{\dd{}{t} \mat{c}{x\\y}  &= \mat{cc}{0&-2\\-2&-3} \mat{c}{x\\y} \\ \det\left(  \mat{cc}{-\lambda&-2\\-2&-3-\lambda} \right) &= 0\\
-\lambda (-3-\lambda) - 4 &=0\\
\lambda^2 +3\lambda  - 4 &= 0\\
(\lambda-1)(\lambda+4) &=0\\
\lambda_{1,2} &=1, -4}
}
}

\slide[Find the fundamental matrix for \hfill \small $\larray{ \dd{x}{t} = - 2y \\ \dd{y}{t} = -2x - 3y}$]{\student{\vspace{-2em}
\algn{\uline{\lambda_1=1:} \quad &  \mat{cc|c}{0-1&-2 &0\\-2&-3-1&0} &&   \mat{cc|c}{-1&-2 &0\\-2&-1&0}
\intertext{row 2 and row 1 are linearly dependent: $R_2-2R_1\rightarrow R_2$}
& \mat{cc|c}{-1&-2 &0\\0&0&0} &-1x-2y&=0\\
&&x&=-2y\\
\vec{v}_1&=\mat{c}{-2\\1}&\vec{x}_1(t)&=\mat{c}{-2\\1}e^{t}
}
}
}

\slide[Find the  fundamental matrix for \hfill \small $\larray{ \dd{x}{t} = - 2y \\ \dd{y}{t} = -2x - 3y}$]{\student{\vspace{-2em}
\algn{\uline{\lambda_2=-4:} \quad &    \mat{cc|c}{4&-2 &0\\-2&1&0}
\intertext{row 2 and row 1 are linearly dependent: $R_2+R_1/2\rightarrow R_2$}
& \mat{cc|c}{4&-2 &0\\0&0&0} &4x-2y&=0\\
&&2x&=y\\
\vec{v}_2&=\mat{c}{1\\2}&\vec{x}_2(t)&=\mat{c}{1\\2}e^{-4t}\\\\
\mathbf{X}(t) &= c_1\mat{cc}{-2e^{t}&e^{-4t}\\e^{t}&2e^{-4t}}
}
}
}
\end{comment}

\slide[General solution to the homogeneous IVP]{
The general solution to \[ \dd{}{t}\vec{x} = \mathbf{A}(t) x, \quad \vec{x}(t_0)=\vec{x}_0\] 
is given by \[\vec{x} = \mathbf{X}(t) \vec{c} \qquad \text{with}\;\; \vec{c}=\mathbf{X}^{-1}(t_0)\vec{x}_0  \]
\student{Proof:\vfill
We know \[ \vec{x}(t) = c_1\vec{x}_1(t)+c_2\vec{x}_2(t)+\dots+c_n\vec{x}_n(t) \quad \Leftrightarrow  \quad \vec{x}(t) = \mathbf{X}(t) \vec{c} \]
where $\vec{c}$ is a column vector, to find it we match the initial condition\vfill
\algn{ \mathbf{X}(t_0) \vec{c} &= \vec{x}_0  \quad\Rightarrow \quad \vec{c} =  \mathbf{X}^{-1}(t_0)\vec{x}_0} }
}
\settoggle{covered}{false}

\slide[Note on inverting matrices   ]{
Given the linear system of equations\[\mathbf{M}\vec{c}=\vec{b}\]
we have the formal solution \[\vec{c}=\mathbf{M}^{-1}\vec{b}.\]\vfill
\student{Simple formulas for inverting $2\times2$ matrices exist, but I do not recommend using them. \vfill
My recommendation, use the augmented matrix
\[\left[ M \big | \vec{b}\right]\]
to solve for the entries in $\vec{c}$. This way no memorization required.
}

}

\slide[Find the  solution to \hfill \small $\larray{ \dd{x}{t} = -3x - 2y \\ \dd{y}{t} = -2x - 6y}$ with $\larray{x(0)=5\\\\ y(0)=4}$]{\student{\vspace{-2em}
\algn{\ucover{\vec{x}(t)}&\ucover{=c_1\mat{c}{-2\\1}e^{-2t} + c_2 \mat{c}{1\\2}e^{-7t} }&\ucover{= \mat{cc}{-2e^{-2t}&e^{-7t}\\e^{-2t}&e^{-7t}}\mat{c}{c_1\\c_2}}\\
\vec{x}(0)&=c_1\mat{c}{-2\\1} + c_2 \mat{c}{1\\2}  = \mat{c}{5\\4}\\
&\mat{cc|c}{-2c_1 & c_2 & 5\\c_1 &2c_2&4} \intertext{$2R_2+R_1\rightarrow R_1$ }
&\mat{cc|c}{0 & 5c_2 & 13\\c_1 &2c_2&4} 
 \intertext{$-\frac25R_1+R_2\rightarrow R_2$ }
&\mat{cc|c}{0 & 5c_2 & 13\\c_1 &0&-\frac{6}{5}}&\larray{c_1=-\frac65\\c_2=\frac{13}{5}} 
}
}
}





\slide[General solution to the eigenproblem (2x2 constant matrix)]{\vspace{-1em}
\[\dd{}{t} \vec{x} = \mat{cc}{a&b\\c&d}\vec{x}\]
\vfill
\[\det\left(\mat{cc}{a-\lambda&b\\c&d-\lambda}\right)=0  \quad \Leftrightarrow  \quad \lambda^2-(a+d)\lambda + ad - bc=0\]\vfill
\student{\[\lambda = \frac{(a+d)\pm\sqrt{(a+d)^2-4(ad-bc)}}{2} \]}
\vfill
Three possibilites:

\enum{
\item ~ \student{2 distinct real eigenvalues/vectors \checkmark}
\item ~ \student{A  complex conjugate pair  of eigenvalues/vectors}
\item ~ \student{One eigenvalue is repeated (see supplement and/or DiffyQs \S 3.7)}
}
}



\slide[Find the eigenvalues for the ODE:  \hfill \small $\larray{ \dd{x}{t} = -x + 2y \\ \dd{y}{t} = -2x - y}$]{\vspace{-1.5em}
\student{\algn{\mathbf{A}  &= \mat{cc}{-1&2\\-2&-1} \\
\det\left(  \mat{cc}{-1-\lambda&-2\\2&-1-\lambda} \right) &= 0\intertext{Characteristic equation}
 (-1-\lambda)^2 + 4 &=0\\
\lambda^2 +2\lambda  + 5 &= 0\\
\lambda_{1,2} &= \frac{-2\pm \sqrt{4-20}}{2} =\frac{-2\pm \sqrt{-16}}{2}\\
 \lambda_{1,2} &=  -1 \pm 2 i }
}
}

\slide[Complex conjugate eigenvalues pairs: $\lambda_{1,2}=r\pm i\omega $]{\vspace{-.5em}
\noindent Associated eigenvectors are also complex conjugates
\student{\[\vec{v}_{1,2}=\vec{a}\pm i \vec{b} \qquad \text{where }\quad \larray{{\rm Re}(\vec{v}_1)=\vec{a} \\ {\rm Im}(\vec{v}_1)=\vec{b}}\]
\vfill
Proof:\vfill
Suppose $\vec{v}_{1}=\vec{a}+i\vec{b}$ with $\lambda_1=r+i \omega$
\algn{\mathbf{A} (\vec{a}+i\vec{b})  &= (r+i\omega) (\vec{a}+i\vec{b})\intertext{Take complex conjugate of both sides}
\mathbf{A} (\vec{a}-i\vec{b}) &=  (r-i\omega) (\vec{a}-i\vec{b})\\
\mathbf{A} \vec{v}_2 &=\lambda_2  \vec{v}_2  }
}
}
\settoggle{covered}{false}

\slide[Find the eigenvectors for the ODE:  \hfill \small $\larray{ \dd{x}{t} = -x + 2y \\ \dd{y}{t} = -2x - y}$]{\vspace{-1.5em}
\student{\[ \lambda_{1,2} =  -1 \pm 2 i\]\vspace{-3em}
\algn{ \intertext{\uline{$\lambda_{1} =  -1 + 2 i$}} \mat{cc|c}{-1-(-1+2i)&2&0\\-2&-1-(-1+2i)&0} & \quad \rightarrow &\mat{cc|c}{-2i &2&0\\-2&-2i&0} \\
\intertext{$R_2-iR_1\rightarrow R_2$ and $\frac12 R_1 \rightarrow R_1$}
\mat{cc|c}{-i &1&0\\0&0&0} &&\rarray{ -i x + y = 0\\y=ix}\\\\
\vec{v}_1=\mat{c}{1\\i} = \mat{c}{1\\0}+i\mat{c}{0\\1}&  & \Rightarrow \vec{v}_2=\mat{c}{1\\-i} }
}
}

\slide[Find the general solution for the ODE:  \hfill \small $\larray{ \dd{x}{t} = -x + 2y \\ \dd{y}{t} = -2x - y}$]{\vspace{-1.5em}
\student{\[ \lambda_{1,2} =  -1 \pm 2 i \qquad \vec{v}_{1,2}=\mat{c}{1\\\pm i}\] 
\algn{\vec{x}(t) &= c_1\underbrace{e^{(-1+2i)t}\mat{c}{1\\ i} }_{\vec{x}_1(t)}+c_2 \underbrace{e^{(-1-2i)t}\mat{c}{1\\- i}}_{\vec{x}_2(t)}
}\vfill
Here both eigensolutions are complex-valued.
\vfill
\centerline{Notice that $\vec{x}_1$ and $\vec{x}_2$ are complex conjugates.}
\vfill
i.e.,
\[\vec{x}_2={\rm Re}(\vec{x}_1)-i{ \rm Im}(\vec{x}_1).\]
\vfill
}
}

\slide[Conversion to real-valued vector solutions:  $\lambda_{1,2}=r\pm i\omega $]{\vspace{-1em}
Suppose you have a complex-valued solution basis  \vspace{0.5em}
\[\vec{x}(t) = c_1\vec{x}_1(t)+c_2\vec{x}_2(t) \qquad\text{with }\arr{c}{\vec{x}_1={\rm Re}(\vec{x}_1)+i{ \rm Im}(\vec{x}_1)\\\vec{x}_2={\rm Re}(\vec{x}_1)-i{ \rm Im}(\vec{x}_1) }\]

We want a real-valued solution basis. Try a  linear combination of $\vec{x}_1\;\&\;\vec{x}_2$.

\student{
\algn{
\vec{y}_1&=\frac{\vec{x}_1+\vec{x}_2}{2}=\frac{{\rm Re}(\vec{x}_1) + {\rm Re}(\vec{x}_1)}{2} = {\rm Re}(\vec{x}_1)\in\mathbb{R}^2\\
\vec{y}_2&=\frac{\vec{x}_1-\vec{x}_2}{2i}=\frac{i{\rm Im}(\vec{x}_1) + i{\rm Im}(\vec{x}_1)}{2i} = {\rm Im}(\vec{x}_1)\in\mathbb{R}^2
}
Then our solution can be written as
\algn{\vec{x}(t)&=c_1\vec{y}_1+c_2\vec{y_2}\\
&=c_1{\rm Re}(\vec{x}_1) + c_2{\rm Im}(\vec{x}_1)}
}

}

\slide[Find the real-valued solution for the ODE:  \hfill \small $\larray{ \dd{x}{t} = -x + 2y \\ \dd{y}{t} = -2x - y}$]{\vspace{-2em}
\[\vec{x}_1(t) = e^{(-1+2i)t}\mat{c}{1\\ i} \student{= e^{-t}\underbrace{(\cos(2t)+i\sin(2t))}_{=e^{2it}\text{ from Euler's Identity}}\bigg(\underbrace{\mat{c}{1\\0}}_{\vec{a}} +i \underbrace{\mat{c}{0\\1}}_{\vec{b}}\bigg)}\] 
\vspace{-.5em}
\student{
\algn{{\rm Re}(\vec{x}_1(t)) & = e^{-t}\left(\cos(2t)\mat{c}{1\\0} -\sin(2t)\mat{c}{0\\1} \right)= e^{-t}\mat{c}{\cos(2t)\\-\sin(2t)}\\
{\rm Im}(\vec{x}_1(t)) & = e^{-t}\left( \sin(2t)\mat{c}{1\\0} +\cos(2t)\mat{c}{0\\1}\right) = e^{-t}\mat{c}{\sin(2t)\\\cos(2t)}\\
\vec{x}(t) &= c_1e^{-t}\mat{c}{\cos(2t)\\-\sin(2t)}+c_2e^{-t}\mat{c}{\sin(2t)\\\cos(2t)}\\
&=e^{-t}\mat{c}{c_1 \cos(2t) +c_2\sin(2t)\\-c_1\sin(2t)+c_2\cos(2t)}
}

}
}

\slide[Solve the IVP: \hfill \small $\larray{ \dd{x}{t} = -x + 2y \\ \dd{y}{t} = -2x - y}\quad $with$\quad \larray{x(0)=2\\\\ y(0)=7}$]{
General Solution:
\[\vec{x}(t)=e^{-t}\mat{c}{c_1 \cos(2t) +c_2\sin(2t)\\-c_1\sin(2t)+c_2\cos(2t)}\]\vfill
\student{
\algn{\vec{x}(0)&=\mat{c}{2\\7}=\mat{c}{c_1\\c_2}\quad \Rightarrow\quad \larray{c_1=2\\c_2=7}
}
}\vfill
}


\end{document}