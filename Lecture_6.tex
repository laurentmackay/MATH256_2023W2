\input{notes.tex}

\iftoggle{dualscreen}{\setbeameroption{show notes on second screen=right}}{}

\usepackage{circuitikz}

\begin{document}
\settoggle{covered}{true}
\section{Lecture 6}
\subsection{Preamble}
\slide[Recall]{
We saw that homogeneous constant coefficient second order linear IVPs\[ay''+by'+cy=0, \quad \text{with } y(0)=y_0, \;y'(0)=v_0\]
have a general solution \[y=c_1e^{r_1t}+c_2e^{r_2t},  \quad \text{with } r_{1,2} = \frac{-b\pm\sqrt{b^2-4ac}}{2a} \text{ for } r_1\neq r_2.\]
Three Questions:
\vfill
\student{\enum{\item What applications motivate these ODEs? \vfill\item How do these solutions behave? \vfill \item What happens when $r_1=r_2$?}\vfill} 
}

\subsection{Derivation of ODE}
\slide[Derivation of spring-dashpot ODE: ]{
\twomini[.45]{.25}{.65}{\includegraphics[width=\columnwidth, angle=90, origin=c]{images/spring_dashpot.pdf}}{
$x(t)=$ displacement from rest position\student{\subitem{$x=0 \Rightarrow$ no elastic  restoring force}}
\vfill
Newton's 2$^{nd}$ Law:\[ F = ma \qquad \text{where $a = \ddn{x}{t}{2}$}\]


}
\algn{F&=\text{sum of forces}\\&=\underbrace{\carray{\text{elastic restoring} \\ \text{force}}}_{\student{\carray{\text{Hooke's Law}\\=-kx }}} + \underbrace{\text{drag force}}_{\student{\carray{\text{opposes motion}\\ =-\beta\dd{x}{t}}}} + \quad \underbrace{\text{external forces}}_{\carray{\student{f(t)}}}\\
&= \student{-kx - \beta \dd{x}{t} +f(t)} }
}


\slide[Derivation of spring-dashpot ODE:]{
\twomini[.38]{.25}{.65}{\includegraphics[width=\columnwidth, angle=90, origin=c]{images/spring_dashpot.pdf}}{
$x(t)=$ displacement from rest position\subitem{$x=0 \Rightarrow$ no elastic  restoring force}
\vfill
Newton's 2$^{nd}$ Law:\[ F = ma \qquad \text{where $a = \ddn{x}{t}{2}$}\]


}
\algn{F&=-kx - \beta \dd{x}{t} +f(t) &
\student{\Rightarrow m \ddn{x}{t}{2}} &\student{= -kx - \beta \dd{x}{t} +f(t)}}

\student{\[\boxed{ m x\pp + \beta x\p  + kx=  f(t) }\]
\vfill
With $f(t)=0$ (no external forcing) we get an ODE of the form \[ay''+by'+cy=0\]
}
}


\subsection{Equivalent applications}

\slide[Torsional motion of a weight on a twisted shaft:]{
\twomini{.32}{.65}{\includegraphics[width=\columnwidth]{images/twisted_shaft.pdf}}{
\vfill\[ I\ddn{\theta}{t}{2} +c \dd{\theta}{t} + k\theta = T(t)\]\vspace*{\fill}
}
}

\slide[L-R-C series circuits:]{
\twomini{.45}{.55}{
\begin{circuitikz} \draw
       (0,-1.5)   to[vsourcesin, l=\textcolor{black}{$E(t)$}] (0,1.5)
	[black] -- (0.5,1.5)
        to[L, l=$L$, black] (1.5,1.5) -- (2.5,1.5)
        to[C, l=$C$, black] (2.5,0) -- (2.5,-1.5)
        to[R, l=$R$, black] (.5,-1.5) -- (0,-1.5) ;
    \end{circuitikz}
}{\vfill
Q=charge on capacitor\\~\\
$\dd{Q}{t}$=current in circuit\\~\\
$E(t)$ = applied voltage\\
\vfill
Kirchoff's Laws:
\[ L\ddn{Q}{t}{2} +R \dd{Q}{t} + \frac1CQ = E(t)\]\vspace*{\fill}
}
}

\slide[Small oscillations of a pendulum:]{
\twomini{.45}{.55}{\includegraphics[width=\columnwidth]{images/pendulum.pdf}}{
\vfill\[ mL^2\ddn{\theta}{t}{2}=-cL \dd{\theta}{t} -mgL\theta + F(t) \]\vspace*{\fill}
}
}

\slide[Equivalence of Problems]{
These 4 physical systems are modelled identically by:\[ay\pp+by\p+cy=f(t)\]
\vfill
Constants have different physical meaning (\& units)
\small\vfill
\begin{tabular}{|>{\centering}m{2cm}|>{\centering}m{2cm}|>{\centering}m{2cm}|>{\centering}m{2cm}|>{\centering}p{2cm}|}
\hline 
\textbf{System} & \textbf{a} & \textbf{b} & \textbf{c} & \textbf{f(t)}\tabularnewline
\hline 
\hline 
\textbf{Spring Dashpot} & Mass & Damping Coeff. & Spring Constant & Applied Force\tabularnewline
\hline 
\textbf{Pendulum} & Mass x (Length)$^{2}$ & Damping x Length & Gravitational Moment & Applied Moment\tabularnewline
\hline 
\textbf{Series Circuit} & Inductance & Resistance & Capacitance$^{-1}$ & Imposed Voltage\tabularnewline
\hline 
\textbf{Twisted Shaft} & Moment of Inertia & Damping & Elastic Shaft Constant & Applied Torque\tabularnewline
\hline 
\end{tabular}

}

\subsection{The characteristic polynomial}

\slide[Roots of the characteristic equation (polynomial)]{\vspace{-2em}
\algn{ay''+by'+cy&=0 &\text{guess: } y&=e^{rt}\\
ar^2e^{rt}+bre^{rt}+ce^rt&=0 &\Rightarrow \underbrace{ar^2+br+c}_{\text{char. poly.}}&=0 }

\[r=r_{1,2}=\frac{-b\pm\sqrt{b^2-4ac} }{2a}\]
Three main cases:\student{
\enum{\item Two distinct real roots: $\underbrace{b^2-4ac}_{\text{discriminant}}>0$ \vfill
\item Repeated real roots: discriminant $= 0$ \vfill \item Complex conjugate roots: discriminant $<0$}}
}

\slide[Case 1:  distinct real roots]{
\[y_h(t)=c_1e^{r_1 t} + c_2 e^{r_2 t}; \qquad r_{1,2}=\frac{-b\pm\sqrt{b^2-4ac} }{2a}\] 
 Two major subcases:
\student{\enum{
\item $ac>0$ :\vfill
 Both roots have the same sign.\vfill
$y_1$ and $y_2$ both grow or decay exponentially.
\vfill
\item $ac<0$:
\vfill
The two roots have opposite sign.
\vfill
One solution grows exponentially, the other decays exponentially.

}}


}




\slide[Qualitative Behaviour: distinct real roots]{\vspace{-1em}
Sum of real exponential functions, three subcases:\vfill
\student{
\enum{
\item All positive roots, $0<\textcolor{blue}{r_1}<\textcolor{red}{r_2}$.\\
\tikzplot{.1}{4}{.1}{1.25}{t}{}{
\draw[red ,domain=0:4, samples=120,  thick] plot ({\x}, {0.01*exp(2*\x)});
\draw[blue ,domain=0:4, samples=120,  thick] plot ({\x}, {0.01*exp(\x)});
}
\vfill
\item  Mixed roots, $\textcolor{blue}{r_1}<0<\textcolor{red}{r_2}$.\\
\tikzplot{.1}{4}{.1}{1.25}{t}{}{
\draw[blue ,domain=0:4, samples=120,  thick] plot ({\x}, {1*exp(-2*\x)});
\draw[red ,domain=0:4, samples=120,  thick] plot ({\x}, {0.05*exp(\x)});
}
\vfill
\item All negative roots,  $\textcolor{blue}{r_1}<\textcolor{red}{r_2}<0$.
\\
\tikzplot{.1}{4}{.1}{1.25}{t}{}{
\draw[red ,domain=0:4, samples=120,  thick] plot ({\x}, {exp(-1*\x)});
\draw[blue ,domain=0:4, samples=120,  thick] plot ({\x}, {exp(-2*\x)});
}
}
}
}


\subsection{Case 2: repeated root}

\slide[Case 2: Repeated real root ($r_1=r_2=r$) \hfill $y_h=c_1y_1 + c_2y_2$]{
Straighforward solution \[y_1=e^{rt}\]  with \[r = \frac{-b\pm\sqrt{b^2-4ac} }{2a} = \frac{-b}{2a}\]
We need another solution that is linearly independent of $y_1$
\student{
\vfill

\vfill
Lets try \[y_2=q(t) y_1(t)\]
Unique choice \[q(t)=Ct \qquad \Rightarrow \qquad y_2(t) = te^{rt}\]
}
}
\slide[Proof that $y_2=te^{rt}$]{\vspace{-2.5em}
\[ay\pp + by\p+cy=0 \qquad \text{with } b^2-4ac=0\quad \Rightarrow \quad r_{1,2}=r=\frac{-b}{2a} \]
\centerline{$$}\vspace{-1.75em}
\algn{\text{Try: } y_2 &= q(t)e^{rt}, \quad 
y_2\p = q'e^{rt}+rqe^{rt} \\
y_2\pp &= q''e^{rt} + 2rq'e^{rt}+ r^2qe^{rt} 
\intertext{plug these into the ODE}
&a\left(  q''e^{rt} + 2rq'e^{rt}+ r^2qe^{rt}  \right) + b \left( q'e^{rt}+rqe^{rt}\right) + c q e^{rt} = 0 \\
& a q''e^{rt} +\left( 2ar +b \right)q'e^{rt} +\underbrace{ (ar^2+br+c)}_{\text{char. poly.}=0}qe^{rt} =0 
\intertext{sub in $r=\frac{-b}{2a}$} 
&aq''e^{rt} +\underbrace{\left( \cancel{2a}\frac{-b}{\cancel{2a}} +b \right)}_{0}e^{rt} =0 \\
& aq''e^{rt} = 0 \quad \Rightarrow  \quad  q''=0 \quad \Rightarrow  \quad  q(t) = Ct+D }
$D=0$ due to linear independence between $y_2$ and $y_1$
}

\slide[Checking for linear independence when $r_1=r_2=r$]{
Both $y_1=e^{rt}$ and $y_2=te^{rt}$ are solutions.
\vfill
$W(y_1,y_2)(t)=y_1y_2\p -y_1\p y_2$?
\student{\algn{W&=e^{rt}\left(rte^{rt} + e^{rt}\right)-re^{rt}te^{rt}\\
&=\cancel{rte^{2rt}}+e^{2rt}-\cancel{rte^{2rt}}\\
&=e^{2rt}  \neq 0 }
\vfill
$y_1$ and $y_2$ are linearly independent!
\vfill
General solution: $y_h = c_1e^{rt} + c_2 te^{rt}$

}
}

\slide[Solve the IVP: $y\pp+4y\p+4y=0 \qquad \rarray{y(0)=2\\y\p(0)=0}$ ]{

\student{\algn{r_{1,2} &= \frac{-4\pm\sqrt{16-4\cdot 4}}{2} = \frac{-4\pm 0}{2} =,-2\\
y_h&=c_1e^{-2t}+c_2te^{-2t} \intertext{initial conditions:}
y(0)=2&=c_1\\
y\p(0)=0&=-2c_1 + c_2(e^{-2t}-2te^{-2t})\evalat{t=0}{} \\&= -4 +c_2 \qquad \Rightarrow \qquad c_2=4\\
\Aboxed{y(t) &= 2 e^{-2t} + 4 t e^{-2t}}
}
}

}

\settoggle{covered}{false}
\subsection{Case 3: complex conjugate roots}
\slide[Review: Complex Numbers]{
\twomini[.9]{.5}{.5}{
Square root of a negative number:
\vfill
Suppose $w>0$
\student{
\algn{\ucover{\sqrt{-w}} & \ucover{=} i\sqrt{w}\\
i&=\text{imaginary unit}\\\\
i \times i &= -1}
}
\vspace{4em}
}{
Suppose $a,b\in \mathbb{R}$\vfill
Complex Number: $z=a+ib$\\
Complex Conjugate: $\bar{z} =a-ib $

\student{\algn{\ucover{\frac{z+\bar{z}}{2}} & \ucover{=}  \frac{2a}{2}\; =  a = \text{Re}(z)\in \mathbb{R}\qquad \\ \\
\ucover{\frac{z-\bar{z}}{2i}} & \ucover{=} \frac{2ib}{2i}=  b = \text{Im}(z) \in \mathbb{R}\qquad
}}

\vspace{3em}
}

}
\slide[Case 3: Complex roots ($b^2-4ac<0$ )  \hfill $y_h=c_1y_1 + c_2y_2$]{
Roots are given by:\algn{r_1&=\alpha + i \beta & \text{where } i =\sqrt{-1} \\
r_2&=\alpha - i \beta}
\student{\vspace{-.75em}\[\alpha = \frac{-b}{2a},\qquad \beta =\frac{\sqrt{4ac-b^2}}{2a}\]}\vfill
The two functions $y_1=e^{(\alpha +i\beta )t}$ \& $y_2=e^{(\alpha  - i\beta) t}=\bar{y}_1$ are solutions.\vfill
\student{What is the exponential of a complex number?\vfill Euler's formula:\[e^{\pm i \alpha} = \cos \alpha \pm i \sin \alpha\]}

}
\settoggle{covered}{false}
\slide{\vspace{-1.5em}
\student{
\algn{y_{1,2}=e^{(\alpha  \pm i\beta ) t} &= e^{\alpha t} e^{\pm i \beta t} \\ &=  \underbrace{e^{\alpha t}}_{\text{Real}} \underbrace{[\underbrace{ \cos(\beta t)}_{\text{Real}} \pm\underbrace{ i \sin (\beta t)}_{\text{Imaginary}}]}_{\text{Complex Conjugates}}&y_1=\bar{y}_2&}

We want purely real solutions
\vfill
\algn{\tilde{y}_1 & = \frac{y_1+y_2}{2}=\text{Re}(y_1) =e^{\alpha t} \cos(\beta t)  \in \mathbb{R} \\
\tilde{y}_2&=\dfrac{y_1-y_2}{2i}=\text{Im}(y_1)=e^{\alpha t} \sin (\beta t) \in \mathbb{R}
}
\vfill
}
}

\slide[Complex roots ($r_{1,2}=\alpha\pm i \beta$)]{
The functions $y_1=e^{\alpha t} \cos (\beta t)$ and $y_2=e^{\alpha t} \sin (\beta t)$ are linearly independent real solutions. 

\student{
\vfill
Sketch the two functions if you are not convinced.\vfill
General solution: $y_h = c_1e^{\alpha t} \cos (\beta t) + c_2 e^{\alpha t} \sin (\beta t)$}

}

\slide[Find the general solution to: $y\pp+6y=0 $ ]{
\student{\algn{r_{1,2} &= \frac{\pm\sqrt{-4\cdot 6}}{2} = \pm \sqrt{-6} = \pm i\sqrt{6} \\
y_h&=c_1 \cos \left(\sqrt{6}t\right) + c_2 \sin\left( \sqrt{6}t\right) 
}
}
}

\slide[Solve the IVP: $y\pp+2y\p+5y=0 \qquad \rarray{y(0)=1\\y\p(0)=-1}$ ]{
\student{\algn{r_{1,2} &= \frac{-2\pm\sqrt{4-4\cdot 5}}{2} = \frac{-2\pm \sqrt{-16}}{2}=-1\pm \frac{\sqrt{16}}{2} i = -1\pm 2i \\
y_h&=e^{-t} \left(c_1 \cos \left(2t\right) + c_2 \sin\left( 2t\right) \right)\intertext{initial conditions:}
y(0)=1&=c_1\\
y\p(0)=-1&= -c_1+ \left(-2 c_1 \sin \left(0\right) + 2 c_2 \cos\left( 0 \right) \right) = -c_1+2c_2\\
-1&=-1+2c_2  \qquad \Rightarrow  \qquad  c_2=0\\\\
\Aboxed{y(t) &= e^{-t}\cos (2t)}
}
}
}

\slide[Qualitative Behaviour: complex roots]{\vspace{-1em}
Three subcases:

\student{
\enum{
\item  $\alpha < 0\quad\Rightarrow\quad$ Exponentially decaying oscillations.
\tikzplot{.1}{10}{.75}{.75}{t}{}{
\draw[domain=0:10, samples=300,  thick, color=blue] plot ({\x}, {0.75*exp(-.2*\x)*sin(360*\x)});
\draw[domain=0:10, samples=300,  thick, color=red] plot ({\x}, {0.75*exp(-.2*\x)*cos(360*\x)});
\draw[domain=0:10, samples=300,  thick, dashed] plot ({\x}, {0.75*exp(-.2*\x)})  node[left, xshift=-4cm, yshift=.4cm]{$e^{\alpha t}$};
\draw[domain=0:10, samples=300,  thick, dashed] plot ({\x}, {-0.75*exp(-.2*\x)}) node[left, xshift=-4cm, yshift=-.4cm]{$-e^{\alpha t}$};
}
\vfill
\item   $\alpha = 0 \quad\Rightarrow\quad$ Sustained periodic oscillations.
\tikzplot{.1}{10}{.75}{.75}{t}{}{
\draw[domain=0:10, samples=300,  thick, color=blue] plot ({\x}, {.72*sin(360*\x)});
\draw[domain=0:10, samples=300,  thick, color=red] plot ({\x}, {.72*cos(360*\x)});
}
\vfill
\item  $\alpha > 0\quad\Rightarrow\quad$ Exponentially growing oscillations.
\tikzplot{.1}{10}{.75}{.75}{t}{}{
\draw[domain=0:10, samples=300,  thick, color=blue] plot ({\x}, {0.1*exp(.2*\x)*sin(360*\x)});
\draw[domain=0:10, samples=300,  thick, color=red] plot ({\x}, {0.1*exp(.2*\x)*cos(360*\x)});
\draw[domain=0:10, samples=300,  thick, dashed] plot ({\x}, {0.1*exp(.2*\x)})  node[left, xshift=-4cm, yshift=-.15cm]{$e^{\alpha t}$};
\draw[domain=0:10, samples=300,  thick, dashed] plot ({\x}, {-0.1*exp(.2*\x)}) node[left, xshift=-4cm, yshift=.15cm]{$-e^{\alpha t}$};
}
}
}
}

\slide[Summary]{
\itmz{
\item For homogeneous linear constant coeffiecient ODEs:
\subitem{Pick an ansatz (e.g., $e^{rt}$) \item Write down the characteristic equation
\item Find the roots \subitem{If you don't have enough functions, make a new one by multiplying by $t$}}
\vfill
\item Write down the general solution according to the roots
\subitem{ Real and distinct $\Rightarrow y_h = c_1e^{r_1t} + c_2 e^{r_2t}$ }
\subitem{ Real and repeated $\Rightarrow y_h = c_1e^{rt} + c_2 te^{rt} $ }
\subitem{ Complex $\Rightarrow y_h= c_1e^{\alpha t} \cos (\beta t) + c_2 e^{\alpha t} \sin (\beta t)$  \hfill with $r_{1,2} = \alpha\pm i\beta$}
\vfill
\item Fit the constants $c_1$ and $c_2$ to the initial conditions
}
}


\end{document}