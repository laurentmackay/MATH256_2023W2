\input{notes.tex}


\iftoggle{dualscreen}{\setbeameroption{show notes on second screen=right}}{}
\usetikzlibrary{arrows}

\begin{document}

\section{Lecture 15}

\subsection{Introduction}
\slide[Recall: we have always rearranged products as sums]{
\vfill
\ex{$Y(s)=\frac{1}{s^2(s^2+1)} \student{= \dfrac{A}{s} + \dfrac{B}{s^2} +\dfrac{Cs+D}{s^2+1} }$}
\vfill
After partial fraction decomposition...\vfill
\student{
\algn{A=0, B&=1, C=0, D=-1\\
Y(s) &= \frac{1}{s^2}-\frac{1}{s^2+1}\\&=\lap{t} -\lap{\sin(t)}\\\\
y(t)&=t-\sin(t) 
}
\vfill
\centerline{It is possible to deal with the product directly!}

}
}
\subsection{Convolution Integrals}

\slide[Convolutions]{
We denote the convolution of two functions $f$ and $g$ by the symbol $f \ast g$, with \[h(t) =(f\ast g)(t) =\int_0^t f(\tau) g(t-\tau) d\tau\]
\textbf{Convolutions are:} 
\begin{enumerate}
    \item Commutative/Symmetric
    \subitem{ $f*g=g*f$ }
    \item Linear
    \subitem{ $f*(g+h)=f*g+f*h$, where $h$ is a function
    \item $f*(cg)=c(f*g)=(cf)*g$, where $c$ is a constant}
    \item Associative \subitem{$f*(g*h)=(f*g)*h$}
\end{enumerate}
\vfill
Convolutions are useful for inverting products of Laplace Transforms\\~\\

}


\subsection{Convolution Theorem}
\slide[Convolution Theorem]{\vspace{-.75em}
If $f(t)=\lapinv{F(s)}$ and $g(t)=\lapinv{G(s)}$ are known functions, then \vfill \[\boxed{ \lapinv{F(s) \cdot G(s)} =f \ast g } = \int _0^t f(\tau) g(t-\tau) d\tau =\int_0^t g(\tau)f(t-\tau)d\tau\]



\vfill
or conversely
\vfill
\[\boxed{\lap{f \ast g} = F(s) \cdot G(s)}\]
}
\slide[Proof of the convolution theorem with $h(t)=f(t) \ast g(t)$]{\vspace{-2em}
\algn{\lap{h(t)} = \int_0^\infty e^{-st} h(t) dt = \int_{t=0}^\infty  \int _{\tau=0}^t f(\tau) g(t-\tau) e^{-st} d\tau\; dt}
\threemini[.32]{.3}{.3}{,3}{
\tikzplot{.1}{3}{.1}{1.75}{\tau}{t}{
\draw[domain=0:2 ] plot (\x,2*\x/3) node[right, yshift=-.1cm]{$t=\tau$};
\draw[pattern=horizontal lines, pattern color=\normcolor, thick] (0,0) -- (4,8/3) -- (0, 8/3);
}}{\[\carray{\text{equivalent areas}\\ \Leftrightarrow\\ \text{switch integration order}}\]}{
\tikzplot{.1}{3}{.1}{1.75}{\tau}{t}{
\draw[domain=0:2 ] plot (\x,2*\x/3) node[right, yshift=-.1cm]{$t=\tau$};
\draw[pattern=vertical lines, pattern color=\normcolor, thick] (0,0) -- (4,8/3) -- (0, 8/3);
}
}
\algn{&= \int_{\tau=0}^\infty  \int _{t=\tau}^\infty f(\tau) g(t-\tau) e^{-st} dt\; d\tau\\
&=\int_{\tau=0}^\infty f(\tau) e^{-s\tau} \int_{t=\tau}^\infty g(t-\tau) e^{-s(t-\tau)}d\tau\;dt &\text{let }u=t-\tau\\
&=\underbrace{\int_{\tau=0}^\infty f(\tau) e^{-s\tau} d\tau}_{F(s)} \underbrace{\int_{u=0}^\infty g(u) e^{-su} du}_{G(s)}  &t=\tau\Rightarrow u =0\\
&=F(s)G(s)}

}


\slide{\ex{ $y\pp+y=t \quad \text{with} \quad y(0)=y\p(0)=0$.} \\~\\  Use the convolution theorem to find $y(t)$.\vfill

\student{\algn{s^2Y(s) + Y(s)  &= \frac{1}{s^2}\\
Y(s)&=\frac{1}{s^2\paren{s^2+1}} =  \frac{1}{s^2} \cdot \frac{1}{s^2+1}=\lap{t^2} \cdot \lap{\sin(t)}\\
y(t) &= t \ast \sin(t)=\intop_0^t (t-\tau)\sin(\tau) d \tau\\
&\overset{\text{by parts}}{=} \left[-t \cos (\tau )-\sin (\tau )+\tau  \cos (\tau )\right]_{\tau=0}^t\\
&=\cancel{-t \cos (t)}-\sin (t)+\cancel{t\cos (t )}+t\cos(0)+\cancel{\sin(0)}+\cancel{0\cos(0)}\\
&=t-sin(t)
}
}\vfill
}

\slide{\ex{ $y\pp+y=g(t)\quad \text{with} \quad y(0)=3, \;y'(0)=5$.} \\~\\  Use the convolution theorem to find an general expression for $y(t)$.
\student{
\algn{s^2Y(s)-3s-5+Y(s)&= G(s)\\
(s^2+1)Y(s) &= G(s)+3s+5\\
Y(s)&=\frac{G(s)}{s^2+1} + 3\frac{s}{s^2+1} + 5\frac{1}{s^2+1}\\\\
y(t)&=\sin(t) \ast g(t) + 3\cos(t) + 5 \sin(t)\intertext{we call $\sin(t)$ the impulse response function.}
y(t) & = \underbrace{\intop_0^t \sin(t-\tau) g(\tau) d\tau}_{\text{particular part}} + \underbrace{ 3\cos(t) + 5 \sin(t)}_{\text{homogeneous part}}  }
\vfill
\centerline{This approach allows us to solve whole classes of ODEs at once.}

}

}

\slide[Impulse Response Function]{
Suppose we want to solve \[ay''+by'+cy=g(t), \quad \text{with } y(0)=y_0, y'(0)=v_0,\]
then we can define the impulse response function $f(t)$ as the solution to  \[af''+bf'+cf=\delta(t), \quad \text{with } y(0)=0, y'(0)=0\]
\student{\algn{
F(s)&=\frac{1}{as^2+bs+c} &f(t)&=\lapinv{F(s)}\intertext{then} 
Y(s)&=G(s)\frac{1}{as^2+bs+c}+\frac{(as+b)y_0+av_0}{as^2+bs+c}\\
y(t)&=g(t)\ast f(t) +\lapinv{ \frac{ay_0s + (by_0+av_0)}{as^2+bs+c}}
}}

}

\slide[]{Use the convolution theorem to find $c>1$ such that the solution to  \[y''+y=\delta(t-1)-\delta(t-c) \quad \text{with } y(0)=0, y'(0)=0\] is zero for $t \geq c$. 
\student{
Assume $t>c$, then 
\algn{y(t)&=\intop_0^t f(t-\tau)(\delta(\tau-1)-\delta(\tau-c)) dt= f(t-1)-f(t-c) \intertext{where $f(t)$ is the impulse response function, i.e., it solves} f''+f&=\delta(t) \\s^2F(s)+F(s)&=1\quad \Rightarrow F(s)=\frac{1}{s^2+1}\\f(t)&=\sin(t)\\
\sin(t-1)&-\sin(t-c)=0\quad \Rightarrow t-c+ 2m \pi=t-1 \quad m \in \mathbb{Z}^+}\[c=1+2m\pi\]
}
}

\slide[]{
ODE:
 \[y''+y=\delta(t-1)-\delta(t-c) \quad \text{with } y(0)=0, y'(0)=0\] \vfill
 Solution:
\[y(t) = u(t-1) \sin (t-1)- u (t-c) \sin (t -c), \quad \text{with } c=1+2m\pi \]\vfill
\centerline{\includegraphics[width=8.5cm]{images/Double_Dirac_SHM.pdf}}
}
\slide{
The convolution theorem also allows us to solve integro-differential equations. \ex{$y'(t)=e^t+2\intop_0^ty(t-\tau)e^\tau d\tau\quad\text{with}\quad y(0)=0$}

\student{\algn{sY(s) =\frac{1}{s-1} +&2 \lap{\intop_0^ty(t-\tau)e^\tau d\tau} = \frac{1}{s-1} +2 Y(s) \frac{1}{s-1}\\
\left(s-\frac{2}{s-1}\right) Y(s) &= \frac{1}{s-1} \qquad \Rightarrow \qquad
\frac{\overbrace{s^2-s-2}^{(s-2)(s+1)}}{\cancel{s-1}} Y(s) = \frac{1}{\cancel{s-1}}\\
Y(s)&=\frac{1}{(s-2)(s+1)}}}
\settoggle{covered}{false}
\student{
\algn{
y(t)&=\intop_0^te^{2\tau} e^{-(t-\tau)}d\tau  = e^{-t} \intop_0^t e^{3\tau} d\tau=\frac{e^{-t}}{3}\left[ e^{3\tau}\right]_{\tau=0}^t\\
&=\frac13\left(e^{2t}-e^{-t}\right)
}
}
}
\settoggle{covered}{false}
\slide[Convolution Theorem: Special Case]{
Let $f(t)$ be an integrable function and $g(t)=1$.
\algn{\lap{f(t)\ast g(t)}&=F(s)G(s)\\
\lap{\intop_0^t f(\tau) d\tau}&=\student{F(s)\lap{1}}\\
&\student{=\frac{F(s)}{s} \qquad \text{or}\qquad \lapinv{\frac{F(s)}{s}} =\intop_0^t f(\tau) d\tau }
}\ex{Find the inverse Laplace tranforms of $H(s)=\frac{1}{s(s+7)}$}
\student{\algn{h(t)&=\intop_0^t e^{-7\tau} d\tau=-\frac17\left[ e^{-7\tau}\right]_{\tau=0}^t = -\frac17 \left( e^{-7t} -1 \right)}}
}
\end{document}