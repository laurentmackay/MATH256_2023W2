\input{notes.tex}

\iftoggle{dualscreen}{\setbeameroption{show notes on second screen=right}}{}

\begin{document}


\section{Lecture 4}
\settoggle{covered}{true}
\subsection{Introduction}
\slide[Recall:]{\vspace{-1em}
We saw how to solve ODEs of the following forms:
 \enum{\item $y\p=p(t)\cdot g(y) \qquad -\qquad \text{Separable Equations}$
 \item $y\p+ay = g(t) \qquad -\qquad \text{Constant Coefficient}$}
\vfill
Now we want to solve something a little more general...
\[y\p + p(t) y = g(t)\]
\student{We use a trick involving the product rule:
\algn{\dd{}{t}(y \cdot \mu ) &= y'\mu+y \mu' = h(t)  &\text{where $\mu=\mu(t)$}&\intertext{solving for the product $y\cdot \mu $}
{\text d} (y \cdot \mu)  &= h(t) {\text d} t & \text{integrate}\Rightarrow (y \cdot \mu )&= H(t) + C\\
&&\Rightarrow  y=\frac{H(t)+C}{\mu(t) }
}
}

}

\settoggle{covered}{false}


\slide{
\ex{Solve $y\p+2ty=t$}\\
\student{First we do something very odd, multiply everything by $e^{t^2}$
\algn{ e^{t^2}y\p + \underbrace{2te^{t^2}}_{\small \dd{}{t}e^{t^2}}y &= te^{t^2} &\rightarrow
 \underbrace{e^{t^2}\dd{}{t}y+\dd{}{t}e^{t^2}y}_{\text{product rule applied to } (y\cdot e^{t^2})} &=t e^{t^2} \\\\
\dd{}{t}\paren{e^{t^2}y} &= t^2e^{t^3} & \int d\paren{e^{t^2}y} &=  \int te^{t^2} dt\\\\
e^{t^2}y &= \frac13 \int e^{u} du & \carray{u = t^2 \\ du = 2t}\\
e^{t^2}y & = \frac12 e^{u} + C \\&= \frac12 e^{t^2} +C & y(t) &= \underbrace{\frac12 e^t }_{y_p} + \underbrace{Ce^{-t^2}}_{y_h}
}
}

}

\subsection{Method of integrating factors}
\slide[Method of integrating factors: $y\p+p(t)y=h(t) $ ]{
\enum{
\item Multiply by the integrating factor $\mu(t)=e^{\intop p(t)\text{d}t  }$ \student{\algn{\mu(t)y\p +\underbrace{p(t)\mu(t)}_{\mu\p(t)}y &= h(t)\mu(t) \\
\dd{}{t}(\mu \cdot y) &= h(t)\mu(t)\\
\text{d}(\mu \cdot y) &= h(t)\mu(t)\text{d}t
}}
\item Integrate: \student{\[\mu \cdot y = \intop h(t)\mu(t)\text{d}t + C \]}

\item Isolate $y(t)$: \student{\[ y(t) = \frac{\intop h(t)\mu(t)\text{d}t + C}{\mu(t)} \]}}
}

\settoggle{covered}{false}

\slide{\ex{Find the general solution of $y\p-3y=e^t$.} \hfill $\boxed {\rarray{ y(t) = \frac{\intop h(t)\mu(t)\text{d}t + C}{\mu(t)} \\ \mu(t)=e^{\intop p(t)\text{d}t }}}$

\student{\algn{\mu(t)&=e^{\intop(-3)dt}=e^{-3t}\\
e^{-3t}y\p-3e^{-3t}y&=e^t e^{-3t}\\
\dd{}{t}\left( e^{-3t}y \right ) &=e^{-2t}\\
\text{d}\left( e^{-3t}y \right ) &=e^{-2t}\text{d}t\\
e^{-3t}y(t)  &= \intop e^{-2t}\text{d}t+C\\
y(t) &= \frac{ -\frac12 e^{-2t}+C}{e^{-3t}}\\
&=\underbrace{-\frac12e^t}_{y_p}+\underbrace{Ce^{3t}}_{y_h}
}
}

}

\slide{\ex{Find the general solution of $y\p-3t^2y=3t^2$.}  \hfill $\boxed {\rarray{ y(t) = \frac{\intop h(t)\mu(t)\text{d}t + C}{\mu(t)} \\ \mu(t)=e^{\intop p(t)\text{d}t }}}$
\vfill\student{\algn{\mu(t)&=e^{-\intop3t^2dt}=e^{-t^3} &                                                                                    & \carray{\text{let }u=t^3\\du=3t^2}\\
e^{-t^3}y(t)  &= \intop3t^2e^{-t^3}\text{d}t&                                                          \intop 3t^2e^{-t^3}\text{d}t & = \intop e^{-u} du\\
                       &                                                                                                                                                             &   & = -e^{-u}+C\\
e^{-t^3}y(t) &= -e^{-t^3}+C\\
\Aboxed{y(t)&=-1+Ce^{t^3}}
}
}
}


\slide[]{\ex{$ty'+5y=24t^3$, with $y(1)=2$}
\student{\vfill
\algn{y'+\frac{5y}{t} &=24t^2&\mu(t)&=e^{\intop \frac{5}{t} dt}=e^{5 \ln(t) }\\&&&=\left(e^{\ln(t)}\right)^5=t^5\\
t^5y(t)&=\intop 24t^7 dt = 3t^8+C\\\\
y(t)&=3t^3+Ct^{-5}\\
y(1)&=2=3+C &C=-1
}\vfill \[\boxed{y(t)=3t^3-t^{-5}}\]
}
}

\slide[Application: Mixing Tank 
\hspace{2em}  ]{
At time $t=0$ a tank contains 10 kg of salt dissolved in 200 liters of water. Water containing $\frac{1}{2}$kg of salt per liter enters the tank at a rate of 2 L/min and the well mixed solution leaves the tank at the same rate.\vfill

\enum{\item[(a)] Let $Q(t)$ be the amount of salt in kilograms in the solution after $t$ minutes have passed. Find a formula for the rate of change in the amount of salt, $\frac{dQ}{dt}$, in terms of the amount of salt in the solution, $Q(t)$. \vfill   
\twomini[.3]{.25}{.75}{
 \includegraphics[height=2.8cm]{images/mixing_tank.pdf}
}{
\student{\algn{\dd{Q}{t} &= \text{in} - \text{out} =0.5 \frac{kg}{L}\times 2\frac{L}{min} - \frac{Q\;kg}{200 L} \times 2\frac{L}{min} \\&=1-\frac{Q}{100} }}}
 \vfill
  \item[(b)] Find $Q(t)$. \hfill\student{  $\carray{\mu(t)=e^{\nicefrac{t}{100}}\\  Q(0)=10 } \quad \Rightarrow \quad Q(t)=100-90 e^{-\nicefrac{t}{100}}$}  }

}

\subsection{Summary}
\slide[Summary]{
\itmz{
\item General linear 1st order ODE: $y\p +p(t)y=h(t)$\vfill
\itmz{
\item To solve,  turn the LHS into an total derivative: $$y\p +p(t)y \quad \rightarrow \quad \text{d}(\mu\cdot y ) = \mu y\p  + \mu\p y$$
\subitem{ Multiply by an integrating factor $\mu(t)=e^{\intop p(t)\text{d}t} $
\item General solution: $ y(t) = \frac{\intop g(t)\mu(t)\text{d}t + C}{\mu(t)}$}
}\vfill
\item This method solves \uline{all} linear 1st order ODEs, provided that $p(t)$ and $h(t)$ are continuous.
}
}

\end{document}

