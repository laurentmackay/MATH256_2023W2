\input{notes.tex}


\iftoggle{dualscreen}{\setbeameroption{show notes on second screen=right}}{}

\usepackage{empheq}


\begin{document}

\section{MATH 256}
\subsection{Introduction}
\slide[Land Acknowledgement]{\vfill
UBC's Point Grey Campus is located on the traditional, ancestral, and unceded territory of the Musqueam. The land it is situated on has always been a place of learning for the Musqueam people, who for millennia have passed on culture, history, and traditions from one generation to the next. I encourage you to learn more at UBC's Indigenous Portal, \url{http://indigenous.ubc.ca}, and on the Musqueam's website, \url{http://musqueam.bc.ca}.\vfill
}

\slide[Who am I?]{
\vfill
\begin{itemize}
\item \textbf{My name:} Laurent MacKay (he/him/his)
\item \textbf{Position:} Postdoctoral Research Fellow \subitem{\textbf{My research:} Mathematical Biology (applied differential equations)}
\item \textbf{Office (student) hours:} TBD: \url{https://www.when2meet.com/?22890274-2QF6r}
\item \textbf{Contact:} Canvas messaging system or email: \url{mailto:lmackay@math.ubc.ca}
\end{itemize}
\vfill
}
\slide[Who are you?]{
\itmz{ \item Primarily Engineering Undergrads$\dots$\vfill
\subitem{Computer? \item Electrical? \item Chemical and Biological? \item Civil? \item Environmental? \item Other?
} \vfill}
}
\subsection{Syllabus Overview}
\slide[What will we learn?]{
\vfill
Analytical methods for solving Differential Equations (DEs) and some example applications (mass-spring systems, circuits, heating/cooling, etc.)\vfill

\begin{itemize}
\item Ordinary Differential Equations (ODEs) - one independent variable\vfill
\subitem{Fancy Integration (separable equations, method of integrating factors) \item Educated Guessing (ansatz method, method of undetermined coeffs.) \item Laplace Transforms \item Fourier Series (periodic solutions) }
\vfill
\item Partial Differential Equations (PDEs) -  multiple independent variables\vfill
\subitem{Heat/Diffusion Equation, Wave Equation, Laplace Equation}
\end{itemize}\vfill
}

\slide[Course Structure]{
\textbf{Every week:}
\begin{itemize}
\item Attend class!
\begin{itemize}
\item I'll do some exercises, you do others with neighbours (bring something to write with)
\item Asking questions is encouraged
\item Skeleton and annotated lecture notes are posted on Canvas
\item Lecture recordings are also posted within 24 hours
\end{itemize}\vfill
\item Attend office hours as needed (will be near the math dept.)
\subitem{Scheduling Poll: \url{https://www.when2meet.com/?22890274-2QF6r} }
\vfill
\item Check \textbf{Canvas} for important announcements and \textbf{Piazza} for student-led discussion (WebWork, Assignments, etc.)

\end{itemize}

}

\slide[MATH 256: Grade Breakdown]{
\itmz{ \item 10\% WebWork - 7 Total \subitem{First one due Jan. 18$^{\text{th}}$}  \vfill \item 10\% 2 Written HW Assignments \subitem{Due dates: TBD \item Submitted via: \textbf{Gradescope}} \vfill \item 30\% 2 Midterms \subitem{Weeks of Feb. 5$^{\text{th}}$ and  March 11$^{\text{th}}$ \item One page (front \& back) of notes is permitted.} \vfill \item 50\% Final Exam \subitem{Do not make travel plans in April...}}
}

\slide[Textbooks \& Resources]{
\itmz{
\item Diffy Qs: Free online text, quite good
\subitem{\url{https://www.jirka.org/diffyqs/html/diffyqs.html}}
\vfill
\item Paul's Online Math Notes 
\subitem{\url{https://tutorial.math.lamar.edu/classes/de/de.aspx} \item Excellent free resource, better solutions than Diffy Qs.}
\vfill
\item  Boyce and Diprima: Optional Hardcopy, also good
\subitem{\url{https://www.google.com/search?q=boyce+and+diprima+elementary+differential+equations+and+boundary+value+problems}}
}
}

\end{document}