\input{notes.tex}


\iftoggle{dualscreen}{\setbeameroption{show notes on second screen=right}}{}
\usetikzlibrary{arrows}

\begin{document}
\section{Lecture 16}
\subsection{Preamble}
\settoggle{covered}{true}

\slide[\begin{minipage}{0.7\textwidth}Intro to Linear Systems:  Skydiving\end{minipage}\begin{minipage}{0.3\textwidth}{\hspace{5.25em}\includegraphics[height=.75cm]{images/airplane.pdf}}\end{minipage}]{  \begin{minipage}{0.6\textwidth} \vspace{-2em} Newton's 2$^{\text{nd}}$ Law: \end{minipage} \begin{minipage}{0.2\textwidth}\centerline{ \includegraphics[width=2cm]{images/skydiver_solo.pdf}} \end{minipage} \hfill \vspace{-2em}\algn{ &ma = F(t)  = -mg - \mu v \intertext{noting that $a=x''$ and $v=x'$, we can rewrite this as}
 & \left. x'' + \frac{\mu}{m} x' = -g \quad  \right\} \text{2$^{nd}$ order ODE, one unknown function}}
\student{
or equivalently, \algn{  & \left. \begin{array}{c l}   x' &=v \\v' &= - \frac{\mu}{m} v-g\end{array}\right\} \text{1$^{st}$ order ODEs, two unknown functions}}
\vfill
Q: How do we find two unknown functions simultaneously?
}

}%end sldie



\slide[Linear Systems of DEs: Matrix Notation]{

\[ \text{$x\rightarrow x_1$, $v\rightarrow x_2$} \quad  \Rightarrow \quad \begin{array}{c l}  x_1' &=x_2\\x_2' &= - \frac{\mu}{m} x_2-g \end{array}\]
\student{
Using matrix notation:
\algn{ \dd{}{t}\mat{c}{x_1\\ x_2}  &= \mat{cc}{0&1\\0&-\frac{\mu}{m} }\mat{c}{x_1\\x_2} +\mat{c}{0\\g}\\\\
&=\mathbf{A} \vec{x} + \text{constant vector}
}
\vfill}

}

\slide[Linear Systems of DEs: Matrix Notation]{

General Linear System IVP:\[\dd{}{t}\vec{x} = \mathbf{A}(t) \vec{x} +\vec{f}(t) \quad\text{with} \quad \vec{x}(t_0)=\vec{x}_0\]
where $\mathbf{A}$ is an $n\times n$ matrix, and both $\vec{x}$ and $\vec{f}$ are $n\times1$ column vectors.
\vfill
\student{
Operator Form:
\algn{\dd{}{t}\vec{x} - \mathbf{A}(t) \vec{x}& = \vec{f}(t)\quad\text{with} \quad  \vec{x}(t_0)=\vec{x}_0\\
\op{\vec{x}} &= \vec{f}(t) }\vfill
\centerline{$\vec{f}(t)=\vec{0} \qquad \Rightarrow \qquad $homogeneous system}
}

}

\subsection{Converting between representations}
\slide[Equivalence of problems]{
For every $n^{th}$ order linear ODE, there is a corresponding system of $n$ 1$^{st}$ order linear ODEs.\vfill
\student{
\[a_n(t)x^{(n)}(t)+a_{n-1}(t)x^{(n-1)}(t) + \dots + a_0(t)x(t) = h(t)\]
can be expressed as 
\[\dd{}{t}\vec{x} = \mathbf{A}(t) \vec{x} +\vec{f}(t)\]
with \[\vec{x}(t)=\mat{c}{x\\x\p \\ \vdots \\ x^{(n-1)}}\]
}
}

\slide{
\ex{Rewrite $x'''+x'+x = t^2$ in the form $\dd{}{t}\vec{x} = \mathbf{A}\vec{x} + \vec{f}(t)$}\vfill
\student{\algn{\vec{x}&=\mat{c}{x_1\\x_2\\x_3} \quad \carray{x_1=x\\ x_2=x\p\\x_3=x\pp} \qquad\qquad \dd{}{t}\vec{x} =\mat{c}{x\p\\x\pp\\x'''} =  \mathbf{A}\vec{x} + \vec{f}(t)\\
x\p&=x_2\\
x\pp&=x_3\\
x'''&=-x -x\p+t^2\\
&=-x_1-x_2+t^2\\\\
 \dd{}{t}\vec{x}&=\mat{c}{x_1'\\x_2'\\x_3'} =\mat{c}{x\p\\x\pp\\x'''} = \underbrace{ \mat{ccc}{0&1&0\\0&0&1\\-1&-1&0}}_{\mathbf{A}} \mat{c}{x_1\\x_2\\x_3} \ + \underbrace{ \mat{c}{0\\0\\t^2}}_{\vec{f}}\\
}
}
}


\subsection{Finding solutions}

\slide[Skydiving: Homogeneous Solutions ]{
Suppose we want to solve $v\pp+\frac{\mu}{m}v\p=0$
\student{Guess $v(t)=e^{rt}$
\algn{r^2+\frac{\mu}{m}r &=0 &
r\left(r+\frac{\mu}{m}\right) &=0 \\ &r=0, -\frac{\mu}{m} 
&v(t) &= c_1+c_2e^{ -\frac{\mu}{m}t}\\
&&=c_1y_1(t) + c_2y_2(t)
}\vfill
What about for the vector expression?\vfill
\centerline{two LI homogeneous solutions $\quad \rightarrow \quad $ two LI vectors}\vfill
\algn{\vec{x}(t)  &=c_1 \mat{c}{1\\0} +c_2 e^{-\frac{\mu}{m} t} \mat{c}{1 \\ -\frac{\mu}{m}}\\\\
&=c_1\vec{x}_1(t) \;\;\;+\quad c_2\vec{x}_2(t) }
}
}




\slide[Homogeneous Problem and Superposition]{\vspace{-1em}
Suppose $\vec{x}_1(t), \;\vec{x}_2(t),\;\dots, \;\vec{x}_k(t)$ all solve the homogeneous problem \[\dd{}{t}\vec{x} = \mathbf{A}(t) \vec{x} \]\vfill
Then \[\vec{x}(t) = c_1\vec{x}_1(t)+c_2\vec{x}_2(t)+\dots+c_k\vec{x}_k(t)\]
also solves the same homogeneous problem.
\vfill
\student{
\algn{\dd{}{t}\vec{x} &= \sum_{i=1}^nc_i\dd{}{t}\vec{x}_i= \sum_{i=1}^nc_i \mathbf{A}(t) \vec{x}_i \\& =  \mathbf{A}(t)  \sum_{i=1}^nc_i\vec{x}_i \\
&=\mathbf{A}(t)\vec{x}(t)}
}
}
\settoggle{covered}{false}

\slide[Homogeneous Problem and Superposition]{\vspace{-1em}
Suppose $\vec{x}_1(t), \;\vec{x}_2(t),\;\dots, \;\vec{x}_k(t)$ all solve the homogeneous problem \[\dd{}{t}\vec{x} = \mathbf{A}(t) \vec{x} \]
\vspace{-2em}

Then the set of solutions \[ \left\{\vec{x}_1(t), \vec{x}_2(t), \dots,\vec{x}_k(t)\right\}\]
form a vector space $V$.
\vfill
\student{
Since we are using ``standard'' addition/multiplication we only need to check 2 properties\vfill
\enum{
\item Additive Closure\[\vec{x}_i+\vec{x}_j \in V \quad \checkmark \]
\item Scalar Closure:\\
For any constant c
\[c \vec{x}_j \in V \quad \checkmark \]
}
}
}

\slide[Homogeneous Problem and Superposition]{\vspace{-1em}
Suppose $\vec{x}_1(t), \;\vec{x}_2(t),\;\dots, \;\vec{x}_k(t)$ all solve the homogeneous problem \[\dd{}{t}\vec{x} = \mathbf{A}(t) \vec{x} \]
\vspace{-2em}

Then the set of solutions \[ \left\{\vec{x}_1(t), \vec{x}_2(t), \dots,\vec{x}_k(t)\right\}\]
form a vector space $V$.\vfill
\hrule
\vspace{.5em}
If $\mathbf{A}$ is $n\times n$, and we can find $n$ linearly independent solution vectors (i.e., $k=n$), then we have a solution basis. 
\vfill
That is, \uline{any} solution $\vec{x}(t)$ can be written as 
\student{\[\vec{x}(t) = c_1\vec{x}_1(t)+c_2\vec{x}_2(t)+\dots+c_n\vec{x}_n(t)\]}\small
See DiffyQs Appendixes A.4 \& A.5 for review of vector spaces and bases.
}


\slide[Finding the solution vectors $\vec{x}_i$]{
Given
\[\dd{}{t}\vec{x} = \mathbf{A} \vec{x},\]
we want to find one of its solution vectors.\vfill
\student{
For a scalar equation, if we have $y\p=ay$ then we know the solution is $y=ce^{at}$.
\vfill
Lets guess $\vec{x}(t)=e^{\lambda t}\vec{v}$, and plug it into the ODE
\algn{\lambda e^{\lambda t} \vec{v} &=  \mathbf{A} e^{\lambda t} \vec{v}\\
\lambda \vec{v} &= \mathbf{A} \vec{v}}
\vfill
$\vec{v}$ is an eigenvector of $\mathbf{A}$, and $\lambda$ is its associated eigenvalue.
}
}

\slide[Eigenvectors/Eigenvalues]{


\itmz{\item A $n \times n$ matrix has $n$ eigenvalues and eigenvectors \subitem{except in some special cases}
}\vfill
\student{
Ignoring those special cases for now, we can write any homogeneous solution as

\[\vec{x}(t) = c_1 e^{\lambda_1t}\vec{v}_1+c_2e^{\lambda_2 t}\vec{v}_2+\dots+c_ne^{\lambda_n t}\vec{v}_n\]
where $\vec{v}_i$ is the eigenvector associated with the eigenvalue $\lambda_i$.
\vfill

}\vspace{3em}
See DiffyQs \S 3.7 for details on those special cases.
}

\slide[Eigenvectors/Eigenvalues]{


\itmz{\item A $n \times n$ matrix has $n$ eigenvalues and eigenvectors \subitem{except in some special cases}
}\vfill
We want to find values $\lambda$ such that for some non-zero $\vec{v}$\[\mathbf{A}\vec{v} = \lambda \vec{v}\]
\student{
Rearrange as
\[(\mathbf{A} - \lambda \mathbf{I}) \vec{v} = \vec{0}\]
If $(\mathbf{A} - \lambda \mathbf{I})^{-1}$ exists, then \[\vec{v} =(\mathbf{A} - \lambda \mathbf{I})^{-1} \vec{0} = \vec{0} \]
So we need 
 \[\det(\mathbf{A} - \lambda \mathbf{I}) = 0\]

}\vspace{3em}
See DiffyQs \S A.6 for a review of determinants
}


\slide[Eigenvectors/Eigenvalues]{
\itmz{\item A $n \times n$ matrix has $n$ eigenvalues and eigenvectors \subitem{except in some special cases}
\item The eigenvalues are obtained from the roots of the characteristic polynomial obtained from   \[\det(\mathbf{A} - \lambda \mathbf{I}) = 0,\] where "det" is short for "determinant" and $\mathbf{I}$ is the $n \times n$  identity matrix
\item  The eigenvectors are computed by solving the linear system  \[(\mathbf{A} - \lambda \mathbf{I}) v= 0\] once each of the eigenvalues $\lambda$ is found.
\item the eigenvectors are not unique, defined up to arbitrary mulitplicative constant
}
}
\settoggle{covered}{false}
\slide[Find the eigenvalues/vectors associated with \hfill \small $\larray{ \dd{x}{t} = -3x - 2y \\ \dd{y}{t} = -2x - 6y}$]{\vspace{-1.5em}
\student{\algn{\dd{}{t} \mat{c}{x\\y}  &= \mat{cc}{-3&-2\\-2&-6} \mat{c}{x\\y} \\ \det\left(  \mat{cc}{-3-\lambda&-2\\-2&-6-\lambda} \right) &= 0\\
(3+\lambda)(6+\lambda) - 4 &=0\\
\lambda^2 +9\lambda + 18 - 4 &= 0\\
 \lambda^2 +9\lambda + 14 &= 0\\
\lambda &= \frac{-9 \pm \sqrt{81 - 4\cdot 14}}{2}=\frac{-9 \pm \sqrt{81 - 56}}{2}\\
&=\frac{-9 \pm \sqrt{25}}{2}=\frac{-9\pm5}{2} \\
\lambda_{1,2} &= -2, -7}
}
}
\slide[Find the eigenvalues/vectors associated with \hfill \small $\larray{ \dd{x}{t} = -3x - 2y \\ \dd{y}{t} = -2x - 6y}$]{\student{\vspace{-2em}
\algn{\uline{\lambda_1=-2:}\quad \mathbf{A} \vec{v}_1 &= -2 \vec{v}_1 & (\mathbf{A}+2\mathbf{I})\vec{v}_1 & = \vec{0}\\\\
 \mat{cc}{-1&-2\\-2&-4} \vec{v}_1 &= \vec{0} \qquad  & \uline{\text{Augmented matrix:}}\quad &  \mat{cc|c}{-1&-2 &0\\-2&-4&0}
\intertext{row 2 and row 1 are linearly dependent: $R_2-2R_1\rightarrow R_2$}
& \mat{cc|c}{-1&-2 &0\\0&0&0} &-1x-2y&=0\\
&&x&=-2y\\
\vec{v}_1&=\mat{c}{-2\\1}&\vec{x}_1(t)&=\mat{c}{-2\\1}e^{-2t}
}
}
}

\slide[Find the eigenvalues/vectors associated with \hfill \small $\larray{ \dd{x}{t} = -3x - 2y \\ \dd{y}{t} = -2x - 6y}$]{\student{\vspace{-2em}
\algn{\uline{\lambda_2=-7:}\quad \mathbf{A} \vec{v}_2 &= -7\vec{v}_2 & (\mathbf{A}+7\mathbf{I})\vec{v}_2 & = \vec{0}\\\\
 \mat{cc}{4&-2\\-2&1} \vec{v}_2 &= \vec{0} \qquad  & \uline{\text{Augmented matrix:}}\quad &  \mat{cc|c}{4&-2 &0\\1&-4&0}
\intertext{row 2 and row 1 are linearly dependent: $R_2+2R_1\rightarrow R_2$}
& \mat{cc|c}{4&-2 &0\\0&0&0} &4x-2y&=0\\
&&x&=y/2\\
\vec{v}_2&=\mat{c}{1\\2}&\vec{x}_2(t)&=\mat{c}{1\\2}e^{-7t}
}
}
}



\slide[Summary]{
\itmz{
\item Linear $n^{th}$ order DEs can be converted to a system of $n$ 1$^{st}$ order DEs\vfill
\item Homogeneous system:  $\dd{}{t}\vec{x} = \mathbf{A}(t) x$
\subitem{Need to find $n$ linearly independent solutions $\left\{ \vec{x}_1(t), \;\vec{x}_2(t),\;\dots, \;\vec{x}_n(t) \right\} $
\item These solution vectors are based on the eigenvectors/eigenvalues of $\mathbf{A}$ \item$\vec{x}_i(t) = e^{\lambda_i t}\vec{v}_i$}
\vfill
\item Finding eigenvalues/eignevectors, we need to solve    \[\det(\mathbf{A} - \lambda \mathbf{I}) = 0\quad\text{and}\quad (\mathbf{A} - \lambda \mathbf{I}) v= 0\]
}
}



\end{document}

