\input{notes.tex}
\iftoggle{dualscreen}{\setbeameroption{show notes on second screen=right}}{}


\begin{document}
\section{Lecture 21}
\subsection{Preamble}
\settoggle{covered}{true}
\slide[Recall: Linear IVPs]{\vspace{-0.5em}

Given the initial state of a system (e.g., mass-spring or electrical circuit),  predict its behaviour at later times.
\vfill
\enum{
\item Scalar ODEs: \vfill
\centerline{$\op{y(t)}=f(t)\quad $ with $\quad y(0)=y_0$ (and maybe $y'(0)=v_0$)}\vfill
\subitem{1$^{\rm st}$ order \student{$\Rightarrow$ solve by integrating factor}
\item 2$^{\rm nd}$ order \student{$\Rightarrow$ solve by undetermined coefficients} \subitem{\student{Equivalent to a system of 2 1$^{\rm st}$ order ODEs} }}
\vfill
\item Systems of 1$^{\rm st}$ order ODEs:
\[\dd{}{t} \vec{x} =\mathbf{A}(t)\vec{x}+\vec{f}(t) \quad\text{with}\quad\vec{x}(0)=\vec{x}_0\]
\vfill
General solution: $\vec{x}(t)=\mathbf{X}(t) \vec{c}+\mathbf{X}\int \mathbf{X}^{-1}\vec{f}(t)dt$ \hfill with\hfill $\vec{c}=\mathbf{X}^{-1}(0)\vec{x}_0$
}

}

\slide[Fundamental Matrices and Solution Bases]{\vspace{-1em}
\[\dd{}{t} \vec{x} =\mathbf{A}(t)\vec{x}+\vec{f}(t) \quad\text{with}\quad\vec{x}(0)=\vec{x}_0\]
\vfill
General solution: $\vec{x}(t)=\mathbf{X}(t) \vec{c}+\mathbf{X}\int \mathbf{X}^{-1}\vec{f}(t)dt$\hfill with\hfill $\vec{c}=\mathbf{X}^{-1}(0)\vec{x}_0$

\vfill

The $n$ columns of the fundamental matrix $\mathbf{X}$ are the L.I. solutions to \[\dd{}{t} \vec{x}_i =\mathbf{A}(t)\vec{x}_i \quad \text{for }i=1,\dots,n\]

\vfill
These \alert{fundamental solutions} form a basis for all homogeneous solutions.
\student{
\[\vec{x}_h = c_1\vec{x}_1+\dots+c_n\vec{x}_n \quad \text{with } c_i=\frac{ \vec{x}_i(0)\cdot\vec{x}_0}{||\vec{x}_i(0)||}\]

\vfill
Other solution bases can be found through linear combinations...
}
}

\slide[The Flow Matrix]{\vspace{-1em}
\[\dd{}{t} \vec{x} =\mathbf{A}(t)\vec{x} \quad\text{with}\quad\vec{x}(0)=\vec{x}_0\]
\vfill
$\vec{x}(t)=\mathbf{X}(t)\vec{c}$\quad with\quad$\vec{c}=\mathbf{X}^{-1}(0)\vec{x}_0\quad \Rightarrow\quad \vec{x}(t)=\mathbf{X}(t)\mathbf{X}^{-1}(0)\vec{x}_0$.

\vfill
\student{
Let $\mathbf{\Psi}(t)=\mathbf{X}(t)\mathbf{X}^{-1}(0)$ be a fundamental matrix in a new basis set \[\{\vec{\psi}_1(t),\dots,\vec{\psi}_n(t)\}\]

\vfill

Observe that at $t=0$ we have 
\[\mathbf{\Psi}(0)=\mathbf{X}(0)\mathbf{X}^{-1}(0) = \mathbf{I}\]
\vfill
So, a homogeneous solution with an initial condition $\vec{x}_h(0)=\vec{x}_0$ can be written as
\[\vec{x}_h(t)=\mathbf{\Psi}(t)\vec{x}_0\]

}
}
\slide[Orthonormal Solution Bases]{
\vspace{-0.75em}
Let $\mathbf{\Psi}(t)=\mathbf{X}(t)\mathbf{X}^{-1}(0)$ be a fundamental matrix in a new basis set \[\{\vec{\psi}_1(t),\dots,\vec{\psi}_n(t)\} \qquad \text{with } \quad  \mathbf{\Psi}(0)=\mathbf{I}\]\vspace{-0.75em}
\student{
\twomini[.3]{.5}{.5}{
Note:  $\vec{\psi}_i(0)=\underbrace{[0,\dots,1,\dots,0]^T}_{  i^{th} \text{ component is non-zero}}$
}{
$\Rightarrow \{\vec{\psi}_i(0)\}$ is an orthonormal basis, i.e., \[\vec{\psi}_i(0)\cdot \vec{\psi}_j(0)=\begin{cases} 1&i=j\\0&\text{otherwise}\end{cases}.\]
}


This means that  if we write \[\vec{x}(t) = c_1\vec{\psi}_1(t)+\dots+c_n\vec{\psi}_n(t) \qquad \text{with }\qquad \vec{x}(0)=\vec{x}_0\]then $c_i=\vec{\psi}_i(0)\cdot\vec{x_0} =i^{\rm th}$ component of $\vec{x}_0$.
\vfill
Q:  How can we do this when $n\to\infty$?\vfill
\centerline{\ex{PDEs: $\op{u(x,t)}=0$ with $u(x,0)=u_0(x)$}}

}
}

\subsection{BVPS}
\slide[BVPs]{

Before we solve PDEs, we need to discuss ODE boundary value problems.
\vfill
Suppose you have data about an ODE solution at  $t=0$ and $t=T$.

\[ ay\pp +by\p+cy=f(t), \qquad \student{ \underbrace{ \larray{y(0)=y_0\\y(T)=y_T} \quad\text{\uline{or}} \quad \larray{y\p(0)=v_0\\y\p(T)=v_T}}_{\text{boundary conditions (BCs)}}}\]
\vfill
\student{\centerline{We call these boundary value problems (BVPs).}}

}

\slide[Periodic BVPs]{
In cases where $f(t)$ is a \uline{periodic function with period $T$}, i.e., \student{\[f(t+T)=f(t) \qquad \forall t,\]} the solution $y(t)$ eventually becomes $T$-periodic.
\vfill
We can find the long-term periodic solution by solving an ODE with \uline{periodic boundary conditions}, given by \vfill
\student{\[ ay\pp +by\p+cy=f(t), \qquad \underbrace{ \larray{y(0)=y(T)\\y\p(0)=y\p(T)}}_{2^{nd} \text{ order} \Rightarrow 2 \text{ BCs}} \]}
\vfill
}
\subsection{Periodic BVPs}
\slide{\ex{$y\pp + y=\cos(2\pi t), \qquad \larray{y(0)=y(1)\\y\p(0)=y\p(1)}$    }
\student{\algn{y(t)&=y_h + y_p\qquad
\arr{rl}{y_h&=c_1\cos(t) + c_2 \sin(t)\\
\text{geuss: } y_p&=A\cos(2\pi t) + B\sin(2\pi t)}\\
\text{\uline{M. U. C.:} }&A=\frac{1}{1-4\pi^2}, \; B=0\\
y(t)&=c_1\cos(t) + c_2 \sin(t) + \frac{1}{1-4\pi^2}\cos(2\pi t)\intertext{Boundary Conditions:}
y(0) = c_1 + \cancel{\frac{1}{1-4\pi^2}}  &= y(1) = c_1 \cos(1) + c_2 \sin(1) +\cancel{ \frac{1}{1-4\pi^2}}\\
 c_1  &= c_1 \cos(1) + c_2 \sin(1) \\\\
y\p(0) = c_2 &= y\p(1)= -c_1\sin(1) + c_2\cos(1)\\
 c_2 &=  -c_1\sin(1) + c_2\cos(1)
}
}

}

\slide{
$\begin{array}{ll}  c_1  &= c_1 \cos(1) + c_2 \sin(1) \\  c_2  &=  -c_1\sin(1) + c_2\cos(1) \end{array}$\vspace{-3em}
\student{
\algn{c_2(1-\cos(1)) & =- c_1\sin(1)& c_1&=c_1\cos(1) -\frac{\overbrace{\sin^2(1)}^{1-\cos^2(1)}}{1-\cos(1)}c_1\\
c_2=-&\frac{\sin(1)}{1-\cos(1)}c_1&c_1 &= c_1\paren{\cos(1) -\frac{(1+\cos(1))\cancel{(1-\cos(1)})}{\cancel{1-\cos(1)}}} \\
&&c_1&=c_1\paren{\cancel{\cos(1)}-\paren{1+\cancel{\cos(1)}}}\\
&&c_1&=-c_1 \qquad \Rightarrow c_1=c_2=0\\
&&\Aboxed{y(t) &= \frac{1}{1-4\pi^2}\sin(2\pi t)}
 }\vfill
 \uline{Alternative approach:}\vfill
Notice that  $y_h(t)=y_h(t+2\pi)$ and $y_p(t)=y_p(t+1)$. \vfill Since the BCs require solutions with period 1, we know the homogeneous part of the solution is zero.
}
}


\slide{
\ex{$y\pp + y=f(t), \qquad \larray{y(0)=y(1)\\y\p(0)=y\p(1)}$ \hfill with $f(t+1)=f(t)$ }
\vfill

\student{ Due to its periodicity, we can express $f(t)$ as \[f(t) = \frac{1}{2} a_0 + \sum_{n=1}^\infty a_n \cos(2n\pi t) + b_n \sin(2n\pi t)\]
This is called the \alert{Fourier Series} of the function $f(t)$, the coefficients $a_n$ and $b_n$ are called \alert{Fourier coefficients}.
\vfill
The coefficients are obtained by taking the \uline{inner product} of the function $f(t)$ and the Fourier basis\[\left\{ \cos( 2n\pi t),\; \sin(2n \pi t) \right\} \qquad n=0,\dots,\infty\]
\vfill

}

}

\settoggle{covered}{false}
\slide{
$y\pp + y =  \frac{1}{2} a_0 + \sum_{n=1}^\infty a_n \cos(2n\pi t) + b_n \sin(2n\pi t),  \qquad \larray{y(0)=y(1)\\y\p(0)=y\p(1)}$. 
\[\text{\uline{Guess:} } y(t)=  \sum_{n} y_n(t) = A_0 +\sum_{n=1}^\infty A_n \cos(2n\pi t) + B_n \sin(2n \pi t) \]

Apply M.U.C. term-by-term for the different values of $n$. \vfill
\student{
For $n\neq0$, the $n^{th}$ particular solution is \algn{y_n &=  A_n \cos(2n\pi t) + B_n \sin(2n\pi t) \\ y_n\pp &= -4n^2\pi^2A_n \cos(n\pi t) -4n^2\pi^2 B_n \sin(n\pi t)}
\vspace{-3em}
\algn{\text{\uline{ODE}: }\quad y_n\pp+y_n &=a_n \cos(2n\pi t) + b_n \sin(2n\pi t)\\\\
A_n(1-4n^2\pi^2)  \cos(n\pi t) &+ B_n(1-4n^2\pi^2)  \sin(n\pi t)&A_n=\frac{a_n}{1-4n\pi^2} \\
&=  a_n \cos(n\pi t) + b_n \sin(n\pi t)& B_n=\frac{b_n}{1-4n\pi^2}  }
}

}

\slide{$y\pp + y =  \frac{1}{2} a_0 + \sum_{n=1}^\infty a_n \cos(2n\pi t) + b_n \sin(2n\pi t),  \qquad \larray{y(0)=y(1)\\y\p(0)=y\p(1)}$. 
\[\text{\uline{Guess:} } y(t)=  \sum_{n} y_n(t) = A_0 +\sum_{n=1}^\infty A_n \cos(2n\pi t) + B_n \sin(2n \pi t) \]

Apply M.U.C. term-by-term for the different values of $n$.\vfill
\student{For $n=0$, we have
\algn{y_0&=A, y_0\pp=0\\
A&=\frac12 a_0\\
y(t) &= \frac12 a_0 + \sum_{n=1}^\infty \frac{a_n}{1-4n\pi^2}  \cos(2n\pi t)  + \frac{b_n}{1-4n\pi^2} \sin(2n\pi t) }\vfill
Given a specific periodic function $f(t)$, we can find its Fourier coefficients $a_n$ and $b_n$ and use the BVP solution above.
}
}

\slide[Inner Products]{
Dot products are an example of an \uline{inner product} for Euclidean vector spaces.

\[\left< \vec{x},\vec{y}\right> = \sum _i x_i y_i\]

5 basic properties define an inner product: \href{https://en.wikipedia.org/wiki/Inner_product_space}{\emph{wikipedia}}
\vfill
\student{
To define inner products for function spaces, sums are replaced by integrals.
\vfill

For $T$-periodic functions $f$ and $g$ we define the following inner product:
 \[\left< f,g\right> = \frac{2}{T}\int_0^T f(t)g(t)dt\]
}

}
\subsection{Fourier Series and Basis}
\slide[Fourier Series]{\vspace{-1em}
Given any  periodic function $f(t)$ with period $T$, we can approximate $f(t)$ as a Fourier series
\[ f(t) \approx \frac{1}{2} a_0 + \sum_{n=1}^\infty a_n \cos\paren{\frac{2n\pi t}{T}} + b_n \sin\paren{\frac{2n\pi t}{T}} \]with
\algn{a_0&=\left< f(t), 1 \right> &&\student{=\frac2T\int_0^T f(t) dt}\\
a_n&=\left< f(t),  \cos\paren{\frac{2n\pi t}{T}} \right> && \student{=\frac2T\int_0^T f(t) \cos\paren{\frac{2n\pi t}{T}} dt}\\
b_n&=\left< f(t),  \sin\paren{\frac{2n\pi t}{T}} \right> && \student{=\frac2T\int_0^T f(t) \sin\paren{\frac{2n\pi t}{T}} dt}}
If $f(t)$ is a continuous function, then the approximation becomes an equality.
}



\slide[The Fourier Basis is Orthonormalized]{
\vfill
Consider $m$ and $n$ to be any two positive integers or zero, then we have\vfill
\algn{
\left< \cos\paren{\nicefrac{2n\pi t}{T}} ,  \sin\paren{\nicefrac{2m\pi t}{T}} \right>  &= 0 \qquad \forall m,n  \qquad \student{\text{(Orthogonality)}}\\\\
\left< \sin\paren{\nicefrac{2n\pi t}{T}} ,  \sin\paren{\nicefrac{2m\pi t}{T}} \right>  &= \left< \cos\paren{\nicefrac{2n\pi t}{T}} ,  \cos\paren{\nicefrac{2m\pi t}{T}} \right>\\
&=\begin{cases} 1  & \text{if } m=n\neq0   \qquad \student{\text{(Normalization)}} \\ 0  &\text{otherwise}\phantom{\neq0} \qquad \student{\text{(Orthogonality)}} \end{cases} }

\vfill
\student{The normalization condition is the reason for the factors of $\frac2T$ in front of the Fourier coefficient integrals.}
}

\end{document}