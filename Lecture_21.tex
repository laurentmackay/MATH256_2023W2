\documentclass[11pt, dvipsnames, handout]{beamer}
\newtoggle{full}
\settoggle{full}{true}

\newtoggle{covered}
\settoggle{covered}{false}

\newtoggle{presentable}
\settoggle{presentable}{false}

\newtoggle{dualscreen}
\settoggle{dualscreen}{false}

\usepackage{pgfplots}
%\pgfplotsset{compat = newest}

\usepackage{pgfpages}

\setbeamertemplate{note page}{\pagecolor{yellow!5}\vfill \insertnote \vfill}
\usepackage{collect}
\definecollection{notes}
\newcounter{notestaken}

\usepackage{xpatch}

\usepackage{ulem}

\usepackage[framemethod=tikz]{mdframed}

\usepackage{scalerel}
\usepackage{calc}

%\usepackage{enumitem}
\setlength\fboxsep{.2em}

\usepackage{graphicx} % Allows including images
\usepackage{booktabs} % Allows the use of \toprule, \midrule and \bottomrule in tables

\xpatchcmd{\itemize}
  {\def\makelabel}
  {\setlength{\itemsep}{0.65 em}\def\makelabel}
  {}
  {}


\xpatchcmd{\beamer@enum@}
  {\def\makelabel}
  {\setlength{\itemsep}{0.65 em}\def\makelabel}
  {}
  {}


%\makeatletter
%\renewcommand{\itemize}[1][]{%
%  \beamer@ifempty{#1}{}{\def\beamer@defaultospec{#1}}%
%  \ifnum \@itemdepth >2\relax\@toodeep\else
%    \advance\@itemdepth\@ne
%    \beamer@computepref\@itemdepth% sets \beameritemnestingprefix
%    \usebeamerfont{itemize/enumerate \beameritemnestingprefix body}%
%    \usebeamercolor[fg]{itemize/enumerate \beameritemnestingprefix body}%
%    \usebeamertemplate{itemize/enumerate \beameritemnestingprefix body begin}%
%    \list
%      {\usebeamertemplate{itemize \beameritemnestingprefix item}}
%      {%
%        \setlength\topsep{1em}%NEW
%        \setlength\partopsep{1em}%NEW
%        \setlength\itemsep{1em}%NEW
%        \def\makelabel##1{%
%          {%
%            \hss\llap{{%
%                \usebeamerfont*{itemize \beameritemnestingprefix item}%
%                \usebeamercolor[fg]{itemize \beameritemnestingprefix item}##1}}%
%          }%
%        }%
%      }
%  \fi%
%  \beamer@cramped%
%  \raggedright%
%  \beamer@firstlineitemizeunskip%
%}
%
%
%
%
%
%\makeatother

%\setlist[beamer@enum@]{topsep=1 em}
%\let\origcheckmark\checkmark %screw you dingbat
%\let\checkmark\undefined %screw you dingbat
%\usepackage{dingbat} 
%\let\checkmark\origcheckmark %screw you dingbat






%\usepackage{fontawesome}

\usepackage{mathtools}
\usepackage{etoolbox, calculator}

\usepackage{xcolor}
\usepackage{tikz}
\usetikzlibrary{arrows.meta}
\usetikzlibrary{calc}
\usepackage[nomessages]{fp}
\usepackage{transparent}
\usepackage{accsupp}
%\usepackage{color, xcolor}

%colorblind-friendly palette
%\definecolor{dblue}{RGB}{51,34,136}
\definecolor{lblue}{RGB}{136,204,238}
%\definecolor{green}{RGB}{17,119,51}
\definecolor{tan}{RGB}{221,204,119}
%\definecolor{mauve}{RGB}{204,102,119}

\usepackage{tcolorbox}



\usepackage{xifthen}
\usepackage{nicefrac}
\usepackage{amsmath}
\usepackage{amsthm}
\usepackage{amssymb}
\theoremstyle{definition}
\newtheorem*{define}{Definition}
\newtheorem*{recall}{Recall}


\DeclareMathOperator{\tr}{tr}

\usepackage{multicol}
%\setlength{\columnsep}{1cm}

\usepackage{tablists, amsmath,vwcol, cancel, polynom}
\usetikzlibrary{shapes, patterns, decorations.shapes}
%\usepackage{tikzpeople}
\tikzstyle{vertex}=[shape=circle, minimum size=2mm, inner sep=0, fill]
\tikzstyle{opendot}=[shape=circle, minimum size=2mm, inner sep=0, fill=white, draw]

% common math quick commands
\newcommand{\nicedd}[2]{\nicefrac{\text{d}#1}{\text{d}#2}}
\newcommand{\dd}[2]{\dfrac{\text{d}#1}{\text{d}#2}}
\newcommand{\pd}[2]{\dfrac{\partial #1}{\partial#2}}
\renewcommand{\d}[1]{\text{d}#1}
\newcommand{\ddn}[3]{\dfrac{\text{d}^{#3}#1}{\text{d}#2^{#3}}}
\newcommand{\pdn}[3]{\dfrac{\partial^{#3}#1}{\partial#2^{#3}}}
\newcommand{\p}[0]{^{\prime}}
\newcommand{\pp}[0]{^{\prime\prime}}
\newcommand{\op}[2][\text{L}]{#1 \left[ #2 \right]}

\newcommand{\lap}[1]{\mathcal{L}\left\{#1\right\}}
\newcommand{\lapinv}[1]{\mathcal{L}^{-1}\left\{#1\right\}}
\newcommand{\lapint}[1]{\int_0^\infty e^{-st}#1dt}
\newcommand{\evalat}[2]{\Big|_{#1}^{#2}}

\newcommand{\paren}[1]{ \left( #1 \right)}

\newcommand{\haxis}[4][\normcolor]{\draw[#1, <->] (-#2,0)--(#3,0) node[right]{$#4$}; }

\newcommand{\circled}[1]{\raisebox{.5pt}{\textcircled{\raisebox{-.9pt} {#1}}}}
\newcommand{\axis}[4]{\draw[\normcolor, <->] (-#1,0)--(#2,0) 
node[right]{$x$};
\draw[help lines, <->] (0,-#3)--(0,#4) node[above]{$y$};}

\newcommand{\laxis}[6]{\draw[<->] (-#1,0)--(#2,0) 
node[right]{$#5$};
\draw[ <->] (0,-#3)--(0,#4) node[above]{$#6$};}
\newcommand{\xcoord}[2]{
	\draw (#1,.2)--(#1,-.2) node[below]{$#2$};}
\newcommand{\textnode}[3]{
	\draw (#1,#2) node[below]{$#3$};}
	
\newcommand{\nxcoord}[2]{
	\draw (#1,-.2)--(#1,.2) node[above]{$#2$};}
\newcommand{\ycoord}[2]{
	\draw (.2,#1)--(-.2,#1) node[left]{$#2$};}
\newcommand{\nycoord}[2]{
	\draw (-.2,#1)--(.2,#1) node[right]{$#2$};}
\newcommand{\dlim}{\displaystyle\lim}
\newcommand{\dlimx}[1]{\displaystyle\lim_{x \rightarrow #1}}
\newcommand{\stickfig}[2]{
	\draw (#1,#2) arc(-90:270:2mm);
	\draw (#1,#2)--(#1,#2-.5) (#1-.25,#2-.75)--(#1,#2-.5)--(#1+.25,#2-.75) (#1-.2,#2-.2)--(#1+.2,#2-.2);}	

%\newcounter{example}
%\setcounter{example}{1}
%\newcounter{preFrameExample}
%\AtBeginEnvironment{frame}{\setcounter{preFrameExample}{\value{example}}}
%\newcommand{\ex}[1]{
%	 \setcounter{example}{\value{preFrameExample}}
%	 \textcolor{green}{\small\fbox{Example \arabic{example}: #1}}\\[8pt]
%	\stepcounter{example}}
%\newcommand{\exans}[1]{
%	\SUBTRACT{\value{preFrameExample}}{1}{\n}
%	 \textcolor{green}{\small\fbox{Solution \n: #1}}\\[8pt]}
\mode<presentation> {

% The Beamer class comes with a number of default slide themes
% which change the colors and layouts of slides. Below this is a list
% of all the themes, uncomment each in turn to see what they look like.


\usetheme{CambridgeUS}
\usecolortheme[named=black]{structure}


\newcommand{\studentcolor}[0]{ForestGreen}
\newcommand{\normcolor}[0]{NavyBlue}
\newcommand{\alertcolor}{Red}

\setbeamercolor{normal text}{fg=\normcolor}
\setbeamercolor{frametitle}{fg=\normcolor}
\setbeamercolor{section in head/foot}{fg=Black, bg=Gray!20}
\setbeamercolor{subsection in head/foot}{fg=Green!70!Black, bg=Gray!10}
\setbeamercolor{alerted text}{fg=\alertcolor}
\setbeamerfont{alerted text}{series=\bf}
\setbeamertemplate{enumerate items}[default]
\setbeamercolor{enumerate item}{fg=\normcolor}

\setbeamertemplate{footline} % To remove the footer line in all slides uncomment this line
%\setbeamertemplate{footline}[page number] % To replace the footer line in all slides with a simple slide count uncomment this line

\setbeamertemplate{navigation symbols}{} % To remove the navigation symbols from the bottom of all slides uncomment this line
}

\newcommand{\alertbox}[1]{\tcbox[on line, colframe=\alertcolor, colback=White, left=2pt,right=2pt,top=2pt,bottom=2pt]{\usebeamercolor*{normal text}#1}}


\newcommand{\startstu}{\setbeamercolor{normal text}{fg=\studentcolor}\usebeamercolor*{normal text}\setbeamercolor{enumerate item}{fg=\studentcolor}\usebeamercolor*{enumerate item}}
\newcommand{\stopstu}{\setbeamercolor{normal text}{fg=\normcolor}\usebeamercolor*{normal text}\setbeamercolor{enumerate item}{fg=\normcolor}\usebeamercolor*{enumerate item}}

\newcommand{\takenote}[1]{ \begin{collect}{notes}{}{}{}{}  #1  \end{collect}  \addtocounter{notestaken}{1}} %\ifthenelse{\value{notestaken}>0}{\hrulefill\\}{}

\makeatletter
\newcommand{\cover}{\alt{\beamer@makecovered}{\beamer@fakeinvisible}}
\newcommand{\ucover}[1]{\iftoggle{full}{}{\beamer@endcovered} \stopstu #1\startstu \iftoggle{full}{}{\beamer@startcovered} }
%\newcommand{\ucover}[1]{\beamer@endcovered \stopstu #1\startstu \beamer@startcovered }
\makeatother

\newcommand{\skippause}{ \addtocounter{beamerpauses}{-1}}
\newcommand{\blockpres}{ \skippause \pause }

\newcommand{\studentify}[1]{\startstu #1  \stopstu }
\newcommand{\student}[1]{\iftoggle{full}{ \pause  \studentify{#1} }{\iftoggle{covered}{\studentify{#1}}{\cover{  #1 }}}}
\newcommand{\cstudent}[1]{\student{\begin{center} #1 \end{center}}}
\newcommand{\fullonly}[1]{\iftoggle{full}{ #1}{}}
\newcommand{\presentonly}[1]{\iftoggle{presentable}{ #1}{}}

\usepackage{xparse}
\usepackage{xifthen}

% shortcuts for commonly-used presentation elements
%\NewDocumentCommand{\slide}{o m}
% {\IfValueTF{#1}{\begin{frame}[t]{#1}}{\begin{frame}[t]} #2 \end{frame}}

\newtoggle{iscovered}

\newcommand{\slide}[2][]{%
%\setcounter{notestaken}{0}
\takenote{#2} 
%\ifthenelse{\equal{#1}{}}{\begin{frame}[t]}{\begin{frame}[t]{#1}} #2 \ifthenelse{\value{notestaken}>0}{ \note{\includecollection{notes}}}{} \end{frame}%
\ifthenelse{\equal{#1}{}}{\begin{frame}[t]}{\begin{frame}[t]{#1}} #2 \iftoggle{covered}{\settoggle{iscovered}{true}}{\settoggle{iscovered}{false}}  \note{ \iftoggle{iscovered}{}{\settoggle{covered}{true}} #2 \iftoggle{iscovered}{}{\settoggle{covered}{false}} } \end{frame}%
%\setcounter{notestaken}{0}
}
\newcommand{\defn}[2][]{%
 \setcounter{listcounter}{0}%
\ifthenelse{\equal{#1}{}}{\begin{block}{Definition}}{\begin{block}{#1 :}}%
 #2 \vspace{0.25em} \ifthenelse{\value{listcounter}>0}{\skippause}{} \pause \end{block}%
}



\newcommand{\arr}[2]{\begin{array}{#1}#2\end{array}}
\newcommand{\mat}[2]{\left[\arr{#1}{#2}\right]}
\newcommand{\carray}[1]{\arr{c}{#1}}
\newcommand{\larray}[1]{\arr{l}{#1}}
\newcommand{\rarray}[1]{\arr{r}{#1}}
\newcommand{\colvec}[1]{\mat{c}{#1}}

\newcommand{\itmz}[1]{\addtocounter{listcounter}{1} \begin{itemize}#1 \end{itemize} }
\newcommand{\subitem}[1]{\addtocounter{listcounter}{1} \begin{itemize} \item #1 \end{itemize}}
%
\newcommand{\enum}[1]{\addtocounter{listcounter}{1} \begin{enumerate} #1  \end{enumerate}  }


\newcommand{\algnlbl}[1]{\begin{align}#1  \end{align}} 
\newcommand{\algn}[1]{\begin{align*}#1  \end{align*}} 
\newcommand{\lgn}[1]{ \action<+->{#1} }
\newcommand{\slgn}[1]{\iftoggle{full}{\action<+->{ \startstu #1 \stopstu}}{ \cover{ #1 } } \takenote{$#1$}}

\newcommand{\chckmrk}{\alert{\checkmark}}

\usepackage{pifont}
\newcommand{\xmark}{\alert{\text{\large \ding{55}}}}

\newcommand{\return}[0]{\raisebox{.5ex}{\rotatebox[origin=c]{180}{$\Lsh$}}}
\usepackage{pbox}
%\newcommand{\ex}[1]{\rotatebox[origin=c]{10}{\uline{ex}}:$\;$\pbox[t][][b]{0.9\linewidth}{#1}}
\newcommand{\ex}[1]{\uline{ex}:$\;$\pbox[t][][t]{0.9\linewidth}{#1}}
\newcommand{\eg}[1]{e.g.,$\;$\pbox[t][][t]{0.9\linewidth}{#1}}
\newcommand{\tikzplot}[8][]{%
\begin{tikzpicture}

\begin{scope}[]%
\clip(-#2,-#4) rectangle (#3,#5);%
#8%
\end{scope}%
\laxis{#2}{#3}{#4}{#5}{#6}{#7}%
#1
\end{tikzpicture}%
}


\newcommand{\cancelslide}[1]{%
\begingroup%
\setbeamertemplate{background canvas}{%
\begin{tikzpicture}[remember picture,overlay]%
\draw[line width=2pt,red!60!black] %
  (current page.north west) -- (current page.south east);%
\draw[line width=2pt,red!60!black] %
  (current page.south west) -- (current page.north east);%
\end{tikzpicture}}%
#1%
\endgroup%
}
\renewcommand{\CancelColor}{\color{red}}
\newcommand{\twocols}[3][0.5]{\begin{columns}\begin{column}{#1\textwidth}#2\end{column}\hspace{1em}\vrule{}\hspace{1em}\begin{column}{#1\textwidth}#3\end{column}\end{columns}}

\newcommand{\twomini}[5][1]{\calculatespace \begin{minipage}[t]{\columnwidth}\begin{minipage}[][#1\contentheight][t]{#2\columnwidth}#4\end{minipage}\hfill\begin{minipage}[][#1\contentheight][t]{#3\columnwidth}#5\end{minipage}\end{minipage}}

\newcommand{\threemini}[7][1]{\calculatespace \begin{minipage}[t]{\columnwidth}\begin{minipage}[][#1\contentheight][t]{#2\columnwidth}#5\end{minipage}\hfill\begin{minipage}[][#1\contentheight][t]{#4\columnwidth}#6\end{minipage}\hfill\begin{minipage}[][#1\contentheight][t]{#3\columnwidth}#7\end{minipage}\end{minipage}}


\newcounter{listcounter}
\setcounter{listcounter}{0}



\newif\ifsidebartheme
\sidebarthemetrue

\newdimen\contentheight
\newdimen\contentwidth
\newdimen\contentleft
\newdimen\contentbottom
\makeatletter
\newcommand*{\calculatespace}{%
\contentheight=\paperheight%
\ifx\beamer@frametitle\@empty%
    \setbox\@tempboxa=\box\voidb@x%
  \else%
    \setbox\@tempboxa=\vbox{%
      \vbox{}%
      {\parskip0pt\usebeamertemplate***{frametitle}}%
    }%
    \ifsidebartheme%
      \advance\contentheight by-1em%
    \fi%
  \fi%
\advance\contentheight by-\ht\@tempboxa%
\advance\contentheight by-\dp\@tempboxa%
\advance\contentheight by-\beamer@frametopskip%
\ifbeamer@plainframe%
\contentbottom=0pt%
\else%
\advance\contentheight by-\headheight%
\advance\contentheight by\headdp%
\advance\contentheight by-\footheight%
\advance\contentheight by4pt%
\contentbottom=\footheight%
\advance\contentbottom by-4pt%
\fi%
\contentwidth=\paperwidth%
\ifbeamer@plainframe%
\contentleft=0pt%
\else%
\advance\contentwidth by-\beamer@rightsidebar%
\advance\contentwidth by-\beamer@leftsidebar\relax%
\contentleft=\beamer@leftsidebar%
\fi%
}
\makeatother

\iftoggle{dualscreen}{\setbeameroption{show notes on second screen=right}}{}


\begin{document}
\section{Lecture 21}
\subsection{Preamble}
\settoggle{covered}{true}
\slide[Recall: Linear IVPs]{\vspace{-0.5em}

Given the initial state of a system (e.g., mass-spring or electrical circuit),  predict its behaviour at later times.
\vfill
\enum{
\item Scalar ODEs: \vfill
\centerline{$\op{y(t)}=f(t)\quad $ with $\quad y(0)=y_0$ (and maybe $y'(0)=v_0$)}\vfill
\subitem{1$^{\rm st}$ order \student{$\Rightarrow$ solve by integrating factor}
\item 2$^{\rm nd}$ order \student{$\Rightarrow$ solve by undetermined coefficients} \subitem{\student{Equivalent to a system of 2 1$^{\rm st}$ order ODEs} }}
\vfill
\item Systems of 1$^{\rm st}$ order ODEs:
\[\dd{}{t} \vec{x} =\mathbf{A}(t)\vec{x}+\vec{f}(t) \quad\text{with}\quad\vec{x}(0)=\vec{x}_0\]
\vfill
General solution: $\vec{x}(t)=\mathbf{X}(t) \vec{c}+\mathbf{X}\int \mathbf{X}^{-1}\vec{f}(t)dt$ \hfill with\hfill $\vec{c}=\mathbf{X}^{-1}(0)\vec{x}_0$
}

}

\slide[Fundamental Matrices and Solution Bases]{\vspace{-1em}
\[\dd{}{t} \vec{x} =\mathbf{A}(t)\vec{x}+\vec{f}(t) \quad\text{with}\quad\vec{x}(0)=\vec{x}_0\]
\vfill
General solution: $\vec{x}(t)=\mathbf{X}(t) \vec{c}+\mathbf{X}\int \mathbf{X}^{-1}\vec{f}(t)dt$\hfill with\hfill $\vec{c}=\mathbf{X}^{-1}(0)\vec{x}_0$

\vfill

The $n$ columns of the fundamental matrix $\mathbf{X}$ are the L.I. solutions to \[\dd{}{t} \vec{x}_i =\mathbf{A}(t)\vec{x}_i \quad \text{for }i=1,\dots,n\]

\vfill
These \alert{fundamental solutions} form a basis for all homogeneous solutions.
\student{
\[\vec{x}_h = c_1\vec{x}_1+\dots+c_n\vec{x}_n \quad \text{with } c_i=\frac{ \vec{x}_i(0)\cdot\vec{x}_0}{||\vec{x}_i(0)||}\]

\vfill
Other solution bases can be found through linear combinations...
}
}

\slide[The Flow Matrix]{\vspace{-1em}
\[\dd{}{t} \vec{x} =\mathbf{A}(t)\vec{x} \quad\text{with}\quad\vec{x}(0)=\vec{x}_0\]
\vfill
$\vec{x}(t)=\mathbf{X}(t)\vec{c}$\quad with\quad$\vec{c}=\mathbf{X}^{-1}(0)\vec{x}_0\quad \Rightarrow\quad \vec{x}(t)=\mathbf{X}(t)\mathbf{X}^{-1}(0)\vec{x}_0$.

\vfill
\student{
Let $\mathbf{\Psi}(t)=\mathbf{X}(t)\mathbf{X}^{-1}(0)$ be a fundamental matrix in a new basis set \[\{\vec{\psi}_1(t),\dots,\vec{\psi}_n(t)\}\]

\vfill

Observe that at $t=0$ we have 
\[\mathbf{\Psi}(0)=\mathbf{X}(0)\mathbf{X}^{-1}(0) = \mathbf{I}\]
\vfill
So, a homogeneous solution with an initial condition $\vec{x}_h(0)=\vec{x}_0$ can be written as
\[\vec{x}_h(t)=\mathbf{\Psi}(t)\vec{x}_0\]

}
}
\slide[Orthonormal Solution Bases]{
\vspace{-0.75em}
Let $\mathbf{\Psi}(t)=\mathbf{X}(t)\mathbf{X}^{-1}(0)$ be a fundamental matrix in a new basis set \[\{\vec{\psi}_1(t),\dots,\vec{\psi}_n(t)\} \qquad \text{with } \quad  \mathbf{\Psi}(0)=\mathbf{I}\]\vspace{-0.75em}
\student{
\twomini[.3]{.5}{.5}{
Note:  $\vec{\psi}_i(0)=\underbrace{[0,\dots,1,\dots,0]^T}_{  i^{th} \text{ component is non-zero}}$
}{
$\Rightarrow \{\vec{\psi}_i(0)\}$ is an orthonormal basis, i.e., \[\vec{\psi}_i(0)\cdot \vec{\psi}_j(0)=\begin{cases} 1&i=j\\0&\text{otherwise}\end{cases}.\]
}


This means that  if we write \[\vec{x}(t) = c_1\vec{\psi}_1(t)+\dots+c_n\vec{\psi}_n(t) \qquad \text{with }\qquad \vec{x}(0)=\vec{x}_0\]then $c_i=\vec{\psi}_i(0)\cdot\vec{x_0} =i^{\rm th}$ component of $\vec{x}_0$.
\vfill
Q:  How can we do this when $n\to\infty$?\vfill
\centerline{\ex{PDEs: $\op{u(x,t)}=0$ with $u(x,0)=u_0(x)$}}

}
}

\subsection{BVPS}
\slide[BVPs]{

Before we solve PDEs, we need to discuss ODE boundary value problems.
\vfill
Suppose you have data about an ODE solution at  $t=0$ and $t=T$.

\[ ay\pp +by\p+cy=f(t), \qquad \student{ \underbrace{ \larray{y(0)=y_0\\y(T)=y_T} \quad\text{\uline{or}} \quad \larray{y\p(0)=v_0\\y\p(T)=v_T}}_{\text{boundary conditions (BCs)}}}\]
\vfill
\student{\centerline{We call these boundary value problems (BVPs).}}

}

\slide[Periodic BVPs]{
In cases where $f(t)$ is a \uline{periodic function with period $T$}, i.e., \student{\[f(t+T)=f(t) \qquad \forall t,\]} the solution $y(t)$ eventually becomes $T$-periodic.
\vfill
We can find the long-term periodic solution by solving an ODE with \uline{periodic boundary conditions}, given by \vfill
\student{\[ ay\pp +by\p+cy=f(t), \qquad \underbrace{ \larray{y(0)=y(T)\\y\p(0)=y\p(T)}}_{2^{nd} \text{ order} \Rightarrow 2 \text{ BCs}} \]}
\vfill
}
\subsection{Periodic BVPs}
\slide{\ex{$y\pp + y=\cos(2\pi t), \qquad \larray{y(0)=y(1)\\y\p(0)=y\p(1)}$    }
\student{\algn{y(t)&=y_h + y_p\qquad
\arr{rl}{y_h&=c_1\cos(t) + c_2 \sin(t)\\
\text{geuss: } y_p&=A\cos(2\pi t) + B\sin(2\pi t)}\\
\text{\uline{M. U. C.:} }&A=\frac{1}{1-4\pi^2}, \; B=0\\
y(t)&=c_1\cos(t) + c_2 \sin(t) + \frac{1}{1-4\pi^2}\cos(2\pi t)\intertext{Boundary Conditions:}
y(0) = c_1 + \cancel{\frac{1}{1-4\pi^2}}  &= y(1) = c_1 \cos(1) + c_2 \sin(1) +\cancel{ \frac{1}{1-4\pi^2}}\\
 c_1  &= c_1 \cos(1) + c_2 \sin(1) \\\\
y\p(0) = c_2 &= y\p(1)= -c_1\sin(1) + c_2\cos(1)\\
 c_2 &=  -c_1\sin(1) + c_2\cos(1)
}
}

}

\slide{
$\begin{array}{ll}  c_1  &= c_1 \cos(1) + c_2 \sin(1) \\  c_2  &=  -c_1\sin(1) + c_2\cos(1) \end{array}$\vspace{-3em}
\student{
\algn{c_2(1-\cos(1)) & =- c_1\sin(1)& c_1&=c_1\cos(1) -\frac{\overbrace{\sin^2(1)}^{1-\cos^2(1)}}{1-\cos(1)}c_1\\
c_2=-&\frac{\sin(1)}{1-\cos(1)}c_1&c_1 &= c_1\paren{\cos(1) -\frac{(1+\cos(1))\cancel{(1-\cos(1)})}{\cancel{1-\cos(1)}}} \\
&&c_1&=c_1\paren{\cancel{\cos(1)}-\paren{1+\cancel{\cos(1)}}}\\
&&c_1&=-c_1 \qquad \Rightarrow c_1=c_2=0\\
&&\Aboxed{y(t) &= \frac{1}{1-4\pi^2}\sin(2\pi t)}
 }\vfill
 \uline{Alternative approach:}\vfill
Notice that  $y_h(t)=y_h(t+2\pi)$ and $y_p(t)=y_p(t+1)$. \vfill Since the BCs require solutions with period 1, we know the homogeneous part of the solution is zero.
}
}


\slide{
\ex{$y\pp + y=f(t), \qquad \larray{y(0)=y(1)\\y\p(0)=y\p(1)}$ \hfill with $f(t+1)=f(t)$ }
\vfill

\student{ Due to its periodicity, we can express $f(t)$ as \[f(t) = \frac{1}{2} a_0 + \sum_{n=1}^\infty a_n \cos(2n\pi t) + b_n \sin(2n\pi t)\]
This is called the \alert{Fourier Series} of the function $f(t)$, the coefficients $a_n$ and $b_n$ are called \alert{Fourier coefficients}.
\vfill
The coefficients are obtained by taking the \uline{inner product} of the function $f(t)$ and the Fourier basis\[\left\{ \cos( 2n\pi t),\; \sin(2n \pi t) \right\} \qquad n=0,\dots,\infty\]
\vfill

}

}

\settoggle{covered}{false}
\slide{
$y\pp + y =  \frac{1}{2} a_0 + \sum_{n=1}^\infty a_n \cos(2n\pi t) + b_n \sin(2n\pi t),  \qquad \larray{y(0)=y(1)\\y\p(0)=y\p(1)}$. 
\[\text{\uline{Guess:} } y(t)=  \sum_{n} y_n(t) = A_0 +\sum_{n=1}^\infty A_n \cos(2n\pi t) + B_n \sin(2n \pi t) \]

Apply M.U.C. term-by-term for the different values of $n$. \vfill
\student{
For $n\neq0$, the $n^{th}$ particular solution is \algn{y_n &=  A_n \cos(2n\pi t) + B_n \sin(2n\pi t) \\ y_n\pp &= -4n^2\pi^2A_n \cos(n\pi t) -4n^2\pi^2 B_n \sin(n\pi t)}
\vspace{-3em}
\algn{\text{\uline{ODE}: }\quad y_n\pp+y_n &=a_n \cos(2n\pi t) + b_n \sin(2n\pi t)\\\\
A_n(1-4n^2\pi^2)  \cos(n\pi t) &+ B_n(1-4n^2\pi^2)  \sin(n\pi t)&A_n=\frac{a_n}{1-4n\pi^2} \\
&=  a_n \cos(n\pi t) + b_n \sin(n\pi t)& B_n=\frac{b_n}{1-4n\pi^2}  }
}

}

\slide{$y\pp + y =  \frac{1}{2} a_0 + \sum_{n=1}^\infty a_n \cos(2n\pi t) + b_n \sin(2n\pi t),  \qquad \larray{y(0)=y(1)\\y\p(0)=y\p(1)}$. 
\[\text{\uline{Guess:} } y(t)=  \sum_{n} y_n(t) = A_0 +\sum_{n=1}^\infty A_n \cos(2n\pi t) + B_n \sin(2n \pi t) \]

Apply M.U.C. term-by-term for the different values of $n$.\vfill
\student{For $n=0$, we have
\algn{y_0&=A, y_0\pp=0\\
A&=\frac12 a_0\\
y(t) &= \frac12 a_0 + \sum_{n=1}^\infty \frac{a_n}{1-4n\pi^2}  \cos(2n\pi t)  + \frac{b_n}{1-4n\pi^2} \sin(2n\pi t) }\vfill
Given a specific periodic function $f(t)$, we can find its Fourier coefficients $a_n$ and $b_n$ and use the BVP solution above.
}
}

\slide[Inner Products]{
Dot products are an example of an \uline{inner product} for Euclidean vector spaces.

\[\left< \vec{x},\vec{y}\right> = \sum _i x_i y_i\]

5 basic properties define an inner product: \href{https://en.wikipedia.org/wiki/Inner_product_space}{\emph{wikipedia}}
\vfill
\student{
To define inner products for function spaces, sums are replaced by integrals.
\vfill

For $T$-periodic functions $f$ and $g$ we define the following inner product:
 \[\left< f,g\right> = \frac{2}{T}\int_0^T f(t)g(t)dt\]
}

}
\subsection{Fourier Series and Basis}
\slide[Fourier Series]{\vspace{-1em}
Given any  periodic function $f(t)$ with period $T$, we can approximate $f(t)$ as a Fourier series
\[ f(t) \approx \frac{1}{2} a_0 + \sum_{n=1}^\infty a_n \cos\paren{\frac{2n\pi t}{T}} + b_n \sin\paren{\frac{2n\pi t}{T}} \]with
\algn{a_0&=\left< f(t), 1 \right> &&\student{=\frac2T\int_0^T f(t) dt}\\
a_n&=\left< f(t),  \cos\paren{\frac{2n\pi t}{T}} \right> && \student{=\frac2T\int_0^T f(t) \cos\paren{\frac{2n\pi t}{T}} dt}\\
b_n&=\left< f(t),  \sin\paren{\frac{2n\pi t}{T}} \right> && \student{=\frac2T\int_0^T f(t) \sin\paren{\frac{2n\pi t}{T}} dt}}
If $f(t)$ is a continuous function, then the approximation becomes an equality.
}



\slide[The Fourier Basis is Orthonormalized]{
\vfill
Consider $m$ and $n$ to be any two positive integers or zero, then we have\vfill
\algn{
\left< \cos\paren{\nicefrac{2n\pi t}{T}} ,  \sin\paren{\nicefrac{2m\pi t}{T}} \right>  &= 0 \qquad \forall m,n  \qquad \student{\text{(Orthogonality)}}\\\\
\left< \sin\paren{\nicefrac{2n\pi t}{T}} ,  \sin\paren{\nicefrac{2m\pi t}{T}} \right>  &= \left< \cos\paren{\nicefrac{2n\pi t}{T}} ,  \cos\paren{\nicefrac{2m\pi t}{T}} \right>\\
&=\begin{cases} 1  & \text{if } m=n\neq0   \qquad \student{\text{(Normalization)}} \\ 0  &\text{otherwise}\phantom{\neq0} \qquad \student{\text{(Orthogonality)}} \end{cases} }

\vfill
\student{The normalization condition is the reason for the factors of $\frac2T$ in front of the Fourier coefficient integrals.}
}

\end{document}